\documentclass[11pt]{preprint}

\setlength{\topmargin}{0mm} \setlength{\oddsidemargin}{0mm}
\setlength{\textwidth}{160mm} \setlength{\textheight}{215mm}

\usepackage{amssymb,amsmath,amscd,amsthm}

\title{Discrete Mathematics, Sect 002, 2016 Spring - Quiz 2}
\author{Name:}
\institute{Courant Institute of Mathematical Sciences, NYU}
\date{September 19, 2016}

\newtheorem*{proposition}{Proposition}



\begin{document}

\maketitle

This quiz is scheduled for 10 minutes. No outside notes or calculators are permitted. To get full credit  in all of the problems, use rigorous justification and unless otherwise indicated, make sure that your solution reads as a perfect English sentence. You should only assume the notion of integers, operations, order relations and geometrical objects as given. If you use a statement or a definition from the textbook, make sure to indicate it.
\vspace{0.2cm}



\begin{enumerate}
\item (10 points) Write down the truth table of the following Boolean expression and argue that it is logically equivalent to $x\vee y$.
\[
(\neg x)\to(x\vee y)
\]
\vspace{1cm}

\textit{This I leave to you.}

\vspace{1cm}
\item (10 points) A bookshelf fits $5$ out of my $10$ different books so I make a choice and put those there. How many possible ways can the shelf look like? (Here you need to explain yourself, try to keep it to the point and brief.)
\vspace{2cm}

\textit{For the leftmost book, there are $10$ choices, for the one to its right, we have $9$ choices and so on until for the rightmost book we have $6$ choices. Therefore there are $10\cdot...\cdot 6=30240$ ways that the shelf can look like.}

\end{enumerate}

\end{document}