\documentclass[11pt]{preprint}

\setlength{\topmargin}{0mm} \setlength{\oddsidemargin}{0mm}
\setlength{\textwidth}{160mm} \setlength{\textheight}{215mm}

\usepackage{amssymb,amsmath,amscd,amsthm}

\title{Discrete Mathematics, Sect 002, 2016 Spring - Quiz 8}
\author{Name:}
\institute{Courant Institute of Mathematical Sciences, NYU}

\newtheorem*{proposition}{Proposition}



\begin{document}

\maketitle

This quiz is scheduled for 15 minutes. No outside notes or calculators are permitted. To get full credit  in all of the problems, use rigorous justification and unless otherwise indicated, make sure that your solution reads as a perfect English sentence. You should only assume the notion of integers, operations, order relations and geometrical objects as given. If you use a statement or a definition from the textbook, make sure to indicate it.
\vspace{0.2cm}

\begin{enumerate}
\item(10 points) For each of the following, find all positive integers $N$ that make each of the equations true.
\begin{enumerate}
\item $N$ div $10=5$. $N=50,51,52,53,54,55,56,57,58,59$.
\item $N$ mod $10=5$. (You have infinitely many of them, just give a description.) $N=10k+5$ where $k$ is an arbitrary integer.
\end{enumerate}
\vspace{3cm}
\item (10 points)  Compute $gcd(72,45)$.
We write
\begin{align*}
72& = 1\cdot 45+27\\
45& = 1\cdot 27+ 18\\
27& = 1\cdot 18 + 9\\
18& = 2\cdot 9 + 0
\end{align*}
and thus $gcd(72,55)=9$.
\end{enumerate}
\end{document}