\documentclass[11pt]{preprint}

\setlength{\topmargin}{0mm} \setlength{\oddsidemargin}{0mm}
\setlength{\textwidth}{160mm} \setlength{\textheight}{215mm}

\usepackage{amssymb,amsmath,amscd,amsthm}

\title{Discrete Mathematics, Sect 002, 2016 Fall - Quiz 5}
\date{October 24, 2016}
\author{Name:}
\institute{Courant Institute of Mathematical Sciences, NYU}

\newtheorem*{proposition}{Proposition}



\begin{document}

\maketitle

This quiz is scheduled for 10 minutes. No outside notes or calculators are permitted. To get full credit  in all of the problems, use rigorous justification and unless otherwise indicated, make sure that your solution reads as a perfect English sentence. You should only assume the notion of integers, operations, order relations and geometrical objects as given. If you use a statement or a definition from the textbook, make sure to indicate it.
\vspace{0.2cm}

\begin{enumerate}
\item(10 points)Prove by the contrapositive method, that if $c$ is an odd integer then the equation $n^2+n-c=0$ has no integer solution for $n$.
\vspace{6cm}

HINT: Remember that the product of two consecutive integer is even.
\item (10 points) Let $n$ be an integer. Prove by contradiction that if $n^2$ is even, then $n$ is even.

HINT: You can use now that an integer is either even or odd.
\end{enumerate}
\end{document}