\documentclass[11pt]{preprint}

\setlength{\topmargin}{0mm} \setlength{\oddsidemargin}{0mm}
\setlength{\textwidth}{160mm} \setlength{\textheight}{215mm}

\usepackage{amssymb,amsmath,amscd,amsthm}
\usepackage{graphics}
\usepackage{tikz}

\def\enumb{\begin{enumerate}}
\def\enume{\end{enumerate}}
\def\itemb{\begin{itemize}}
\def\iteme{\end{itemize}}
\def\integers{\mathbb{Z}}

\def\multiset#1#2{\ensuremath{\left(\kern-.3em\left(\genfrac{}{}{0pt}{}{#1}{#2}\right)\kern-.3em\right)}}



\newtheorem{proposition}{Proposition}
\newtheorem{theorem}{Theorem}
\newtheorem*{solution}{Solution}

\title{Discrete Mathematics, 2016 Fall - Worksheet 17}
\author{Instructor: Zsolt Pajor-Gyulai, CIMS}
\date{November 9, 2016}



\begin{document}

\maketitle

In all of the above problems explain your answer in full English sentences.

\enumb
\item Please express the following permutations in disjoint cycle form.
\enumb
\item $\pi=\left[\begin{array}{cccccc}
1&2&3&4&5&6\\
2&3&4&5&6&1
\end{array}\right]$.

\begin{solution}
\[
(1 2 3 4 5 6)
\]
\end{solution}
\item $\sigma=\left[\begin{array}{cccccc}
1&2&3&4&5&6\\
2&4&6&1&3&5
\end{array}\right]$
\begin{solution}
\[
(1 2 4)(3 6 5)
\]
\end{solution}
\enume
\item Prove that in the cycle decomposition produced by the algorithm discussed on the slides, the resulting cycles are pairwise disjoint.

\begin{proof}
For the sake of contradiction, assume that some $s\in A$ that is not an element of some cycle $(t,\pi(t),\pi^{(2)}(t),\dots)$ but some $\pi^{(k)}(s)$ is. Let $(k)$ the smallest such iterate. Then $\pi^{(k-1)}(s)$ is not in the cycle. However, $\pi^{(k)}(s)$ being in the cycle implies that $\pi^{(m)}(t)=\pi^{(k)}(s)$ for some $m$. However. This means that if $a=\pi^{(k-1)}(s)$ and $b=\pi^{(m-1)}(t)$ then $a\neq b$ (since one of them is in the cycle, the other one isn't), but
\[
\pi(a)=\pi(b),
\]
contradicting the one-to-one-ness of $\pi$.

\end{proof}

\item How many permutations in $S_n$ have exactly one cycle?
\begin{solution}
Note that without loss of generality, we can assume that our cycle looks like
\[
(1 a_2 a_3 \dots a_n),
\]
because we can always cyclically 'rotate' a cycle. Now for $a_2 \dots a_n$, we can choose any arrangements of the remaining $n-1$ numbers, and thus the answer is $(n-1)!$.

\end{solution}


\item Let $\pi,\sigma$ be given by
\[
\pi=(1)(2,3,4,5)(6,7,8,9),\qquad\sigma=(1,3,5,7,9,2,4,6,8)
\]
Calculate the following
\enumb
\item $\pi\circ\sigma$
\[
(1,4,7,6,9,3,2,5,8)
\]
\item $\sigma\circ\pi$
\[
(1,3,6,9,8,2,5,4,7)
\]
\item $\pi^{-1}$
\[
(1)(2,5,4,3)(6,9,8,7)
\]
\item $\pi^{-1}\circ\pi$.
\[
id=(1)(2)(3)(4)(5)(6)(7)(8)(9)
\]
\enume

\item Write the following permutations as the composition of transpositions and determine whether the permutation is even or odd.
\enumb
\item $(1,3)(2,4,5)=(1,3)\circ(2,5)\circ (2,4)$

\item $(1,2,4,3)(5)=(1,3)\circ (1,4)\circ (1,2)$
\item $[(1,3)(2,4,5)]^{-1}=(1,3)(2,5,4)=(1,3)\circ (2,4)\circ (2,5)$.
\enume

\item Prove the following group facts: 
\enumb
\item If $(G,*)$ is a group and $g\in G$, then $(g^{-1})^{-1}=g$.

\begin{proof}
This follows from
\[
g^{-1}*g=g*g^{-1}=e
\]
\end{proof}
\item  If $(G,*)$ is a group with identity element $e$, then $e^{-1}=e$.

\begin{proof}
This is an immediate consequence of
\[
e*e=e
\]
\end{proof}

\item If $(G,*)$ is a group and $g,h\in G$, then $(g*h)^{-1}=h^{-1}*g^{-1}$.

\begin{proof}
Note that on one hand,
\[
h^{-1}*g^{-1}*g*h=h^{-1}*e*h=h^{-1}*h=e.
\]
On the other hand,
\[
g*h*h^{-1}*g^{-1}=g*e*g^{-1}=gg^{-1}=e,
\]
which proves the claim.
\end{proof}

\enume

\item Show that the alternating group $(A_n,\circ)$ is indeed a group.

\begin{proof}
First we have to show that $A_n$ is closed under $\circ$. This indeed holds as if $\tau,\sigma\in A_n$, then
\[
\tau=\tau_1\circ\dots\circ\tau_a,\qquad \sigma=\sigma_1\circ\dots\circ\sigma_b
\]
where $\tau_i$ and $\sigma_i$ are transpositions and $a$, and $b$ are even numbers. Then
\[
\tau\circ\sigma=\tau_1\circ\dots\circ\tau_a\circ \sigma_1\circ\dots\circ\sigma_b
\]
which means that $\tau\circ\sigma\in A_n$ as $a+b$ is even.

Associativity and the inverse element $e=id_{\{1,\dots,n\}}$ are inherited from $S_n$. To show that there are inverses, note that the inverse of $\tau\in A_n$ in $S_n$ is
\[
\tau^{-1}=\tau_a^{-1}\circ\dots\circ \tau_1^{-1}
\]
which is a decomposition into an even number of transpositions. Thus $\tau^{-1}\in A_n$.
\end{proof}
\enume
\end{document}