\documentclass[11pt]{preprint}

\setlength{\topmargin}{0mm} \setlength{\oddsidemargin}{0mm}
\setlength{\textwidth}{160mm} \setlength{\textheight}{215mm}

\usepackage{amssymb,amsmath,amscd,amsthm}
\usepackage{graphics}
\usepackage{tikz}

\def\enumb{\begin{enumerate}}
\def\enume{\end{enumerate}}
\def\itemb{\begin{itemize}}
\def\iteme{\end{itemize}}
\def\integers{\mathbb{Z}}

\def\multiset#1#2{\ensuremath{\left(\kern-.3em\left(\genfrac{}{}{0pt}{}{#1}{#2}\right)\kern-.3em\right)}}



\newtheorem{proposition}{Proposition}
\newtheorem{theorem}{Theorem}
\newtheorem*{solution}{Solution}

\title{Discrete Mathematics, 2016 Fall - Worksheet 17}
\author{Instructor: Zsolt Pajor-Gyulai, CIMS}
\date{November 9, 2016}



\begin{document}

\maketitle

In all of the above problems explain your answer in full English sentences.

\enumb
\item Please express the following permutations in disjoint cycle form.
\enumb
\item $\pi=\left[\begin{array}{cccccc}
1&2&3&4&5&6\\
2&3&4&5&6&1
\end{array}\right]$.


\item $\sigma=\left[\begin{array}{cccccc}
1&2&3&4&5&6\\
2&4&6&1&3&5
\end{array}\right]$

\enume
\item Prove that in the cycle decomposition produced by the algorithm discussed on the slides, the resulting cycles are pairwise disjoint.



\item How many permutations in $S_n$ have exactly one cycle?


\item Let $\pi,\sigma$ be given by
\[
\pi=(1)(2,3,4,5)(6,7,8,9),\qquad\sigma=(1,3,5,7,9,2,4,6,8)
\]
Calculate the following
\enumb
\item $\pi\circ\sigma$
\item $\sigma\circ\pi$
\item $\pi^{-1}$
\item $\pi^{-1}\circ\pi$.
\enume

\item Write the following permutations as the composition of transpositions and determine whether the permutation is even or odd.
\enumb
\item $(1,3)(2,4,5)$
\item $(1,2,4,3)(5)$
\item $[(1,3)(2,4,5)]^{-1}$.
\enume

\item Prove the following group facts: 
\enumb
\item If $(G,*)$ is a group and $g\in G$, then $(g^{-1})^{-1}=g$.
\item  If $(G,*)$ is a group with identity element $e$, then $e^{-1}=e$.
\item If $(G,*)$ is a group and $g,h\in G$, then $(g*h)^{-1}=h^{-1}*g^{-1}$.
\enume

\item Show that the alternating group $(A_n,\circ)$ is indeed a group.
\enume
\end{document}