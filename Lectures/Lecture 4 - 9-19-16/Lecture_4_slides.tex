\documentclass{beamer}

\mode<presentation>
{
  \usetheme{Frankfurt}
  \usecolortheme{orchid}
  \setbeamercovered{invisible}
  \setbeamertemplate{footline}[frame number]
}

\usepackage[english]{babel}
\usepackage[latin1]{inputenc}
\usepackage{times}
\usepackage[T1]{fontenc}
\usepackage{tikz}
\usepackage{array}
\usepackage{listings}

\def\blue{\color{blue}~}
\def\black{\color{black}~}
\def\bl[#1]#2{\begin{block}{#1}#2\end{block}}
\def\integers{\mathbb{Z}}
\def\enumb{\begin{enumerate}}
\def\enume{\end{enumerate}}
\def\itemb{\begin{itemize}}
\def\iteme{\end{itemize}}


\usepackage{remreset}
\makeatletter
\@removefromreset{subsection}{section}
\makeatother
\setcounter{subsection}{1}

\title{Discrete Mathematics, Section 002, Spring 2016}
\subtitle{Lecture 4: Factorials, Sets}

\author[Zsolt]{Zsolt Pajor-Gyulai \\ \texttt{zsolt@cims.nyu.edu}}
\date{September 19, 2016}

\pgfdeclareimage[height=1cm]{NYUlogo}{NYUlogo.jpg}

\institute[NYU] 
{
\normalsize Courant Institute of Mathematical Sciences
}
\titlegraphic{\pgfuseimage{NYUlogo}}

\begin{document}

\begin{frame}
  \titlepage
\end{frame}

\AtBeginSection[]
{
\begin{frame}
\frametitle{Outline}
\tableofcontents[currentsection]
\end{frame}}


\section{Factorials and products}

\begin{frame}{Factorials}
A special case of what we did last time:
\begin{block}{}
How many lists of length $n$ can we make using $n$ elements without repetition?
\end{block}
or alternatively
\begin{block}{}
How many ways can we order $n$ elements?\pause
\end{block}
\[
(n)_n=n(n-1)(n-2)\cdots 3\cdot 2\cdot 1
\]\pause
This occurs a lot, we call it \textbf{$n$-factorial} and denote it by $n!$.
\pause
\end{frame}
\begin{frame}
\itemb
\item Note that
\bl[]{
\[
n!=n\cdot (n-1)!
\]}
\item Also
\iteme
\bl[]{\vspace{-0.3cm}
\begin{align*}
(n)_k&=n(n-1)\dots (n-k+1)=\\
&=\frac{n(n-1)\dots 2\cdot 1}{(n-k)(n-k-1)\dots 2\cdot 1}=\frac{n!}{(n-k)!}
\end{align*}}

\end{frame}

\begin{frame}{Factorials}
Two special cases:
\begin{enumerate}
\item $1!=1$: Hardly suprising.\pause
\item $0!=1$:\pause
\begin{itemize}
\item How many list of length zero can we form out of zero elements?
\[
\textrm{One, the empty list.}
\]\pause
\item When $n=1$, $n!=n(n-1)!$ becomes $1=1\cdot 0!$\pause
\item At the end of the day we make the definition like this, because it's convenient.
\end{itemize}
\end{enumerate}
\end{frame}

\begin{frame}{Product notation}
Another way to write factorials:
\[
n!=\prod_{k=1}^nk
\]
\begin{itemize}
\item $\prod$ stands for product.
\item $k$ is a dummy variable, a place holder that ranges from the lower value to the upper value.
\end{itemize}\pause
Another example,
\[
\prod_{k=1}^5(2k+3)=(2\cdot 1+3)(2\cdot 2+3)(2\cdot 3 +3)(2\cdot 4 +3)(2\cdot 5+3)
\]
\end{frame}

\begin{frame}[fragile]{Product notation}
Interpretation as a for loop to compute
\[
\prod_{k=1}^{n}f_k
\]

\begin{lstlisting}[language=Python]
def evaluate():
    prod = 1
    for k in range(1,n+1):
        prod *= f_k
    
    return prod
\end{lstlisting}
	

\end{frame}

\begin{frame}{Product notation}
Further examples:
\begin{itemize}
\item Constant to the right of the product sign:
\[
\prod_{k=1}^n 2=2^n
\]\pause
\item Empty product:
\[
0!=\prod_{k=1}^0 k=1
\]
\end{itemize}
\end{frame}

\section{Introduction to sets, Subsets}

\begin{frame}
Last time we talked about lists.
\begin{block}{What was a list?}
A \textbf{list} is an ordered sequence of objects.
\end{block}\pause
\begin{block}{What is a set?}
A \textbf{set} is a repetition-free, unordered collection of objects.
\end{block}\pause
\begin{itemize}
\item An object is either a member of a set or not.\pause
\item There is no order to the members.\pause
\item Simplest way to specify: list elements.
\[
\left\{2,3,\frac{1}{2}\right\}\qquad\left\{3,\frac{1}{2},2\right\}\qquad\left\{2,2,3,\frac{1}{2}\right\}
\]\pause
\center{\alert{These are all the same set!}}
\end{itemize}
\end{frame}

\begin{frame}
For example, sets of numbers:
\begin{itemize}
\item Naturals: $\mathbb{N}$\pause
\item Integers: $\mathbb{Z}$\pause
\item Rationals: $\mathbb{Q}$
\end{itemize}\pause

\begin{block}{Membership in a set}
An object $x$ that belongs to a set $A$ is called an \textbf{element} of it.
\[
\textrm{Notation:}\qquad x\in A
\]
\end{block}
For example, $x\in\mathbb{Z}$ reads:
\begin{itemize}
\item "$x$ is a member/element of $\mathbb{Z}$"\pause
\item "$x$ is in $\mathbb{Z}$"\pause
\item "$x$ is an integer"
\end{itemize}
\end{frame}

\begin{frame}
\begin{block}{Cardinality of a set}
The \textbf{cardinality} of $A$, denoted by $|A|$, is the number of objects in the set.
\end{block}\pause
For example,
\[
\left|\left\{2,3,\frac{1}{2}\right\}\right|=3,\qquad |\mathbb{Z}|=\infty
\]\pause
A set $A$ with $|A|\in\mathbb{N}$ is \textbf{finite}, otherwise it's \textbf{infinite}.
\begin{block}{Empty set}
The \textbf{empty set} $\emptyset$ is a set with no elements.
\end{block}\pause
\begin{itemize}
\item The statement $x\in\emptyset$ is false.\pause
\item $|\emptyset|=0$
\end{itemize}
\end{frame}

\begin{frame}{Specifying sets}
\begin{itemize}
\item List the elements between curly braces.
\[
\{3,4,9\}, \{table, chair, lamp\}
\]\pause
Clumsy for large sets, and impossible for infinite sets.\pause
\item Set-builder notation:
\[
\{\textrm{dummy variable}: \textrm{conditions}\}
\]\pause
For example,
\[
\mathbb{N}=\{x: x\in\mathbb{Z}, x\geq 0\}
\]
This is a set of objects satisfying
\begin{itemize}
\item $x\in\mathbb{Z}$
\item $x\geq 0$
\end{itemize}
\end{itemize}
\end{frame}

\begin{frame}{Specifying sets}
\begin{itemize}
\item Alternative set-builder notation:
\[
\{\textrm{dummy variable}\in \textrm{set}: \textrm{conditions}\}
\]\pause
For example,
\[
\{x\in\mathbb{Z}: 2|x\}
\]
This is the set of those integers that are divisible by $2$.
\end{itemize}\pause

\center{Now practice these on the worksheet!}
\end{frame}

\begin{frame}{Equality of sets}
\begin{Definition}
Two sets are equal if they have exactly the same elements.
\end{Definition}\pause

 To show $A=B$, we have the following template:
\begin{block}{}
Let $A$ and $B$ be the sets.
\begin{itemize}
\item Suppose $x\in A$.\dots Therefore $x\in B$.
\item Suppose $x\in B$.\dots Therefore $x\in A$.
\end{itemize}
Therefore $A=B$.
\end{block}
\end{frame}

\begin{frame}[t]
\begin{block}{Proposition}
The following two sets are equal:
\[
E=\{x\in\mathbb{Z}: x\textrm{ is even}\},
\]
\[
F=\{x\in\mathbb{Z}: x=a+b\textrm{ where $a$ and $b$ are both odd}\}.
\]
\end{block}

\begin{proof}
\uncover<2->{Let $E$ and $F$ as in the statement of the proposition. We seek to prove that $E=F$.}

\uncover<2->{~~~~Suppose $x\in E.$}\only<3>{\color{blue}}\uncover<3->{ Therefore $x$ is even, hence $2|x$ and so $x=2y$ for some $y\in\mathbb{Z}$.}\only<3>{\black} \only<4>{\color{blue}}\uncover<4->{ Note that $2y+1$ and $-1$ are both odd and that $x=2y=(2y+1)+(-1)$.}\only<4>{\color{black}}\uncover<2->{ Therefore}\only<3>{\color{blue}}\uncover<3->{ $x$ is the sum of two odd numbers and so}\only<3>{\color{black}}\uncover<2->{ $x\in F$.}

\uncover<2->{~~~~Suppose $x\in F.$}\only<5>{\color{blue}}\uncover<5->{ Therefore $x$ is the sum of two odd integers. It was shown in Exercise 5.1 that $x$ is then even.}\only<5>{\color{black}}\uncover<2->{ Therefore $x\in E$.}
\end{proof}
\end{frame}

\begin{frame}{Subsets}
\begin{definition}
Suppose $A$ and $B$ are sets. We say that $A$ is a \textbf{subset} of $B$ provided every element of $A$ is also an element of $B$. We denote this by $A\subseteq B$.
\end{definition}\pause
For example,
\begin{itemize}
\item $\{1,2,3\}\subseteq \{1,2,3,4\}$\pause
\item $A\subseteq A$ for every set $A$\pause
\item $\emptyset\subseteq A$.
\item If $B\subseteq A$ and $B\neq A,\emptyset$, we call it a \textbf{proper subset}.
\end{itemize}\pause

\begin{block}{}
To show that $A\subseteq B$:

~~~~Let $x\in A,\dots$ Therefore $x\in B$.
\end{block}
\end{frame}

\begin{frame}{Subsets}
\begin{definition}
A list of three integers $(a,b,c)$ is called a \textbf{Pythagorean triple} provided $a^2+b^2=c^2$.
\end{definition}\pause

\bl[Proposition]{
Let $P$ be the set of Pythagorean triples.
\[
P=\{(a,b,c): a,b,c\in\mathbb{Z}\textrm{ and } a^2+b^2=c^2\}
\]
and
\begin{align*}
T=\{(p,q,r): &p=x^2-y^2, q=2xy,\\
&\textrm{ and } r=x^2+y^2\textrm{ where } x,y\in\mathbb{Z}\}
\end{align*}
Then $T\subseteq P$.
}
\end{frame}

\begin{frame}
\bl[]{
\vspace{-0.3cm}
\begin{align*}
T=\{(p,q,r): &p=x^2-y^2, q=2xy,\\
&\textrm{ and } r=x^2+y^2\textrm{ where } x,y\in\mathbb{Z}\}
\end{align*}}

For example $x=3$, $y=2$,
\[
p=x^2-y^2=9-4=5,\qquad q=2xy=12,\qquad r=x^2+y^2=13
\]
and $(p,q,r)\in P$.
\end{frame}

\begin{frame}
\bl[]{
\vspace{-0.3cm}
\begin{align*}
T=\{(p,q,r): &p=x^2-y^2, q=2xy,\\
&\textrm{ and } r=x^2+y^2\textrm{ where } x,y\in\mathbb{Z}\}
\end{align*}}
\begin{proof}
Let $P$ and $T$ as in the statement of the proposition.

~~~~Let $(p,q,r)\in T$.\only<2>{\color{blue}}\uncover<2->{ Therefore there are integers $x$ and $y$ such that $p=x^2-y^2$, $q=2xy$, and $r=x^2+y^2$.}\only<3>{\color{blue}}\uncover<3->{ Note that $p,q, r$ are integers because $x$ and $y$ are integers.}\only<4>{\color{blue}}\uncover<4->{We calculate
\begin{align*}
p^2+q^2&=(x^2-y^2)^2+(2xy)^2=(x^4-2x^2y^2+y^4)+4x^2y^2=\\
&=x^4+2x^2y^2+y^4=(x^2+y^2)^2=r^2
\end{align*}}
 \color{black}Therefore $(p,q,r)\in P$.
\end{proof}
\end{frame}

\begin{frame}{Caveats}
We have to be careful to make the distinction between
\bl[]{
\[
x \qquad\textrm{vs.}\qquad \{x\}
\]}\pause
and
\bl[]{
\[
\in\qquad\textrm{vs.}\qquad \subseteq
\]}\pause
For example,
\itemb
\item  $x\in\{x\}$
\item But $x\subseteq\{x\}$ or $x=\{x\}$ are incorrect.
\iteme
\end{frame}

\begin{frame}
\bl[Question]{
How many subsets does a set have?}\pause

For example, what about $A=\{1,2,3\}$?
\[
\begin{tabular}{|c|c|c|}
\hline
Number of elements&Subsets&Number\\
\hline
0&$\emptyset$&1\\
1&\{1\},\{2\},\{3\}&3\\
2&\{1,2\},\{1,3\},\{2,3\}&3\\
3&\{1,2,3\}&1\\
\hline
\end{tabular}
\]
Total: $8$.\pause
\bl[Alternatively]{
For every element, we have two choices independently of each other: include/not include.
}
Therefore $|A|=2^3=8$.
\end{frame}

\begin{frame}
\begin{theorem}
Let $A$ be a finite set. The number of subsets of $A$ is $2^{|A|}$.
\end{theorem}\pause

\begin{proof}
Let $A$ be a finite set and let $n=|A|$. Let the $n$ elements of $A$ be named $a_1,a_2,\dots,a_n$. To each subset $B$ of $A$, we can associate a list of length $n$; each element of the list is one of the words "yes" or "no". The $k$th element of the list is "yes" precisely when $a_k\in B$. This establishes a one to one correspondence with $yes/no$ lists of length $n$. The number of such lists is $2^n$, so the number of subsets of $A$ is $2^n$ where $n=|A|$.
\end{proof}\pause
\center{This is a so-called \textbf{bijective} proof.}
\end{frame}

\begin{frame}{Power set}
\begin{definition}
Let $A$ be a set. The \textbf{power set} of $A$ is the set of all subsets of $A$. We denote it by $2^A$.
\end{definition}\pause
For example, 
\[
2^{\{1,2,3\}}=\left\{\emptyset,\{1\},\{2\},\{3\},\{1,2\},\{1,3\},\{2,3\},\{1,2,3\}\right\}
\]\pause
The notation is created so that
\[
|2^A|=2^{|A|}
\]
\end{frame}

\end{document}