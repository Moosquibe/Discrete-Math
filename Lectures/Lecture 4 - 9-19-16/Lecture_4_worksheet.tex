\documentclass[11pt]{preprint}

\setlength{\topmargin}{0mm} \setlength{\oddsidemargin}{0mm}
\setlength{\textwidth}{160mm} \setlength{\textheight}{215mm}

\usepackage{amssymb,amsmath,amscd,amsthm}

\def\enumb{\begin{enumerate}}
\def\enume{\end{enumerate}}
\def\itemb{\begin{itemize}}
\def\iteme{\end{itemize}}
\def\integers{\mathbb{Z}}



\newtheorem{proposition}{Proposition}

\title{Discrete Mathematics, 2016 Spring - Worksheet 4}
\author{Instructor: Zsolt Pajor-Gyulai, CIMS}



\begin{document}

\maketitle

In all of the above problems explain your answer in full English sentences.

\begin{enumerate}

\item Solve the equation $n!=720$ for $n$.

\item There are six different French books, eight different Russian books and five different Spanish books.

\begin{enumerate}
\item In how many different ways can these books be arranged on a bookshelf?
\item In how many different ways can these books be arranged if all books in the same language are grouped together?
\end{enumerate}


\item Calculate the following products:
\begin{enumerate}
\item $\prod_{k=1}^4(2k+1)$
\item $\prod_{k=1}^n\frac{1}{k}$
\end{enumerate}

\item Can factorial be extended to negative integers? Think about the formula $n!=n(n-1)!$. What would be the value of $(-1)!$?

\item Write out the following sets by listing their elements between curly braces and find their cardinality.
\begin{enumerate}
\item $\{x\in\mathbb{N}: x\leq 10$ and $3|x\}$
\item $\{x\in\mathbb{Z}: x$ is a prime and $2|x\}$ 
\item $\{x\in\mathbb{Z}: 10|x$ and $x|100\}$
\item $\{x\in\mathbb{Z}: 1\leq x^2\leq 2\}$
\end{enumerate}

\item For each of the following sets, find a way to rewrite the set using set builder notation.
\begin{enumerate}
\item $\{1,2,3,4,5,6,7,8,9,10\}$
\item $\{-8,-6,-4,-2,0,2,4,6,8\}$
\item $\{1,4,9,16,25,36,49,64,81,100\}$
\end{enumerate}

\item
\enumb
\item Let $A=\{x\in\mathbb{Z}:4|x\}$ and let $B=\{x\in\integers:2|x\}$.  Prove that $A\subseteq B$.
\item Generalize the previous problem. Let $a,b\in\integers$ and let
\[
A=\{x\in\integers:a|x\},\qquad B=\{x\in\integers: b|x\}.
\]
Find and prove a necessary and sufficient conditions for $A\subseteq B$. In other words, find and prove a theorem of the form
\[
\textrm{"$A\subseteq B$ if and only if \textit{some condition involving $a$ and $b$.}"}
\]
\enume

\item Compute each of the following by writing either $\in$ or $\subseteq$ in place of $\bigcirc$.

\itemb
\item $2\bigcirc\{1,2,3\}$
\item $\{2\}\bigcirc\{1,2,3\}$
\item $\{2\}\bigcirc\{\{1\},\{2\},\{3\}\}$
\item $\emptyset\bigcirc\{1,2,3\}$
\item $\mathbb{N}\bigcirc\integers$
\item $\{2\}\bigcirc\integers$
\iteme


\item This problem is about power sets.
\enumb
\item Write out the elements and give the cardinality of the set $2^{\emptyset}$. (Hint: Start with the cardinality.)
\item Find the cardinality of the following sets.
\enumb
\item $2^{2^{\{1,2,3\}}}$
\item $\{x\in 2^{\{1,2,3,4\}}:|x|=1\}$
\enume 
\item Complete $\{2\}\bigcirc 2^{\integers}$ by a $\subseteq$ or $\in$.
\enume

\item (Russel's paradox) Consider the set of all sets $R$ that are not elements of themselves, i.e. $x\in R$ if $x$ is a set but $x\notin x$. Does $R$ contain itself as an element? The answer to this question signifies the breakdown of naive set theory which led to the development of axiomatic set theory. (You need to take a course in mathematical logic to learn more about this.)

\end{enumerate}

\underline{\textbf{Optional programming exercises (no credit)}}
\begin{enumerate}
\item[PE20)] Note that $10!=10\cdot 9\cdot 8\cdot...\cdot 2\cdot 1= 3628800$. Find the sum of the digits in the number $100!$.
\item[PE34)] 145 is a curious number as $1!+4!+5!=1+24+120=145$. Find the sum of all numbers which are equal to the sum of the factorial of their digits.
\end{enumerate}
\end{document}