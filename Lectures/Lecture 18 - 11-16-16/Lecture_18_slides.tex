\documentclass{beamer}

\mode<presentation>
{
  \usetheme{Frankfurt}
  \usecolortheme{orchid}
  \setbeamercovered{invisible}
  \setbeamertemplate{footline}[frame number]
}

\usepackage[english]{babel}
\usepackage[latin1]{inputenc}
\usepackage{times}
\usepackage[T1]{fontenc}
\usepackage{tikz}
\usepackage{array}
\usepackage{cancel}


\usetikzlibrary{shapes,backgrounds}

\def\multiset#1#2{\ensuremath{\left(\kern-.3em\left(\genfrac{}{}{0pt}{}{#1}{#2}\right)\kern-.3em\right)}}

\def\blue{\color{blue}~}
\def\black{\color{black}~}
\def\bl[#1]#2{\begin{block}{#1}#2\end{block}}
\def\integers{\mathbb{Z}}
\def\enumb{\begin{enumerate}}
\def\enume{\end{enumerate}}
\def\itemb{\begin{itemize}}
\def\iteme{\end{itemize}}
\def\div{~\textrm{div}~}
\def\mod{~\textrm{mod}~}


\usepackage{remreset}
\makeatletter
\@removefromreset{subsection}{section}
\makeatother
\setcounter{subsection}{1}

\title{Discrete Mathematics, Section 001, Fall 2016}
\subtitle{Lecture 18: First steps in number theory.}
\date{November 16, 2016}

\author[Zsolt]{Zsolt Pajor-Gyulai \\ \texttt{zsolt@cims.nyu.edu}}

\pgfdeclareimage[height=1cm]{NYUlogo}{NYUlogo.jpg}

\institute[NYU] 
{
\normalsize Courant Institute of Mathematical Sciences
}
\titlegraphic{\pgfuseimage{NYUlogo}}

\begin{document}

\begin{frame}
  \titlepage
\end{frame}

\AtBeginSection[]
{
\begin{frame}
\frametitle{Outline}
\tableofcontents[currentsection]
\end{frame}}

\section{Dividing}

\begin{frame}{Division with remainder}
\bl[Theorem]{Let $a,b\in\mathbb{Z}$ with $b\neq 0$. There exist integers $q$ and $r$ such that
\[
a=\color{red} q\color{black}b+\color{red}r\color{black}\qquad 0\leq r< |b|
\]
Moreover, there is only one such pair of integers $(q,r)$.}
\[
q:\textrm{quotient}\qquad r: \textrm{remainder}
\]

\bl[Example]{Let $a=-37$ and $b=5$. Then $q=-8$ and $r=3$ because
\[
-37=\color{red}-8\color{black}\times 5+\color{red}3\color{black}\qquad\textrm{and }\qquad 0\leq 3<5.
\]\vspace{-0.5cm}}
\end{frame}

\begin{frame}{Division with remainder}
\bl[Theorem]{Let $a,b\in\mathbb{Z}$ with $b\neq 0$. There exist integers \color{red}$q$\color{black}~ and \color{red}$r$\color{black}~ such that
\[
a=\color{red} q\color{black}b+\color{red}r\color{black}\qquad 0\leq r<|b|
\]
Moreover, there is only one such pair of integers $(q,r)$.}
Things to prove:
\itemb
\item There is such a pair $(\color{red}q\color{black},\color{red}r\color{black})$:
\itemb
\item $a=\color{red}q\color{black}b+\color{red}r\color{black}$
\item $0\leq\color{red} r\color{black}<|b|$
\iteme 
\item There is at most one such pair.
\iteme
\vspace{0.3cm}
\underline{Idea of the proof for $b>0$:} Keep substracting multiples of $b$ from $a$, then the smallest natural number we can get like this will be \color{red}$r$\color{black}:
\[
a-\color{red}q\color{black}b=\color{red}r\color{black}
\]

\end{frame}

\begin{frame}{Existence proof when $b>0$}
\bl[]{
 There is a pair $(\color{red}q\color{black},\color{red}r\color{black})$:
\itemb
\item $a=\color{red}q\color{black}b+\color{red}r\color{black}$
\item $0\leq \color{red}r\color{black}<b$
\iteme }
Let
\[
B=\{a-bk: k\in\mathbb{Z}, a-bk\geq 0\}\subseteq \mathbb{N}
\]
and note that $B\neq\emptyset$ as
\enumb
\item if $a\geq 0$, then $a\in B$ (choose $k=0$),
\item if $a<0$, then choose $k<\frac{a}{b}$.
\enume
Thus, the Well-Ordering Principle states that there is a least element $\color{red}r\color{black}\in B$. Since $\color{red}r\color{black}\in B$, there is a $\color{red}q\color{black}\in\mathbb{Z}$ such that
\[
\color{red}r\color{black}=a-b\color{red}q\color{black}
\]
Thus $a=\color{red}q\color{black}b+\color{red}r\color{black}$ and $\color{red}r\color{black}\geq 0$. It remains to show that $\color{red}r\color{black}<b$. 

  [...]
\end{frame}

\begin{frame}
{Existence proof when $b>0$}
\bl[]{
 There is a pair $(\color{red}q\color{black},\color{red}r\color{black})$:
\itemb
\item $a=\color{red}q\color{black}b+\color{red}r\color{black}$
\item $0\leq \color{red}r\color{black}<b$
\iteme }
[...]

For the sake of contradiction, suppose that $\color{red}r\color{black}\geq b$. Then
\[
a-\color{red}q\color{black}b=\color{red}r\color{black}\geq b
\]
and therefore 
\[
r'=\color{red}r\color{black}-b=(a-\color{red}q\color{black}b)-b=a-(\color{red}q\color{black}+1)b\geq 0.
\]
This implies $r'\in B$, but $r'<\color{red}r\color{black}$ and $\color{red}r\color{black}$ was the least element of $B$.$~~\Rightarrow\Leftarrow~$. This finishes the existence proof.
\end{frame}

\begin{frame}{Uniqueness proof}
\bl[]{There is at most one pair $(\color{red}q\color{black},\color{red}r\color{black})$ such that
\itemb
\item $a=\color{red}q\color{black}b+\color{red}r\color{black}$
\item $0\leq \color{red}r\color{black}<|b|$
\iteme}

Suppose, for the sake of contradiction, that there are two different pairs of numbers $(q,r)$ and $(q',r')$ that satisfies the conditions; that is
\[
\begin{array}{cc}
a=qb+r&0\leq r< |b|\\
a=q'b+r'&0\leq r'<|b|
\end{array}
\]
Combining these
\[
qb+r=q'b+r'\qquad\Rightarrow\qquad  r-r'=(q'-q)b.
\]
and therefore $b|r-r'$. But $0\leq r,r'<|b|$ and therefore
\[
r=r'.
\]\vspace{-0.6cm}

[...]
\end{frame}
\begin{frame}{Uniqueness proof}
\bl[]{There is at most one pair $(q,r)$ such that
\itemb
\item $a=qb+r$
\item $0\leq r<|b|$
\iteme}
[...]

Thus
\[
qb+r=a=q'b+r'=q'b+r\qquad\Rightarrow\qquad qb=q'b
\]
and since $b>0$, this implies $q=q'$. This means that 
\[
(q,r)=(q',r')
\]
and therefore the two pairs weren't different. $~\Rightarrow\Leftarrow~$. Therefore, the quotient and remainder are unique. This finishes the proof of the theorem. \qed
\end{frame}

\begin{frame}{A simple corollary}
\bl[Corollary]{Every integer is either even or odd, but not both}
\begin{proof}
Let $n\in\mathbb{Z}$. We can find $q,r\in\mathbb{Z}$ such that $n=2q+r$ where $r=0,1$. If $r=0$ then $n$ is even and if $r=1$ then $n$ is odd.
\end{proof}
\end{frame}

\begin{frame}{Div and Mod}
\bl[Definition]{
Let $a,b\in\mathbb{Z}$ with $b>0$ and let $q,r\in\mathbb{Z}$ be the unique integers such that $a=qb+r$ and $0\leq r<b$. Then we say
\[
a~\textrm{div}~b=q,\qquad a~\textrm{mod}~ b=r.
\]}
For example,
\[
\begin{array}{cc}
11\div 3 =3&11\mod 3=2\\
23\div 10=2&23\mod 10=3\\
-37\div 5=-8&-37\mod 5=3
\end{array}
\]
\textbf{Q:} What is the connection to the earlier definition of $\mod$?
\bl[Proposition]{
Let $a,b,n\in\mathbb{Z}$ with $n>0$. Then
\[
a\equiv b~~ (\textrm{mod } n)\qquad\Leftrightarrow\qquad a\mod n=b\mod n.
\]
}
\end{frame}

\section{Greatest Common Divisor}

\begin{frame}{Definitions}
\bl[Definition]{Let $a,b\in\mathbb{Z}$. We call $d\in\mathbb{Z}$ a \textbf{common divisor} of $a$ and $b$ provided $d|a$ and $d|b$.}
For example, if $a=30$ and $b=24$, then the common divisors are
\[
-6,-3,-2,-1,1,2,3,6
\]\vspace{-0.4cm}

\bl[Definition]{Let $a,b\in\mathbb{Z}$. We call $d\in\mathbb{Z}$ the \textbf{greatest common divisor} of $a$ and $b$, provided 
\enumb
\item[(1)] $d$ is a common divisor of $a$ and $b$,
\item[(2)] if $e$ is a common divisor of $a$ and $b$, then $e\leq d$.
\enume
Notation: $gcd(a,b)$}
For example, $gcd(30,24)=6$.
\end{frame}

\begin{frame}{Computing $gcd(a,b)$}
We assume for simplicity that $a$  and $b$ are positive integers.
\underline{Alternative 1:} Brute force
\itemb
\item For every positive integer $k$ from $1$ to $min(a,b)$, check whether $k|a$ and $k|b$. If so, save that number $k$ on a list.
\item Choose the largest number on the list, that is $gcd(a,b)$.
\iteme
This is terribly slow. 
\end{frame}

\begin{frame}{Computing $gcd(a,b)$}
\underline{Alternative 2}: Euclidean algorithm.
\itemb
\item If $b|a$ then $gcd(a,b)=|b|$.
\item If $b\not|~a$, then write
\[
\begin{array}{cc}
a=bq_1+r_1,& 0<r_1<|b|\\
b=r_1q_2+r_2,& 0<r_2<r_1\\
r_1=r_2q_3+r_3&0<r_3<r_2\\
\vdots&\vdots\\
r_{n-2}=r_{n-1}q_n+r_n& 0<r_n<r_{n-1}\\
r_{n-1}=r_nq_{n+1}
\end{array}
\]
\iteme
This finishes in finite steps as the sequence of remainders decreases:
\[
|b|>r_1>r_2>\dots>r_{n-1}>r_n>0
\]
Then $gcd(a,b)=r_n$.
\end{frame}

\begin{frame}{Computing $gcd(a,b)$}
\itemb
\item For example, find $gcd(689,234)$.
\[
\begin{array}{c}
\textcolor{blue}{689}=2\cdot\textcolor{blue}{234}+\color{blue}221\color{black}\\
\textcolor{blue}{234}=1\cdot \color{blue}221\color{black}+\color{blue}13\color{black}\\
\color{blue}221\color{black}=17\cdot \color{blue}13\color{black}
\end{array}
\]
and therefore $gcd(689,234)=13$.
\item Another example, find $gcd(431,29)$.
\[
\begin{array}{c}
\textcolor{blue}{431}=14\cdot \textcolor{blue}{29}+\textcolor{blue}{25}\\
\textcolor{blue}{29}=1\cdot \textcolor{blue}{25}+\textcolor{blue}{4}\\
\textcolor{blue}{25}=6\cdot \textcolor{blue}{4}+\textcolor{blue}{1}\\
\textcolor{blue}{4}=4\cdot \textcolor{blue}{1}
\end{array}
\]
and therefore $gcd(431,29)=1$.
\iteme
\end{frame}

\begin{frame}{Computing $gcd(a,b)$}
\[
\begin{array}{cc}
a=bq_1+r_1,& 0<r_1<|b|\\
b=r_1q_2+r_2,& 0<r_2<r_1\\
r_1=r_2q_3+r_3&0<r_3<r_2\\
\vdots&\vdots\\
r_{n-2}=r_{n-1}q_n+r_n& 0<r_n<r_{n-1}\\
r_{n-1}=r_nq_{n+1}
\end{array}
\]
\itemb
\item $r_n$ is a common divisor:
\[
r_n|r_{n-1}\rightarrow r_n|r_{n-2}\rightarrow\dots\rightarrow r_n|b\rightarrow r_n|a
\]
\item $r_n$ is a $gcd$: Let $c|a$, $c|b$. Then
\[
c|(a-bq_1)=r_1\to c|(b-r_1q_2)=r_2\to\dots\to c|(r_{n-2}-r_{n-1}q_n)=r_n
\]
\iteme
Both argument can be made precise by induction.
\end{frame}

\begin{frame}
\[
\begin{array}{cc}
a=bq_1+r_1,& 0<r_1<|b|\\
b=r_1q_2+r_2,& 0<r_2<r_1\\
r_1=r_2q_3+r_3&0<r_3<r_2\\
\vdots&\vdots\\
r_{n-2}=r_{n-1}q_n+r_n& 0<r_n<r_{n-1}\\
r_{n-1}=r_nq_{n+1}
\end{array}
\]
By starting the algorithm from the second line, we also proved

\bl[Proposition]{Let $a$ and $b$ be positive integers and let $c=a\mod b$. Then
\[
gcd(a,b)=gcd(a,a\mod b)
\]}
The Euclidean algorithm can be written as
\[
gcd(a,b)=gcd(b,r_1)=gcd(r_1,r_2)=\dots=gcd(r_n,0)=r_n
\]
\end{frame}

\begin{frame}{Recursive version of the Euclidean algorithm}
\bl[]{
Input: Positive integers $a$ and $b$.\\
Output: $gcd(a,b)$
\enumb
\item[(1)] Let $c= a\mod b$.
\item[(2)] If $c=0$, then we return $b$ and stop.
\item[(3)] Otherwise, return $gcd(b,c)$.
\enume}

\textbf{Q:} What about when $a$ or $b$ is not a positive integer?
\itemb
\item Note that the list of divisors for $a$ and $-a$ are the same.
\item Same for $b$ and $-b$. 
\iteme
\bl[]{$gcd(a,b)=gcd(|a|,|b|)$}
There is only one exception when this does not help: $a=b=0$.
\end{frame}

\begin{frame}
\bl[Theorem]{Let $a$ and $b$ be positive integers. There are $u,v\in\mathbb{Z}$ such that 
\[
gcd(a,b)=ua+vb
\]}
First:
\[
\begin{array}{cccc}
a=bq_1+r_1&\to&r_1=a-bq_1\\
\end{array}
\]
Second:
\[
\begin{array}{c}
b=r_1q_2+r_2\\
\downarrow\\
r_2=b-r_1q_2=b-q_2(a-bq_1)=a(-q_2)+b(1+q_1q_2)
\end{array}
\]
and one can proceed similarly until reaching $r_n=ua+vb$.
\end{frame}

\begin{frame}
For example, find $x$ and $y$ integers such that
\[
\textcolor{blue}{431}x+\textcolor{blue}{29}y=gcd(\textcolor{blue}{431},\textcolor{blue}{29})(=1)
\]
Write
\[
\begin{array}{c}
\textcolor{blue}{431}=14\cdot \textcolor{blue}{29}+\textcolor{blue}{25}\\
\textcolor{blue}{29}=1\cdot \textcolor{blue}{25}+\textcolor{blue}{4}\\
\textcolor{blue}{25}=6\cdot \textcolor{blue}{4}+\textcolor{blue}{1}\\
\textcolor{blue}{4}=4\cdot \textcolor{blue} {1}
\end{array}
\]
Therefore 
\[
25=\textcolor{blue}{431}-14\cdot \textcolor{blue}{29}
\]
\[
4=\textcolor{blue}{29}-1\cdot 25=\textcolor{blue}{29}-\textcolor{blue}{431}+14\cdot \textcolor{blue}{29}=15\cdot\textcolor{blue}{29}-\textcolor{blue}{431}
\]
\[
1=25-6\cdot 4=(\textcolor{blue}{431}-14\cdot \textcolor{blue}{29})-6(15\cdot \textcolor{blue}{29}-\textcolor{blue}{431})=7\cdot \textcolor{blue}{431}-(6\cdot 15+14)\cdot \textcolor{blue}{29}
\]
and so $x=7$ and $y=-104$.
\end{frame}

\begin{frame}{Relative primes}
\bl[Proposition]{For $a,b\in\mathbb{Z}$ positive, $gcd(a,b)$ is the smallest integer of the form $ax+by$.}
\begin{proof}
Note $gcd(a,b)|(ax+by)$ and therefore $gcd(a,b)\leq ax+by$.
\end{proof}
\bl[Definition]{Let $a$ and $b$ be integers. We call $a$ and $b$ \textbf{relatively prime} provided $gcd(a,b)=1$.}

\bl[Corollary]{Let $a$ and $b$ be integers. There exist integers $x$ and $y$ such that $ax+by=1$ if and only if $a$ and $b$ are relatively prime.}
\end{frame}

\begin{frame}[t]{Diophantine equations}
Algebraic equations involving only integers are usually called Diophantine equations.
\bl[Theorem]{
Let $a,b,c$ be integers. The equation $ax+by=c$ has integer solution if and only if $gcd(a,b)|c$.
}
\bl[Proof]{
\enumb
\only<1>{\item[$\Rightarrow$] Assume that the pair of integers $x_0,y_0$ is a solution. Then
\[
gcd(a,b)|ax_0+by_0=c.
\]}
\only<2>{\item[$\Leftarrow$]  Assume $gcd(a,b)|c$, i.e. there is a $t\in\mathbb{Z}$, such that $gcd(a,b)t=c$. Take $u,v\in\mathbb{Z}$ such that\vspace{-0.2cm}
\[
gcd(a,b)=au+bv
\]\vspace{-0.6cm}\\
and multiply by $t$ to get\vspace{-0.2cm}
\[
c=t\cdot gcd(a,b)=a(ut)+b(vt).
\]\vspace{-0.7cm}\\
Thus $x=ut$ and $y=vt$ is a solution.\qed}
\enume}

\end{frame}


\end{document}