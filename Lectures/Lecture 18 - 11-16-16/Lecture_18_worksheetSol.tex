\documentclass[11pt]{preprint}

\setlength{\topmargin}{0mm} \setlength{\oddsidemargin}{0mm}
\setlength{\textwidth}{160mm} \setlength{\textheight}{215mm}

\usepackage{amssymb,amsmath,amscd,amsthm}
\usepackage{graphics}
\usepackage{tikz}

\def\enumb{\begin{enumerate}}
\def\enume{\end{enumerate}}
\def\itemb{\begin{itemize}}
\def\iteme{\end{itemize}}
\def\integers{\mathbb{Z}}

\def\multiset#1#2{\ensuremath{\left(\kern-.3em\left(\genfrac{}{}{0pt}{}{#1}{#2}\right)\kern-.3em\right)}}



\newtheorem{proposition}{Proposition}
\newtheorem{theorem}{Theorem}

\title{Discrete Mathematics, 2016 Fall - Worksheet 18}
\author{Instructor: Zsolt Pajor-Gyulai, CIMS}
\date{November 16, 2016}



\begin{document}

\maketitle

In all of the above problems explain your answer in full English sentences.

\enumb
\item For the given integers $a,b$, find the integers $q$ and $r$ such that $a=qb+r$ and $0\leq r<b$.
\enumb
\item $a=100$, $b=3$
\[
100=33\cdot 3 +1
\]
\item $a=-100$, $b=3$
\[
-100=(-34)\cdot 3 + 2
\]
\enume
\item For the given integers $a,b$, compute $a\textrm{ div } b$ and $a\textrm{ mod } b$.
\enumb
\item $a=99$, $b=3$.
\[
a{\rm~div~} b = 33\qquad a{\rm~mod~} b = 0
\]
\item $a=-99$,$b=3$.
\[
a{\rm~div~} b = -33\qquad a{\rm~mod~}b = 0
\]
\item $a=10$, $b=3$.
\[
a{\rm~div~}b = 3\qquad a{\rm~mod~} b = 1
\]
\enume

\item Please calculate:
\enumb
\item $gcd(20,25)$
We write
\begin{align*}
20& = 0\cdot 25+20\\
25& = 1\cdot 20+ 5\\
20& = 4\cdot 5 + 0
\end{align*}
and thus $gcd(20,25)=5$.
\item $gcd(-89,-98)$.
We write
\begin{align*}
-89&=1\cdot(-98)+9\\
-98&= -11\cdot 9 + 1\\
9&=9\cdot 1 + 0
\end{align*}
and thus $gcd(-89,-98)=1$
\enume
\item For each pair of integers $a,b$ in the previous problem, find integers $x$ and $y$ such that $ax+by=gcd(a,b)$.
Working our way backwards from the bottom in the first case, we can write
\[
gcd(20,25)=5 = 25 - 1\cdot 20 =(-1)\cdot 20 + 1\cdot 25
\]
and hence $x=-1$ and $y=1$. In the second one,
\[
gcd(-89,-98)=1=11\cdot 9 + (-1)\cdot 98 = 11\cdot(98-89)-98 = 11\cdot(-89)+(-10)\cdot (-98)
\]
and hence $x=11$ and $y=-10$.
\item Let $a$ and $b$ be positive integers. Prove that $2^a$ and $2^b-1$ are relatively prime.

Note that by the theorem in class, we can conclude that $2^a$ and $2^b$ are relative primes if we can find integers $x,y\in\mathbb{Z}$ such that
\[
x2^a+y(2^b-1)=1
\]
If $a\leq b$ then $y=-1$ and $x=2^{b-a}$ works. On the other hand if $a>b$, then the situation is more complicated. Let $N$ be a number such that $2^Nb>a$. Then we write
\[
(2^b-1)(2^b+1)=2^{2b}-1
\]
\[
(2^{2b}-1)(2^{2b}+1)=2^{4b}-1
\]
\[
\vdots
\]
\[
(2^{2^{N-1}b}-1)(2^{2^{N-1}b}+1)=2^{2^Nb}-1
\]
Then take $x=2^{2^{N}b-a}$ and 
\[
y=-(2^b+1)(2^{2b}+1)\dots(2^{2^{N-1}b}+1)
\]
\item Decide if the following diophantine equations have a solution or not and if yes find a solution:
\itemb
\item $3x+4y=2$

Since $gcd(3,4)=1$, the solution exists and one possible solution is given by $x=-2$, $y=2$. (taking $t=2$, $u=-1$, $v=1$ on the slides)
\item $6x-2y=4$
Since $gcd(6,2)=2|4$, the solution exists. We can write $gcd(6,2)=6-2\cdot 2$ so $u=1$ and $v=-2$. Since $t=2$, we have $x=2$, $y=-4$.
\iteme
\enume
\end{document}