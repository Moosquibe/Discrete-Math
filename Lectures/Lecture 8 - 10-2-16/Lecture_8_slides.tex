\documentclass{beamer}

\mode<presentation>
{
  \usetheme{Frankfurt}
  \usecolortheme{orchid}
  \setbeamercovered{invisible}
  \setbeamertemplate{footline}[frame number]
}

\usepackage[english]{babel}
\usepackage[latin1]{inputenc}
\usepackage{times}
\usepackage[T1]{fontenc}
\usepackage{tikz}
\usepackage{array}
\usepackage{cancel}


\usetikzlibrary{shapes,backgrounds}

\def\blue{\color{blue}~}
\def\black{\color{black}~}
\def\bl[#1]#2{\begin{block}{#1}#2\end{block}}
\def\integers{\mathbb{Z}}
\def\enumb{\begin{enumerate}}
\def\enume{\end{enumerate}}
\def\itemb{\begin{itemize}}
\def\iteme{\end{itemize}}
\def\multiset#1#2{\ensuremath{\left(\kern-.3em\left(\genfrac{}{}{0pt}{}{#1}{#2}\right)\kern-.3em\right)}}



\usepackage{remreset}
\makeatletter
\@removefromreset{subsection}{section}
\makeatother
\setcounter{subsection}{1}

\title{Discrete Mathematics, Section 002, Fall 2016}
\subtitle{Lecture 8: Binomial and Multinomial coefficients}

\author[Zsolt]{Zsolt Pajor-Gyulai \\ \texttt{zsolt@cims.nyu.edu}}
\date{October 2, 2016}
\pgfdeclareimage[height=1cm]{NYUlogo}{NYUlogo.jpg}

\institute[NYU] 
{
\normalsize Courant Institute of Mathematical Sciences
}
\titlegraphic{\pgfuseimage{NYUlogo}}

\begin{document}

\begin{frame}
  \titlepage
\end{frame}

\AtBeginSection[]
{
\begin{frame}
\frametitle{Outline}
\tableofcontents[currentsection]
\end{frame}}

\section{Binomial coefficients}

\begin{frame}
Let $A=2^{\{1,2\}}$ and $R$ be the has-the-same-size relation.
\[
\begin{tabular}{| >{\centering\arraybackslash}b{1in}|| >{\centering\arraybackslash}b{1in}|}
\hline
\textrm{Equivalence class}& \textrm{Size of class}\\
\hline
$[\emptyset]$ & 1\\
$[\{1\}]$& 2\\
$[\{1,2\}]$& 1\\
\hline
\end{tabular}
\]
Also 
\bl[]{
\[
(x+y)^2=1\cdot x^2+2\cdot xy+1\cdot y^2
\]}
\end{frame}

\begin{frame}
Let $A=2^{\{1,2,3\}}$ and $R$ be the has-the-same-size relation.
\[
\begin{tabular}{| >{\centering\arraybackslash}b{1in}|| >{\centering\arraybackslash}b{1in}|}
\hline
\textrm{Equivalence class}& \textrm{Size of class}\\
\hline
$[\emptyset]$ & 1\\
$[\{1\}]$& 3\\
$[\{1,2\}]$& 3\\
$[\{1,2,3\}$& 1\\
\hline
\end{tabular}
\]
Also 
\bl[]{
\[
(x+y)^3=1\cdot x^2+3\cdot x^2y+3\cdot xy^2+1\cdot y^3
\]}

\end{frame}

\begin{frame}
Let $A=2^{\{1,2,3,4\}}$ and $R$ be the has-the-same-size relation.
\[
\begin{tabular}{| >{\centering\arraybackslash}b{1in}|| >{\centering\arraybackslash}b{1in}|}
\hline
\textrm{Equivalence class}& \textrm{Size of class}\\
\hline
$[\emptyset]$ & 1\\
$[\{1\}]$& 4\\
$[\{1,2\}]$& 6\\
$[\{1,2,3\}$& 4\\
$[\{1,2,3,4\}]$& 1\\
\hline
\end{tabular}
\]

Also 
\bl[]{
\[
(x+y)^4=1\cdot x^4+4\cdot x^3y+6\cdot x^2y^2+4\cdot xy^3+1\cdot y^4
\]}
\end{frame}

\begin{frame}{The connection}
Note that
\bl[]{
\[
(x+y)^2=(x+y)(x+y)=xx+xy+yx+yy
\]}
also
\bl[]{\vspace{-0.3cm}
\[
(x+y)^3=xxx+xxy+xyx+xyy+yxx+yxy+yyx+yyy
\]}
In general
\[
(x+y)^n=\underbrace{(x+y)}_{1}\underbrace{(x+y)}_{2}\dots\underbrace{(x+y)}_{n}
\]
\itemb
\item To form a term, we make a lists of length $n$ of $x$-s and $y$-s. 
\item Then we need to identify the ones that only differ in the order of the $x$,$y$-s. 
\iteme
This defintes an equivalence relation $R$ on the set of such lists.
\end{frame}

\begin{frame}{The connection}
In general
\[
(x+y)^n=\underbrace{(x+y)}_{1}\underbrace{(x+y)}_{2}\dots\underbrace{(x+y)}_{n}
\]
\itemb
\item To form a term, we make a lists of length $n$ of $x$-s and $y$-s. 
\item Then we need to identify the ones that only differ in the order of the $x$,$y$-s. 
\iteme
This defintes an equivalence relation $R$ on the set of such lists.
\[
|[\underbrace{x\dots x}_{n-k}\underbrace{y\dots y}_{k}]|=?
\]
\end{frame}

\begin{frame}{The connection}
For example
\[
|[xxxyyy]|=?
\]
We only have to keep track of where the $y$-s are.
\[
\begin{tabular}{| >{\centering\arraybackslash}b{1in}|| >{\centering\arraybackslash}b{1in}|}
\hline
\textrm{List in class}& \textrm{Pos. of $y$-s}\\
\hline
$xxxyyy$ & $\{4,5,6\}$\\
$\hdots$&$\hdots$\\
$xyxyxy$& $\{2,4,6\}$\\
$yxyxyx$& $\{1,3,5\}$\\
$\hdots$& $\hdots$\\
$yyyxxx$& $\{1,2,3\}$\\
\hline
\end{tabular}
\]
On the right hand side we list all three element subsets of $\{1,2,3,4,5,6\}$.
\end{frame}

\begin{frame}
Similarly,
\[
|[\underbrace{x\dots x}_{n-k}\underbrace{y\dots y}_{k}]|
\]
is given by the number of all $k$ element subsets of $\{1,2,\dots,n\}$.
\bl[Definition]{
Let $n,k\in\mathbb{N}$. The symbol $\binom{n}{k}$ denotes the number of $k$-element subsets of an $n$-element set. It is called the \textbf{binomial coefficient} and we say it as $n$ choose $k$.
}
\bl[Binomial theorem]{
\[
(x+y)^n=\sum_{k=0}^{n}\binom{n}{k}x^{n-k}y^k
\]
}

Do Problems 1-3 on the Worksheet!
\end{frame}

\begin{frame}{Examples}
\itemb
\item Evaluate $\binom{n}{0}$.\\

The only zero element subset is $\emptyset$,
\[
\binom{n}{0}=1
\]
\item Evaluate $\binom{n}{1}$.\\

There are exactly $n$ ways to choose $1$ element out of $n$ to form a $1$ element subset.
\[
\binom{n}{1}=n
\]
\iteme
\end{frame}


\begin{frame}{Examples}
\itemb
\item Relate $\binom{n}{n-k}$ and $\binom{n}{k}$.\\

Choosing a $k$ element subset specifies exactly an $n-k$ element subset which is the complement. This works both ways and therefore
\bl[Proposition]{
Let $n,k\in\mathbb{N}$ with $0\leq k\leq n$. Then
\[
\binom{n}{k}=\binom{n}{n-k}
\]
}
\item This allows us to compute e.g.

\iteme
\begin{align*}
\binom{5}{0}&=1,\quad\binom{5}{1}=5\quad\binom{5}{2}=\binom{5}{3}=?,\\
\binom{5}{4}&=\binom{5}{1}=5,\quad\binom{5}{5}=\binom{5}{0}=1.
\end{align*}
\end{frame}

\begin{frame}{A combinatorial proof}
\bl[Pascal's identity]{
Let $n$ and $k$ be integers with $0<k<n$. Then
\vspace{-0.2cm}
\[
\binom{n}{k}=\binom{n-1}{k-1}+\binom{n-1}{k}
\]}
\begin{proof}
Question: How many $k$ element subsets does the set $\{1,2,\dots,n\}$ have?
\itemb 
\item $\binom{n}{k}$, by definition.
\item When forming a $k$-element subset of $\{1,2,\dots, n\}$ we either put the element $n$ into it or not.
\itemb
\item If it does then there are $\binom{n-1}{k-1}$ choices complete the subset.
\item If it does not then there are $\binom{n-1}{k}$ choices for the subset.
\iteme
\iteme
Therefore $\binom{n-1}{k-1}+\binom{n-1}{k}$ also gives an answer.
\end{proof}
\end{frame}

\begin{frame}{Pascal's triangle}
\bl[Pascal's identity]{
Let $n$ and $k$ be integers with $0<k<n$. Then
\vspace{-0.2cm}
\[
\binom{n}{k}=\binom{n-1}{k-1}+\binom{n-1}{k}
\]}
\begin{figure}
\centering
\begin{tabular}{rccccccccc}
$n=0$:&    &    &    &    &  1\\\noalign{\smallskip\smallskip}
$n=1$:&    &    &    &  1 &    &  1\\\noalign{\smallskip\smallskip}
$n=2$:&    &    &  1 &    &  2 &    &  1\\\noalign{\smallskip\smallskip}
$n=3$:&    &  1 &    &  3 &    &  3 &    &  1\\\noalign{\smallskip\smallskip}
$n=4$:&  1 &    &  4 &    &  6 &    &  4 &    &  1\\\noalign{\smallskip\smallskip}
\end{tabular}
\end{figure}
\center{The $k$th element (starting from 0) in the $n$-th row gives $\binom{n}{k}$.}
\vspace{0.4cm}
Do Problems 4-5 on the Worksheet!
\end{frame}

\begin{frame}{Formula for $\binom{n}{k}$}
\bl[Theorem]{
Let $n$ and $k$ be integers with $0\leq k\leq n$. Then\vspace{-0.2cm}
\[
\binom{n}{k}=\frac{n!}{k!(n-k)!}
\]}
\bl[Proof]{
Consider $A$ to be all the rearrangements of the numbers $\{1,2,\dots,n\}$ and for each rearrangement consider the subset given by the first $k$ elements. We define an equivalence relation under which two rearrangements are equivalent if they give the same subset. Indeed, define the equivalence relation $R$ such that two rearrangements are equivalent provided the first $k$ elements (and therefore the last $n-k$ automatically) are the same but possibly rearranged amongst each other.
[...]}
\end{frame}

\begin{frame}{Formula for $\binom{n}{k}$}\vspace{-0.2cm}
\bl[Theorem]{
Let $n$ and $k$ be integers with $0\leq k\leq n$. Then\vspace{-0.2cm}
\[
\binom{n}{k}=\frac{n!}{k!(n-k)!}
\]}
\begin{proof}

[...]
The cardinality of each equivalence class is the number of arrangements of the first $k$ elements times the number of arrangements of the last $n-k$ elements which is clearly $k!\cdot (n-k)!$. There is a one to one correspondence between subsets of $k$ elements and the equivalence classes and therefore
\[
\binom{n}{k}=\frac{|A|}{|[.]|}=\frac{n!}{k!(n-k)!},
\]
and the theorem is proved.
\end{proof}
\end{frame}
\begin{frame}
Now use this to do Problem 6-8 on the Worksheet!
\end{frame}
\section{Multinomial coefficients}
\begin{frame}{Multinomial coefficients}
\bl[Alternative way to think of $\binom{n}{k}$]{
Let $A$ be an $n$-element set and we want to hand out $k$ labels saying `good' and $n-k$ labels saying `bad'. How many ways can we do this? $\qquad\to\qquad\binom{n}{k}$.
}
This can be generalized:
\bl[Multinomial coefficient]{
Let $\binom{n}{a~b~c}$ be the number of ways to label the elements of an $n$-element set with three types of labels, where we hand out $a$ label of Type I, $b$ label of Type II and $c$ label of Type III.
}
Compute this for a few simple cases by doing Problem 9 from the worksheet!
\end{frame}

\begin{frame}{Formula for the multinomial coefficient}
\bl[Propositon]{
\[
\binom{n}{a~b~c}=\left\{\begin{array}{cc}
\frac{n!}{a!b!c!}&\textrm{if }a+b+c=n\\
0&\textrm{otherwise}
\end{array}\right.
\]
}
\begin{proof}
\only<1>{~~~~If $a+b+c\neq n$, then there is no way to hand out exactly one label to all elements. Therefore in this case $\binom{n}{a~b~c}=0$.

[...]
}

\only<2>{[...]
~~Assume $a+b+c=n$. On the set  $D$ of all rearrangements of the set of $n$ elements $N$, define the equivalence relation $R$ where two arrangements are equivalent if their first $a$ elements form the same subset of $N$ and similarly the following $b$ elements and the remaining $c$ elements after.\vspace{-0.4cm}
\[
\underbrace{********}_a|\underbrace{*****}_b|\underbrace{********}_c
\]
[...]\qedhere}

\only<3->{[...]
\[
\underbrace{********}_a|\underbrace{*****}_b|\underbrace{********}_c
\]\vspace{-0.4cm}

In any equivalence class, there are those rearrangements where these subsets are rearranged individually. A labelling can be identified with an equivalence class. This can be done $a!b!c!$ ways. Therefore\vspace{-0.2cm}
\[
\binom{n}{a~b~c}=\frac{|D|}{|[d]|}=\frac{n!}{a!b!c!},\qquad d\in D.\qedhere
\]}
\end{proof}
\end{frame}

\begin{frame}{Multinomial theorem}
\bl[Theorem]{
\[
(x+y+z)^n=\sum_{a+b+c=n}\binom{n}{a~b~c}x^ay^bz^c
\]
where the sum is over all natural numbers $a$, $b$, $c$, with $a+b+c=n$.
}
\end{frame}
\end{document}