\documentclass[11pt]{preprint}

\setlength{\topmargin}{0mm} \setlength{\oddsidemargin}{0mm}
\setlength{\textwidth}{160mm} \setlength{\textheight}{215mm}

\usepackage{amssymb,amsmath,amscd,amsthm}

\def\enumb{\begin{enumerate}}
\def\enume{\end{enumerate}}
\def\itemb{\begin{itemize}}
\def\iteme{\end{itemize}}
\def\integers{\mathbb{Z}}



\newtheorem{proposition}{Proposition}
\newtheorem{theorem}{Theorem}

\title{Discrete Mathematics, 2016 Fall - Worksheet 6}
\author{Instructor: Zsolt Pajor-Gyulai, CIMS}

\date{September 26, 2016}

\begin{document}

\maketitle

In all of the above problems explain your answer in full English sentences.

\enumb
\item Similar to what we did on the slide for $<$, define the corresponding relation set to two of the integer relations $\leq,=,>,\geq$.

\vspace{0.2cm}
\[
R_{\leq}=\{(x,y)\in \mathbb{Z}\times\mathbb{Z}: x\leq y\}
\]
\[
R_{=}=\{(x,x)\in \mathbb{Z}\times\mathbb{Z}: x\in\mathbb{Z}\}
\]
\[
R_{>}=\{(x,y)\in \mathbb{Z}\times\mathbb{Z}: x>y\}
\]
\[
R_{\geq}=\{(x,y)\in \mathbb{Z}\times\mathbb{Z}: x\geq y\}
\]
\vspace{0.1cm}

\item Write the following relations on the set $\{1,2,3,4,5\}$ as sets of ordered pairs.
\enumb
\item The $\leq$ relation.
\[
R=\{(1,1),(1,2),(1,3),(1,4),(1,5),(2,2),(2,3),(2,4),(2,5),(3,3),(3,4),(3,5),(4,4),(4,5),(5,5)\}
\]
\item The `divides' relation.
\[
R_{|}=\{(1,1),(1,2),(1,3),(1,4),(1,5),(2,2),(2,4),(3,3),(4,4),(5,5)\}
\]
\item The $=$ relation.
\[
R_{=}=\{(1,1),(2,2),(3,3),(4,4),(5,5)\}
\]
\enume

\item Each of the following is a relation on the set $\{1,2,3,4,5\}$. Express these relations in words and then find their inverses.
\enumb
\item $R=\{(1,2),(2,3),(3,4),(4,5)\}$. Being consecutive integers.
\[
R^{-1}=\{(2,1),(3,2),(4,3),(5,4)\}
\]
\item $R=\{(1,1),(2,1),(2,2),(3,1),(3,2),(3,3),(4,1),(4,2),(4,3),(4,4),(5,1),(5,2),(5,3),(5,4),(5,5)\}$  Greater than or equal.
\[
R^{-1}=\{(1,1),(1,2),(2,2),(1,3),(2,3),(3,3),(1,4),(2,4),(3,4),(4,4),(1,5),(2,5),(3,5),(4,5),(5,5)\}
\]
\item $R=\{(1,5),(2,4),(3,3),(4,2),(5,1)\}$. Sum to $6$.
\[
R^{-1}=\{(5,1),(4,2),(3,3),(2,4),(1,5)\}
\]
\enume

\item What is the inverse of the following relations?
\enumb
\item $\leq$.
\[
\geq
\]
\item $\{(x,y):x,y\in\integers,x-y=1\}$.
\[
\{(x,y):x,y\in\integers, y-x =1\}
\]
\item $\{(x,y):x,y\in\integers, xy>0\}$.
\[
\{(x,y): x,y\in\integers, xy>0\}.
\]
\enume

\item For each of the following relations on $\{1,2,3,4,5\}$ determine whether the relation is reflexive, irreflexive, symmetric, antisymmetric, and/or transitive:
\itemb
\item $R=\{(1,1),(2,2),(3,3),(4,4),(5,5)\}$. Reflexive, Symmetric, Transitive, Antisymmetric
\item $R=\{(1,2),(2,3),(3,4),(4,5)\}$. Irreflexive, Antisymmetric (Vacuously)
\item $R=\{(1,1),(1,2),(1,3),(1,4),(1,5)\}$. Antisymmetric, Transitive
\item $R=\{(1,1),(1,2),(2,1),(3,4),(4,3)\}$. Symmetric
\item $R=\{1,2,3,4,5\}\times\{1,2,3,4,5\}$. Reflexive, Symmetric, Transitive
\iteme

\item For the following relations on the set of humans beings, please determine whether the relation if reflexive, irreflexive, symmetric, antisymetric, and/or transitive.
\enumb
\item has the last name as REFLEXIVE, SYMMETRIC, ANTISYMMETRIC, TRANSITIVE
\item is the child of IRREFLEXIVE, ANTISYMMETRIC (Vacuously)
\item is married to SYMMETRIC, IRREFLEXIVE
\item has a common parent as REFLEXIVE, SYMMETRIC
\enume

\item Consider the relation $|$ (divisible) on
\enumb
\item On the naturals. REFLEXIVE, ANTISYMMETRIC, TRANSITIVE
\item On the integers. REFLEXIVE, TRANSITIVE
\enume
Decide what properties do they have.

\item Show that the following relation is an equivalence relation:
\[
A=\{B\in 2^{\integers}: |B|<\infty\}, \qquad R=\{(B,C): B,C \in A, |B|=|C|\}
\]
\begin{proof}
We have to show that $R$ is reflexive, symmetric and transitive.

Obviously $|B|=|B|$ for every $B\in A$ and therefore $(B,B)\in R$ and $R$ is reflexive. Also if $(B,C)\in R$, i.e. $|B|=|C|$, then $|C|=|B|$, i.e. $(C,B)\in R$ and therefore $R$ is symmetric. Finally, if $(B,C)\in R$, i.e. $|B|=|C|$ and $(C,D)\in R$ i.e., $|C|=|D|$ for some $B,C,D\in A$, then $|B|=|D|$ and thus $(B,D)\in R$ and therefore $R$ is transitive.
\end{proof}
\item Which of the following are equivalence relations?
\enumb
\item $R=\{(1,1),(1,2),(2,1),(2,2),(3,3)\}$ on the set $\{1,2,3\}$. THIS ONE.
\item $|$ on $\integers$. NOT THIS ONE
\item $\leq$ on $\integers$. NOT THIS ONE
\item Is-an-anagram-of on the set of English words. THIS ONE
\enume
\item For each of the following congruences, find all integers $N$, with $N>1$, that make the congruence true
\enumb
\item $23\equiv 13$ (mod $N$) $N=1,2,5,10$
\item $10\equiv 5$ (mod $N$) $N=1,5$
\item $6\equiv 60$ (mod $N$) $N=1,2,3,6,9,18,27,54$
\enume

%\item There is only one possible equivalence relation on a one-element set: If $A=\{1\}$, then $R=\{(1,1)\}$ is the only possible equivalence relation. On a two element set $A=\{1,2\}$, there are exactly two equivalence relations:
%\[
%R_1=\{(1,1),(2,2)\},\qquad R_2=\{(1,1),(1,2),(2,1),(2,2)\}.
%\]
%How many different equivalence relations are there on a three and on a four element set?

\enume

\end{document}