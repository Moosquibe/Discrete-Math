\documentclass{beamer}

\mode<presentation>
{
  \usetheme{Frankfurt}
  \usecolortheme{orchid}
  \setbeamercovered{invisible}
  \setbeamertemplate{footline}[frame number]
}

\usepackage[english]{babel}
\usepackage[latin1]{inputenc}
\usepackage{times}
\usepackage[T1]{fontenc}
\usepackage{tikz}
\usepackage{array}
\usepackage{cancel}


\usetikzlibrary{shapes,backgrounds}

\def\blue{\color{blue}~}
\def\black{\color{black}~}
\def\bl[#1]#2{\begin{block}{#1}#2\end{block}}
\def\integers{\mathbb{Z}}
\def\enumb{\begin{enumerate}}
\def\enume{\end{enumerate}}
\def\itemb{\begin{itemize}}
\def\iteme{\end{itemize}}


\usepackage{remreset}
\makeatletter
\@removefromreset{subsection}{section}
\makeatother
\setcounter{subsection}{1}

\title{Discrete Mathematics, Section 002, Fall 2016}
\subtitle{Lecture 6: Relations, Equivalence relations}
\date{September 26, 2016}

\author[Zsolt]{Zsolt Pajor-Gyulai \\ \texttt{zsolt@cims.nyu.edu}}

\pgfdeclareimage[height=1cm]{NYUlogo}{NYUlogo.jpg}

\institute[NYU] 
{
\normalsize Courant Institute of Mathematical Sciences
}
\titlegraphic{\pgfuseimage{NYUlogo}}

\begin{document}

\begin{frame}
  \titlepage
\end{frame}

\AtBeginSection[]
{
\begin{frame}
\frametitle{Outline}
\tableofcontents[currentsection]
\end{frame}}

\section{Relations}

\begin{frame}{A different take on ordinary relations}
\bl[]{Ordinary relations you are probably familiar with: $<,\leq,=,>,\geq$.}

Let us look at them a little differently. For example, let
\[
R_<=\{(a,b): a\in\integers, b\in\integers, a<b\}.
\]
Then 
\[
(a,b)\in R_<\qquad\Leftrightarrow\qquad a<b.
\]
You can also think about this as:
\[
\textrm{'The pair $(a,b)$ passes the test of $<$  if and only if $(a,b)\in R_<$}'.
\]
Note that $R_<\subseteq\integers\times\integers$.

\begin{center}
Do Problem 1 on the Worksheet!
\end{center}
\end{frame}

\begin{frame}{Abstract notion of a relation}
\itemb
\item We want to have a notion of a general relation on a set $A$. 
\item We want to have a set $R$ of all those ordered pairs 
\[
(a_1,a_2)\in A\times A
\]
that 'pass the test' of the relation.
\item In this case, we say $a_1$ is in relation $R$ with $a_2$ or shorthand
\[
a_1 R a_2\qquad (\Leftrightarrow (a_1,a_2)\in R)
\]
\item Similarly, if $(a_1,a_2)\notin R$, we say $a_1$ is not in relation $R$ with $a_2$ or shorthand
\[
a_1\cancel{R}a_2.
\]
\item This way, we can call the set $R$ itself a relation.
\bl[]{\vspace{-0.4cm}
\[
\begin{array}{clr}
(a_1,a_2)\in R &\qquad \textrm{$a_1$  is in relation $R$ with $a_2$}& a_1 R a_2\\
(a_1,a_2)\notin R &\qquad\textrm{$a_1$ is not in relation $R$ with $a_2$}& a_1\cancel{R} a_2
\end{array}
\]}
\iteme
\end{frame}

\begin{frame}{Abstract notion of a relation}
\bl[]{\vspace{-0.4cm}
\[
\begin{array}{clr}
(a_1,a_2)\in R &\qquad \textrm{$a_1$  is in relation $R$ with $a_2$}& a_1 R a_2\\
(a_1,a_2)\notin R &\qquad\textrm{$a_1$ is not in relation $R$ with $a_2$}& a_1\cancel{R} a_2
\end{array}
\]}
For example:
\[
A=\{1,2,3,4\}\qquad R=\{(1,2), (1,3), (2,2), (3,2)\}
\]
\vspace{-0.5cm}
\itemb
\item $1 R 2$
\item $2 \cancel{R} 1$
\item $4$ is not in relation $R$ with anything, not even with itself.
\item Nothing is in relation $R$ with 4, not even itself.
\item $2 R 2$.
\iteme
\end{frame}

\begin{frame}{Abstract notion of a relation}
\bl[Relation]{
A \textbf{relation} is a set of ordered pairs.
}
\bl[Relation on a set]{
We say $R$ is a \textbf{relation on a set $A$}, provided $R\subseteq A\times A$.
}

\bl[Relation between sets]{
We say $R$ is a \textbf{relation from set $A$ to set $B$} provided $R\subseteq A\times B$.}
\end{frame}

\begin{frame}{Further example}
Let $A=\{1,2,3,4\}$ and $B=\{4,5,6,7\}$.
\begin{align*}
R&=\{(1,1), (2,2), (3,3), (4,4)\}\\
S&=\{(1,2),(3,2)\}\\
T&=\{(1,4),(1,5),(4,7)\}\\
U&=\{(4,4),(5,2),(6,2),(7,3)\}\\
V&=\{(1,7),(7,1)\}
\end{align*}
All of these are relations.
\itemb
\item $R$ is the equality relation on $A$.
\item $S$ is a relation on $A$
\item $T$ is a relation from $A$ to $B$.
\item $U$ is a relation from $B$ to $A$.
\item $V$ is a relation but it is not from $A$ to $B$ nor from $B$ to $A$.
\iteme
\end{frame}

\begin{frame}
\begin{center}
Do Problem 2 on the worksheet!
\end{center}
\end{frame}

\begin{frame}{Inverse relation}
\bl[Inverse relation]{
Let $R$ be a relation. The \textbf{inverse of $R$}, denoted $R^{-1}$, is the relation formed by reversing the order of all the ordered pairs in $R$. In symbols, $R^{-1}=\{(x,y): (y,x)\in R\}$}\vspace{-0.5cm}
\[
R=\{(1,5),(2,6),(3,7),(3,8)\}\to R^{-1}=\{(5,1), (6,2), (7,3), (8,3)\}
\]\vspace{-0.5cm}
\bl[Proposition]{Let $R$ be a relation. Then $(R^{-1})^{-1}=R$.}
\begin{proof}
~~~~Suppose $(x,y)\in R$. Then $(y,x)\in R^{-1}$ and thus $(x,y)\in (R^{-1})^{-1}$.

~~~~Now suppose $(x,y)\in (R^{-1})^{-1}$. Then $(y,x)\in R^{-1}$ and so $(x,y)\in R$.\qedhere
\end{proof}

\end{frame}

\begin{frame}
\begin{center}
Do Problems 3-4 on the Worksheet!
\end{center}
\end{frame}

\begin{frame}{Properties of relations}
\begin{definition}
Let $R$ be a relation on a set $A$.
\itemb
\item If for all $x\in A$ we have $x R x$, we call $R$ \textbf{reflexive}.
\item If for all $x\in A$ we have $x\cancel{R} x$, we call $R$ \textbf{irreflexive}.
\item If for all $x,y\in A$ we have $xRy\Rightarrow y R x$, we call $R$ \textbf{symmetric}.
\item If for all $x,y\in A$, we have\vspace{-0.2cm}
\[
(x R y\wedge y R x)\Rightarrow x=y,
\]

\vspace{-0.2cm}we call $R$ \textbf{antisymmetric}.
\item If for all $x,y,z\in A$ we have\vspace{-0.2cm}
\[
(x R y\wedge y R z)\Rightarrow x R z,
\]

\vspace{-0.2cm}we call $R$ \textbf{transitive}.
\iteme
\end{definition}
\end{frame}

\begin{frame}
\bl[]{
\textbf{reflexive}: $\forall x\in A, x R x$.\\
\textbf{symmetric}: $\forall x,y\in A, xRy\Rightarrow y R x$\\
\textbf{transitive}: $\forall x,y,z\in A$, $(xRy\wedge yRz)\Rightarrow xRz$\\
\textbf{antisymmetric}: $\forall x,y\in A, (xRy\wedge yRx)\Rightarrow x=y$.\\
\textbf{irreflexive}: $\forall x\in A$, $x\cancel{R} x$.}
The relation $=$ (equality) on the integers:
\itemb
\item Reflexive: Any integer is equal to itself.
\item Symmetric: If $x=y$ then $y=x$.
\item Transitive: If $x=y$ and $y=z$ then $x=z$.
\item Trivially antisymmetric.
\item Not irreflexive.
\iteme

\begin{center}
Practice this by Problems 6-7 on the worksheet!
\end{center}
\end{frame}

\section{Equivalence relations}

\begin{frame}
Some relations we come across in math, have strong resemblence to equality.
\begin{Definition}
Le $R$ be a relation on a set $A$. We say $R$ is an \textbf{equivalence relation} provided it is reflexive, symmetric and transitive.
\end{Definition}
For example:
\[
A=\{B\in 2^{\integers}: |B|<\infty\}, \qquad R=\{(B,C): B,C \in A, |B|=|C|\}
\]

\begin{center}
Prove this on Problem 8 on the Worksheet!
\end{center}
\end{frame}

\begin{frame}
Another example:
\bl[Congruence modulo $n$]{
Let $n$ be a positive integer. We say that $x$ and $y$ are congruent modulo $n$, and write 
\[
x\equiv y \textrm{ (mod n)}
\]
provided $n|(x-y)$.
}
For example:
\itemb
\item $3\equiv 13$ (mod 5) because $5|-10=3-13$.
\item $4\equiv 4$ (mod 5) because $5|0=4-4$.
\item $16\not\equiv 3$ (mod 5) because $5\nmid 13=16-3$.
\iteme

\begin{center}
Get familiar by doing Problem 10 on the worksheet!
\end{center}
\end{frame}

\begin{frame}
\begin{Theorem}
Let $n$ be a positive integer. The 'is congruent to mod $n$' relation is an equivalence relation on the set of integers.
\end{Theorem}
\begin{proof}
Let $n$ be a positive integer and let $\equiv$ denote congruence mod $n$. We need to show that $\equiv$ is reflexive, symmetric, and transitive.
\itemb
\item Claim: $\equiv$ is reflexive. \only<2>{\color{blue}}\uncover<2->{Let $x$ be an arbitrary integer.} \only<3>{\color{blue}}\uncover<3->{Since $0\cdot n=0$, we have $n|0$, which we can rewrite as $n|(x-x)$.} \only<3>{\color{black}}\uncover<2->{Therefore $x\equiv x$. Thus $\equiv$ is reflexive.}\only<2>{\color{black}}
\item Claim: $\equiv$ is symmetric.\only<4>{\color{blue}}\uncover<4->{ Let $x$ and $y$ be integers and suppose $x\equiv y$.}\only<5>{\color{blue}}\uncover<5->{ This means that $n|(x-y)$.}\only<6>{\color{blue}}\uncover<6->{ So there is an integer $k$ such that $(x-y)=kn$. But then $(y-x)=(-k)n$}\uncover<5->{ and so $n|(y-x)$.}\uncover<4->{ Therefore $y\equiv x$. Thus $\equiv$ is symmetric.}
\item \color{black}Claim: $\equiv$ is transitive.$\cdots$ \uncover<6->{(HW)}$\cdots$ Thus $\equiv$ is transitive.
\iteme
Therefore $\equiv$ is an equivalence relation.
\end{proof}
\end{frame}


\end{document}