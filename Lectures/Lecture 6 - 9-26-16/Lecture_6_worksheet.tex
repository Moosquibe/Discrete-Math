\documentclass[11pt]{preprint}

\setlength{\topmargin}{0mm} \setlength{\oddsidemargin}{0mm}
\setlength{\textwidth}{160mm} \setlength{\textheight}{215mm}

\usepackage{amssymb,amsmath,amscd,amsthm}

\def\enumb{\begin{enumerate}}
\def\enume{\end{enumerate}}
\def\itemb{\begin{itemize}}
\def\iteme{\end{itemize}}
\def\integers{\mathbb{Z}}



\newtheorem{proposition}{Proposition}
\newtheorem{theorem}{Theorem}

\title{Discrete Mathematics, 2016 Fall - Worksheet 6}
\author{Instructor: Zsolt Pajor-Gyulai, CIMS}

\date{September 26, 2016}

\begin{document}

\maketitle

In all of the above problems explain your answer in full English sentences.

\enumb
\item Similar to what we did on the slide for $<$, define the corresponding relation set to two of the integer relations $\leq,=,>,\geq$.

\item Write the following relations on the set $\{1,2,3,4,5\}$ as sets of ordered pairs.
\enumb
\item The $\leq$ relation.
\item The `divides' relation.
\item The $=$ relation.
\enume

\item Each of the following is a relation on the set $\{1,2,3,4,5\}$. Express these relations in words and then find their inverses.
\enumb
\item $\{(1,2),(2,3),(3,4),(4,5)\}$
\item $\{(1,1),(2,1),(2,2),(3,1),(3,2),(3,3),(4,1),(4,2),(4,3),(4,4),(5,1),(5,2),(5,3),(5,4),(5,5)\}$.
\item $\{(1,5),(2,4),(3,3),(4,2),(5,1)\}$.
\enume

\item What is the inverse of the following relations?
\enumb
\item $\leq$.
\item $\{(x,y):x,y\in\integers,x-y=1\}$.
\item $\{(x,y):x,y\in\integers, xy>0\}$.
\enume

\item For each of the following relations on $\{1,2,3,4,5\}$ determine whether the relation is reflexive, irreflexive, symmetric, antisymmetric, and/or transitive:
\itemb
\item $R=\{(1,1),(2,2),(3,3),(4,4),(5,5)$.
\item $R=\{(1,2),(2,3),(3,4),(4,5)\}$.
\item $R=\{(1,1),(1,2),(1,3),(1,4),(1,5)\}$.
\item $R=\{(1,1),(1,2),(2,1),(3,4),(4,3)\}$.
\item $R=\{1,2,3,4,5\}\times\{1,2,3,4,5\}$.
\iteme

\item For the following relations on the set of humans beings, please determine whether the relation if reflexive, irreflexive, symmetric, antisymmetric, and/or transitive.
\enumb
\item has the last name as
\item is the child of
\item is married to
\item has a common parent as
\enume

\item Consider the relation $|$ (divisible) on
\enumb
\item On the naturals.
\item On the integers.
\enume
Decide what properties do they have.

\item Show that the following relation is an equivalence relation:
\[
A=\{B\in 2^{\integers}: |B|<\infty\}, \qquad R=\{(B,C): B,C \in A, |B|=|C|\}
\]

\item Which of the following are equivalence relations?
\enumb
\item $R=\{(1,1),(1,2),(2,1),(2,2),(3,3)\}$ on the set $\{1,2,3\}$.
\item $|$ on $\integers$.
\item $\leq$ on $\integers$.
\item Is-an-anagram-of on the set of English words.
\enume
\item For each of the following congruences, find all integers $N$, with $N>1$, that make the congruence true
\enumb
\item $23\equiv 13$ (mod $N$)
\item $10\equiv 5$ (mod $N$)
\item $6\equiv 60$ (mod $N$)
\enume

%\item There is only one possible equivalence relation on a one-element set: If $A=\{1\}$, then $R=\{(1,1)\}$ is the only possible equivalence relation. On a two element set $A=\{1,2\}$, there are exactly two equivalence relations:
%\[
%R_1=\{(1,1),(2,2)\},\qquad R_2=\{(1,1),(1,2),(2,1),(2,2)\}.
%\]
%How many different equivalence relations are there on a three and on a four element set?

\enume

\end{document}