\documentclass[11pt]{preprint}

\setlength{\topmargin}{0mm} \setlength{\oddsidemargin}{0mm}
\setlength{\textwidth}{160mm} \setlength{\textheight}{215mm}

\usepackage{amssymb,amsmath,amscd,amsthm}

\newtheorem{proposition}{Proposition}

\title{Discrete Mathematics, 2016 Spring - Worksheet 3}
\author{Instructor: Zsolt Pajor-Gyulai, CIMS}

\date{September 14, 2016}



\begin{document}

\maketitle

In all of the above problems explain your answer in full English sentences.

\begin{enumerate}
\item Disprove the following statements:
\begin{enumerate}
\item If $a$ and $b$ are integers with $a|b$, then $a\leq b$.

\vspace{0.1cm}
\textit{For example, $2|-4$ but $2>-4$.}
\vspace{0.1cm}

\item If $a$, $b$, and $c$ are positive integers with $a|(bc)$, then $a|b$ or $a|c$.

\vspace{0.1cm}
\textit{For example, $6|12=3\cdot 4$ but neither $6|3$ nor $6|4$. Thus the statement is false.}
\vspace{0.1cm}

\item If $p$ and $q$ are prime, then $p+q$ is composite.

\vspace{0.1cm}
\textit{For example, $2$ and $3$ are primes, but $5=2+3$ is not composite. Thus the statement is false.}
\vspace{0.1cm}

\item Two right triangles have the same area if and only if the lengths of their hypotenuses are the same.

\vspace{0.1cm}
\textit{Consider two right triangles, one with side length $3, 4, 5$ and one with $\sqrt{12.5}$, $\sqrt{12.5}$, $5$. They both have hypotenuse $5$ but the area of the first one is $3\cdot 4/2=6$, while the area other one is $\sqrt{12.5}\cdot\sqrt{12.5}/2=6.25$. Thus the statement is false.}
\vspace{0.1cm}

\end{enumerate}

\item What does it mean for an if and only if statement to be false? What properties should a couterexample for an if-and-only-if statement have?

\vspace{0.1cm}
\textit{An if and only if statement, $A\Leftrightarrow B$ is false, if there are circumstances under which $A$ is true but $B$ is not, or the other way around. Therefore a counterexample must exhibit such a circumstance.}
\vspace{0.1cm}

\item Evaluate the following Boolean expressions:
\begin{enumerate}
\item $True\wedge True\wedge True\wedge True\wedge False$. \textit{FALSE}
\item $(\neg True)\vee True$. \textit{TRUE}
\item $\neg(True\vee True)$. \textit{FALSE}
\item $(True\vee True)\wedge False$. \textit{FALSE}
\item $True\vee(True\wedge False)$. \textit{TRUE}
\end{enumerate}
\item Prove the following Boolean identities by truth tables:
\begin{enumerate}
\item $\neg(x\wedge y)=(\neg x)\vee (\neg y)$ and $\neg(x\vee y)=(\neg x)\wedge(\neg y)$ (DeMorgan's laws).
\item $x\rightarrow y= (\neg x)\vee y$.
\item $x\leftrightarrow y = (\neg x)\leftrightarrow (\neg y)$.
\end{enumerate}
where $=$ stands for logically equivalent.

\vspace{0.1cm}
\textit{For each part, write down the two truth tables and check that they are identical.}
\vspace{0.1cm}

\item Find a logically equivalent Boolean expression to $x\leftrightarrow y$ only in terms of the basic Boolean operations $\wedge,\vee,\neg$.

\vspace{0.1cm}
\[
x\leftrightarrow y = (x\rightarrow y)\wedge (y\rightarrow x)=((\neg x)\vee y)\wedge ((\neg y)\vee x)
\]
\vspace{0.1cm}

\item How many different binary operation can there be? 

\vspace{0.1cm}
\textit{Since there are 4 different different inputs, each of which we can assign two possible values, this number is $2^4=16$.}
\vspace{0.1cm}

%\item (Hard problem) 
%\begin{enumerate}
%\item Show that all binary operations can be expressed in terms of the three basic Boolean operations.
%\item Show that $x\vee y$ can be expressed in terms of just $\wedge$ and $\neg$.
%\end{enumerate}
%This means that all binary Boolean operations can be reduced to just two binary operations. This fact has great implications in digital technology.

\item A person's inititals are the two-element lists consisting of the initial letters of their first and last names. For example, mines are $ZP$. 
\begin{enumerate}
\item How many possible initials are there?

\vspace{0.1cm}
\textit{Since we have $26$ possibilities both for the first and the second letters, by the multiplication principle, we have $26^2=676$ possible initials.}
\vspace{0.1cm}

\item How many initials are there where the two letters are different?

\vspace{0.1cm}
\textit{Since we have $26$ possibilites for the first letter and once that is chosen we have $25$ possibilities for the second one, the multiplication principle implies that there are $25\cdot 26=650$ such initials.}
\vspace{0.1cm}
\end{enumerate}
\item A club has 10 members.
\begin{enumerate}
\item A club has 10 members who wish to elect a president and a vice-president. How many ways can these positions be filled (assuming the club is not a cult-of personality dictatorship and one person can only have one title)?

\vspace{0.1cm}
\textit{We have $10$ ways to elect the president and once that is done we have $9$ ways to elect a VP. The multiplication principle implies that there are $10\cdot 9 = 90$ possible outcomes of the election.}
\vspace{0.1cm}

\item Now suppose the club also wants to elect a secretary and a treasurer. How many outcomes are there for the election then?

\vspace{0.1cm}
\textit{Once we picked the president (10 ways) and the VP (9 ways) then we have 8 people to choose the treasurer from which leaves us with 7 choices for the treasurer. Thus we have $10\cdot 9\cdot 8\cdot 7=5040$ possible outcomes.}
\vspace{0.1cm}
\end{enumerate}

\item In how many ways can a black rook and a white rook be placed on different squares of a chess board such that neither is attacking the other?

\vspace{0.1cm}
\textit{We can put the black rook anywhere (64 choices) which blocks out exactly one row and a column, so exactly 15 places (don't double count the position of the black rook!). This means that we have to put the white rook somewhere on the remaining $64-15=49$ places. Therefore we have $64\cdot 49 =3136$ total possibilities. }
\vspace{0.2cm}

\noindent\underline{\textbf{Further problems to practice:}}
\vspace{0.1cm}

\item License plates in a cetrtain state consist of six characters: The first three characters ar uppercase letters (A-Z), and the last three characters are digits (0-9).
\begin{enumerate}
\item How many license plates are possible?
\item How many license plates are possible if no character maybe repeated on the same plate?
\end{enumerate}

\item A telephone number (in the US and Canada) is a ten digin number whose first digit cannot be a $0$ or a $1$. How many telephone numbers are possible?

\item A US Social Security number is a nine-digit number. The first digit may be zero.
\begin{enumerate}
\item How many of these are even?
\item How many have all of their digits even?
\item How many read the same backwards? (e.g. 122979221)
\end{enumerate}

\end{enumerate}
\end{document}