\documentclass{beamer}

\mode<presentation>
{
  \usetheme{Frankfurt}
  \usecolortheme{orchid}
  \setbeamercovered{invisible}
  \setbeamertemplate{footline}[frame number]
}

\usepackage[english]{babel}
\usepackage[latin1]{inputenc}
\usepackage{times}
\usepackage[T1]{fontenc}
\usepackage{tikz}
\usepackage{array}
\usepackage{listings}

\usepackage{remreset}
\makeatletter
\@removefromreset{subsection}{section}
\makeatother
\setcounter{subsection}{1}

\title{Discrete Mathematics, Section 002, Spring 2016}
\subtitle{Lecture 2: Theorems, Proofs}
\date{September 12, 2016}

\author[Zsolt]{Zsolt Pajor-Gyulai \\ \texttt{zsolt@cims.nyu.edu}}

\pgfdeclareimage[height=1cm]{NYUlogo}{NYUlogo.jpg}

\institute[NYU] 
{
\normalsize Courant Institute of Mathematical Sciences
}
\titlegraphic{\pgfuseimage{NYUlogo}}

\begin{document}

\begin{frame}
  \titlepage
\end{frame}

\AtBeginSection[]
{
\begin{frame}
\frametitle{Outline}
\tableofcontents[currentsection]
\end{frame}}

\section{Theorems}

\begin{frame}[t]{What is a theorem?}
\begin{block}{}
A theorem is a declarative statement about mathematics for which there is a proof.
\end{block}
In other word it is a sentence that expresses an idea about how something is, that we can assign a value of true or false.
\only<2>{\begin{block}{}\center{If a supply of a commodity decreases then its price increases.}\end{block}}
\only<3>{\begin{block}{}\center{The free acceleration on the surface of the earth is $9.8 m/s^2$.}\end{block}}

\vspace{0.3cm}
\uncover<4->{In mathematics there are four type of statements.}
\begin{itemize}
\item<4-> \textbf{Theorems:} Statements that are known to be true.
\only<5>{\begin{theorem}
If $a$ and $b$ are the length of the legs of a right triangle and $c$ is the length of the hypothenuse, then $a^2+b^2=c^2$.
\end{theorem}}
\only<6->{\item \textbf{Conjecture:} Statements that we suspect to be true but cannot prove it yet.}
\only<7>{\begin{block}{Twin prime conjecture}
There are infinitely many primes $p$ such that $p+2$ is also a prime. (E.g. 47)
\end{block}
}
\item<8->\textbf{Mistake:} A statement that is false.
\only<9>{\begin{block}{}Every even number is divisible by three\end{block}}
\item<10->\textbf{Nonsensical:} A statement that does not make sense.
\only<11>{\begin{block}{} The root of a triangle is a circle.\end{block}}
\end{itemize}

\end{frame}

\begin{frame}[t]{The notion of truth}
Different disciples have different notion of truth. In most disciples a statement is true if it has some appropriate level of predictive value.
\begin{itemize}
\item<2-3> \textbf{Physics:} A statement is true if it has predictive value.
\begin{block}{}
The free acceleration near the surface of the earth is $9.8 m/s^2$.
\end{block}
\begin{itemize}
\item<2-3> The value $9.8$ is an approximation.
\item<2-3> 'Near' is vague.
\item<2-3> The actual value has slight variations near the surface.
\end{itemize}
\item<3> \textbf{Meteorology:}
\begin{block}{} The weather in Baltimore in July is hot and humid.\end{block}
\begin{itemize}
\item<3> Definitely not every day every July.
\end{itemize}

\end{itemize}
\end{frame}

\begin{frame}[t]{The notion of truth}
Different disciples have different notion of truth. In most disciples a statement is true if it has some appropriate level of predictive value or descriptive power.
\begin{itemize}
\item \textbf{Mathematics:}
\begin{itemize}
\item Much stricter than any other disciples.
\item The only truth is absolute, unconditional and without exception.
\begin{theorem}
If $a$ and $b$ are the length of the legs of a right triangle and $c$ is the length of the hypothenuse, then $a^2+b^2=c^2$.
\end{theorem}\pause
\item It is true because we can prove it, not because we measure triangles with a ruler.
\end{itemize}
\end{itemize}\pause
Absolute statements can only be made about abstract objects and therefore mathematics works with these. Then usually our results tell us something about the real world as well.
\end{frame}

\begin{frame}{If $A$ then $B$. ($A\Rightarrow B$).}
\begin{block}{}
\center{If $x$ and $y$ are even integers, then so is $x+y$.}
\end{block}
This is not the same as everyday English usage:
\begin{block}{}
\center{If you mow the lawn, then I will pay you $20$}.
\end{block}
What happens if you don't mow the lawn?\pause
\begin{itemize}
\item Everyday English: You don't get anything.\pause
\item Mathematics: No information.
\end{itemize}
\begin{figure}
\centering
\begin{tabular}{| >{\centering\arraybackslash}b{1in} | >{\centering\arraybackslash}b{1in} || >{\centering\arraybackslash}b{1in}|}
\hline
A& B & \\
\hline
True & True & Possible\\
True & False& Impossible\\
False & True & Possible\\
False & False & Possible\\
\hline
\end{tabular}
\end{figure}

\end{frame}

\begin{frame}{If $A$ then $B$. ($A\Rightarrow B$)}
\textbf{Alternative terminology:}
\begin{itemize}
\item $A$ implies $B$.\pause
\item Whenever $A$ we have $B$.\pause
\item $A$ is sufficient for $B$.\pause
\item $B$ is necessary for $A$.\pause
\item $A$ only if $B$.\pause
\item $B\Leftarrow A$. 
\end{itemize}\pause

\end{frame}

\begin{frame}[fragile]
\textbf{Vacuous truth:}
\begin{block}{}
If an integer is both a perfect square and prime, then it is negative.
\end{block}
$A\Rightarrow B$ is only false if it is possible for $A$ to be true but $B$ to be false at the same time.$\rightarrow$ This is true!
\vspace{0.5cm}

\underline{An algorithmic explanation:}
\begin{lstlisting}[language=Python]
def evaluate():
    for a in Integers:
        if (a in PerfSq) and (a in Primes) ...
                         and (a > 0):
            return False		
	
    return True
\end{lstlisting}
\end{frame}

\begin{frame}{$A$ if and only if $B$. ($A\Leftrightarrow B$)}
\begin{block}{}
\center{If an integer $x$ is even, then $x+1$ is odd, and if $x+1$ is odd, then $x$ is even.}
\end{block}
This is inconveniently long. Rather:
\begin{block}{}
\center{An integer $x$ is even if and only if $x+1$ is odd.}
\end{block}

\begin{figure}
\centering
\begin{tabular}{| >{\centering\arraybackslash}b{1in} | >{\centering\arraybackslash}b{1in} || >{\centering\arraybackslash}b{1in}|}
\hline
A& B & \\
\hline
True & True & Possible\\
True & False& Impossible\\
False & True & Impossible\\
False & False & Possible\\
\hline
\end{tabular}
\end{figure}
Alternative terminology:
\begin{itemize}
\item $A$ iff $B$.\pause
\item $A$ is necessary and sufficient for $B$.\pause
\item $A$ is equivalent to $B$.
\end{itemize}
\end{frame}

\begin{frame}{And, Or and Not}

\begin{figure}
\centering
\begin{tabular}{| >{\centering\arraybackslash}b{1in} | >{\centering\arraybackslash}b{1in} || >{\centering\arraybackslash}b{1in}|}
\hline
A& B & A and B \\
\hline
True & True & True\\
True & False& False\\
False & True & False\\
False & False & False\\
\hline
\end{tabular}
\end{figure}

\begin{figure}
\centering
\begin{tabular}{| >{\centering\arraybackslash}b{1in} | >{\centering\arraybackslash}b{1in} || >{\centering\arraybackslash}b{1in}|}
\hline
A& B & A or B \\
\hline
\alert{True} & \alert{True} & \alert{True}\\
True & False& True\\
False & True & True\\
False & False & False\\
\hline
\end{tabular}
\end{figure}

\begin{figure}
\centering
\begin{tabular}{| >{\centering\arraybackslash}b{1in}  || >{\centering\arraybackslash}b{1in}|}
\hline
A& Not A \\
\hline
True & False\\
False & True\\
\hline
\end{tabular}
\end{figure}

\end{frame}

\begin{frame}{What theorems are called}
\begin{itemize}
\item \textbf{Result} A modest word for a theorem.\pause
\item \textbf{Fact} A very minor theorem. \pause
\item \textbf{Proposition} A minor theorem.\pause
\item \textbf{Lemma} A theorem whose main purpose is to be used in the proof of a more important theorem.\pause
\item \textbf{Corollary} A result with a short proof whose main step is to use a previously proved bigger theorem.\pause
\item \textbf{Claim} Similar to a lemma but it usually appears inside the proof of a theorem to help structure it.
\end{itemize}
\end{frame}

\section{Proof}

\begin{frame}{What is a proof?}
We create mathematical concepts via definitions and then posit assertions about them as theorems. Finally, we have to prove that our assertions are correct.\pause
\begin{block}{}
\center{All prime numbers are odd}
\end{block}
\pause
We cannot prove this by experimentation:
\[
3,5,7,\cdots \qquad\textrm{all odd.}
\]\pause
Nevertheless, $2$ is not odd and therefore the statement is false.
\end{frame}

\begin{frame}{What is a proof?}
Another example:
\begin{block}{Goldbach's Conjecture}
Every even integer greater than two is the sum of two primes.
\end{block}\pause
\begin{align*}
4&=2+2 &6&=3+3 &8&=3+5&10&=3+7\\
12&=5+7  &14&=7+7&16&=11+5&18&=11+7
\end{align*}
A computer can tell you that the first few billion even numbers are good too.\pause
\center{\alert{We still do not know if we can or cannot find a larger one that would break the conjecture!}}
\end{frame}

\begin{frame}{What is a proof?}
\begin{block}{}
A proof is an essay that incontrovertibly shows that a statement is true.
\end{block}\pause
\begin{itemize}
\item Mathematical proofs are highly structured \pause
\item They are written in a stylized manner using logical constructions.\pause
\item Ultimately, writing proofs is an art and you can only learn doing it well by doing it a lot.
\end{itemize}

\end{frame}

\begin{frame}{An example}

\begin{theorem}
The sum of two even integers is even.
\end{theorem}\pause
\vspace{0.4cm}
We will use:

\begin{block}{Definition 1}
An integer is called \textbf{even} if it is divisible by two.
\end{block}

\begin{block}{Definition 2}
Let $a$ and $b$ be integers. We say that $a$ is \textbf{divisible} by $b$ provided there is an integer $c$ such that $bc=a$.
\end{block}

\end{frame}

\begin{frame}


\begin{theorem}
The sum of two even integers is even.
\end{theorem}
In a math paper the proof would look like this:
\begin{proof}
~~~~We show that if $x$ and $y$ are even integers, then $x+y$ is an even integer.

~~~~ Let $x$ and $y$ be even integers. Since $x$ is even, we know by Definition 1 that $2|x$. By Definition 2, this implies that there is an integer $a$ such that $x=2a$. Likewise, since $y$ is even, we have $2|y$ and therefore there is another integer $b$ such that $y=2b$. Observe that
\[
x+y=2a+2b=2(a+b).
\]
This means that there is an integer $c(=a+b)$ such that $x+y=2c$, which in turn implies $2|x+y$ which in turn implies that $x+y$ is even.
\end{proof}
\end{frame}

\begin{frame}{Let's break it down}
\begin{block}{}
1. 'We show that if $x$ and $y$ are even integers, then $x+y$ is an even integer.'
\end{block}\pause
\begin{itemize}
\item Spells out the proposition in an $if-then$ form.\pause
\item Announces the structure of the proof.\pause
\item Can be considered a preamble.\pause
\item Unless it serves a purpose, you can omit it. However, it is in good style to include it whenever it makes matters clearer.
\end{itemize}\pause

\begin{block}{}
2. 'Let $x$ and $y$ be even integers.'
\end{block}\pause

\begin{itemize}
\item We introduce names $x$, $y$ for the two integers.\pause
\item By this we do the proof for all integers at once!
\end{itemize}

\end{frame}

\begin{frame}{Let's break it down}
\begin{block}{}
3. 'Since $x$ is even, we know by Definition 1 that $2|x$.'
\end{block}\pause
\begin{itemize}
\item We unravel the definition of even.
\end{itemize}\pause

\begin{block}{}
4. 'By Definition 2, this implies that there is an integer $a$ such that $x=2a$.'
\end{block}\pause
\begin{itemize}
\item We unravel the definition of divisible.\pause
\item A new integer comes into play, we do not know what it is as it depends on $x$. But we know that there is one! We denote it by $a$ whatever it is.
\end{itemize}

\end{frame}

\begin{frame}{Let's break it down}
\begin{block}{}
5. 'Likewise, since $y$ is even, we have $2|y$ and therefore there is another integer $b$ such that $y=2b$.'
\end{block}\pause
\begin{itemize}
\item We do the same thing as in $3-4$ but this time for $y$. No reason to repeat the argument, we just indicate that it is the same by 'Likewise'.
\end{itemize}\pause
\begin{block}{}
6. 'This means that there is an integer $c(=a+b)$ such that $x+y=2c$, which in turn implies $2|x+y$'
\end{block}\pause
\begin{itemize}
\item Unravel the definitions the exact same way as before but at the end of the proof!
\end{itemize}

\end{frame}

\begin{frame}
\uncover<1>{At this point:}
\begin{proof}
\uncover<1->{~~~~We show that if $x$ and $y$ are even integers, then $x+y$ is an even integer.

~~~~ \alert{Let $x$ and $y$ be even integers.} Since $x$ is even, we know by Definition 1 that $2|x$. \alert{By Definition 2, this implies that there is an integer $a$ such that $x=2a$.} Likewise, since $y$ is even, we have $2|y$ and therefore there is another integer $b$ such that $y=2b$.} \uncover<4->{\color{blue}Observe that
\[
x+y=2a+2b=2(a+b).
\]
This means that\color{black}~ }\uncover<1->{there is an integer $c(=a+b)$ such that $x+y=2c$, which in turn implies $2|x+y$ \alert{which in turn implies that $x+y$ is even.}}
\end{proof}

\begin{itemize}
\item<2-> We make the connection between the beginning and the end.
\item<3-> This is where we have to think.
\end{itemize}

\end{frame}

\begin{frame}{Proof template}
\begin{block}{{Proof template for a direct proof of an if-then theorem}}
\begin{enumerate}
\item Write the first sentence(s) of the proof by restating the hypothesis as an if-then statement. Invent suitable notation, assign letters as names to variables, etc.\pause
\item Write the last sentence of the proof by restating the conclusion of the result.\pause
\item Unravel the definitions, working forward from the beginning of the proof and backward from the end.\pause
\item Figure out what you know and what you need. Try to forge an argument.
\end{enumerate}
\end{block}

\end{frame}

\begin{frame}{Another example}
\begin{theorem}
Let $a$, $b$, $c$ be integers. If $a|b$ and $b|c$ then $a|c$.
\end{theorem}\pause
This on your worksheet, give it a try now!
\end{frame}

\begin{frame}{A more involved example}
Pick a positive integer, cube it and then add it to one:
\begin{align*}
3^3+1&=27+1=28=2\cdot 14\\
4^3+1&=64+1=65=5\cdot 13\\
5^3+1&=125+1=126=2\cdot 63\\
6^3+1&=216+1=217=7\cdot 31\\
\vdots\\
x^3+1&= composite?
\end{align*}\pause
Try to formulate a theorem:
\begin{block}{Theorem (Draft 1)}
If $x$ is an integer, then $x^3+1$ is composite.
\end{block}\pause
\center{Is this okay?}
\end{frame}

\begin{frame}

\begin{block}{Theorem (Draft 1)}
If $x$ is an integer, then $x^3+1$ is composite.
\end{block}
Recall:
\begin{definition}
A \alert{positive} integer $a$ is called \textbf{composite} provided there is an integer $b$ such that $1<b<a$ and $b|a$.
\end{definition}\pause
\center{This is a problem because $x^3+1$ can be negative or zero!}\pause
\begin{block}{Theorem (Draft 2)}
If $x$ is a positive integer, then $x^3+1$ is composite.
\end{block}\pause
But if $x=1$ then $x^3+1=2$ which is not composite.
\end{frame}

\begin{frame}[t]
\begin{block}{Theorem}
Let $x$ be an integer. If $x>1$, then $x^3+1$ is composite.
\end{block}
\begin{proof}
\uncover<2->{~~~~Let $x$ be an integer and suppose $x>1$.} \uncover<4->{\phantom{Note that $x^3+1=(x+1)(x^2-x+1)$. Because $x$ is an integer, both $x+1$ and $x^2-x+1$ are integers. Therefore $(x+1)|(x^3+1)$.}

~~~~\phantom{Since $x>1$, we have $x+1>1+1=2>1$.}

~~~~\phantom{Also $x>1$ implies $x^2>x$, and since $x>1$, we have $x^2>1$. Multiplying both sides by $x$ again yields $x^3>x$. Adding $1$ to both sides gives $x^3+1>x+1$.}

~~~~\phantom{Thus $x+1$ is an integer with $1<x+1<x^3+1$.}}

~~~~\uncover<3->{\color{blue}Since\color{black}} \uncover<4->{\phantom{$x+1$}}\uncover<3->{\color{blue}~is a divisor of $x^3+1$ and $1<$\phantom{$x+1$}$<x^3+1$, }\uncover<2->{\color{black}we have that $x^3+1$ is composite}.
\end{proof}


\only<2>{\center{The statement is already if-then, no need to repeat it.}}
\only<3->{\center{Unraveling the definition in the end.}}

\only<4>{\center{What should be the right divisor?}}
\end{frame}

\begin{frame}
\begin{block}{}
Since \phantom{$x+1$} is a divisor of $x^3+1$ and $1<$\phantom{$x+1$}$<x^3+1$, we have that $x^3+1$ is composite.
\end{block}\pause
Recall:
\begin{definition}
Let $a$ and $b$ integers. If $a|b$ and $1<a<b$ then $a$ is called a \textbf{factor} of $b$.
\end{definition}\pause
We need to find a factor!\pause
\[
x^3+1=(x+1)(x^2-x+1)
\]
and so $x+1$ and $x^2-x+1$ are both factors of $x^3+1$ as they are both integers (since $x$ is an integer).


\end{frame}

\begin{frame}[t]
\begin{block}{Theorem}
Let $x$ be an integer. If $x>1$, then $x^3+1$ is composite.
\end{block}
\begin{proof}
~~~~Let $x$ be an integer and suppose $x>1$. \color{blue}~Note that $x^3+1=(x+1)(x^2-x+1)$. Because $x$ is an integer, both $x+1$ and $x^2-x+1$ are integers. Therefore $(x+1)|(x^3+1)$.\color{black}

~~~~\phantom{Since $x>1$, we have $x+1>1+1=2>1$.}

~~~~\phantom{Also $x>1$ implies $x^2>x$, and since $x>1$, we have $x^2>1$. Multiplying both sides by $x$ again yields $x^3>x$. Adding $1$ to both sides gives $x^3+1>x+1$.}

~~~~\phantom{Thus $x+1$ is an integer with $1<x+1<x^3+1$.}

~~~~Since \color{blue}~$x+1$\color{black}~is a divisor of $x^3+1$ and $1<$\phantom{$x+1$}$<x^3+1$, we have that $x^3+1$ is composite.
\end{proof}


\only<2>{\center{We are not done yet, as we still need to show that $x+1$\\ 'fits into the gap'.}}
\end{frame}

\begin{frame}
We need
\begin{block}{}
\[
1<x+1<x^3+1
\]
\end{block}\pause
First inequality: 
\[
x>1\qquad\rightarrow\qquad x+1>1+1=2>1
\]\pause
The second inequality:
\begin{align*}
x>1\qquad&\rightarrow\qquad x^2>x>1 &(\textrm{multiplying by $x$})\\
&\rightarrow\qquad x^3>x &(\textrm{multiplying by $x$})\\
&\rightarrow\qquad x^3+1>x+1 &(\textrm{adding $1$})
\end{align*}
\end{frame}

\begin{frame}[t]
\begin{block}{Theorem}
Let $x$ be an integer. If $x>1$, then $x^3+1$ is composite.
\end{block}
\begin{proof}
~~~~Let $x$ be an integer and suppose $x>1$. Note that $x^3+1=(x+1)(x^2-x+1)$. Because $x$ is an integer, both $x+1$ and $x^2-x+1$ are integers. Therefore $(x+1)|(x^3+1)$.

~~~~ \color{blue}Since $x>1$, we have $x+1>1+1=2>1$.

~~~~Also $x>1$ implies $x^2>x$, and since $x>1$, we have $x^2>1$. Multiplying both sides by $x$ again yields $x^3>x$. Adding $1$ to both sides gives $x^3+1>x+1$.

~~~~Thus $x+1$ is an integer with $1<x+1<x^3+1$.\color{black}

~~~~Since $x+1$ is a divisor of $x^3+1$ and $1<x+1<x^3+1$, we have that $x^3+1$ is composite.
\end{proof}

\end{frame}

\begin{frame}{If-and-only-if Theorems}
\begin{block}{Direct proof of an if-and-only-if theorem}
To prove a statement of the form '$A$ iff $B$',
\begin{itemize}
\item ($\Rightarrow$) Prove 'If $A$, then $B$'.
\item ($\Leftarrow$) Prove 'If $B$, then $A$'.
\end{itemize}
\end{block}\pause
\begin{theorem}
Let $x$ be an integer. Then $x$ is even if and only if $x+1$ is odd.
\end{theorem}\pause
\begin{proof}
Let $x$ be an integer.

~~~~($\Rightarrow$) Suppose $x$ is even, ~~$\dots$ Therefore $x+1$ is odd.

~~~~($\Leftarrow$) Suppose $x+1$ is odd,~~$\dots$ Therefore $x$ is even.
\end{proof}

Now prove the if-then substatements as before. (Worksheet)

\end{frame}


\end{document}