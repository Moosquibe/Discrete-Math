\documentclass{beamer}

\mode<presentation>
{
  \usetheme{Frankfurt}
  \usecolortheme{orchid}
  \setbeamercovered{invisible}
  \setbeamertemplate{footline}[frame number]
}

\usepackage[english]{babel}
\usepackage[latin1]{inputenc}
\usepackage{times}
\usepackage[T1]{fontenc}
\usepackage{tikz}
\usepackage{array}
\usepackage{cancel}


\usetikzlibrary{shapes,backgrounds}

\def\multiset#1#2{\ensuremath{\left(\kern-.3em\left(\genfrac{}{}{0pt}{}{#1}{#2}\right)\kern-.3em\right)}}

\def\blue{\color{blue}~}
\def\black{\color{black}~}
\def\bl[#1]#2{\begin{block}{#1}#2\end{block}}
\def\integers{\mathbb{Z}}
\def\enumb{\begin{enumerate}}
\def\enume{\end{enumerate}}
\def\itemb{\begin{itemize}}
\def\iteme{\end{itemize}}
\def\div{~\textrm{div}~}
\def\mod{~\textrm{mod}~}


\usepackage{remreset}
\makeatletter
\@removefromreset{subsection}{section}
\makeatother
\setcounter{subsection}{1}

\title{Discrete Mathematics, Section 001, Fall 2016}
\subtitle{Lecture 20: Chinese remainder theorem, Factoring}
\date{November 30, 2016}

\author[Zsolt]{Zsolt Pajor-Gyulai \\ \texttt{zsolt@cims.nyu.edu}}

\pgfdeclareimage[height=1cm]{NYUlogo}{NYUlogo.jpg}

\institute[NYU] 
{
\normalsize Courant Institute of Mathematical Sciences
}
\titlegraphic{\pgfuseimage{NYUlogo}}

\begin{document}

\begin{frame}
  \titlepage
\end{frame}

\AtBeginSection[]
{
\begin{frame}
\frametitle{Outline}
\tableofcontents[currentsection]
\end{frame}}

\section{Chinese remainder theorem}

\begin{frame}{Solving two equations}
We have seen how to solve one congruence and noted that there were an infinite number of solutions. Now we want to solve two congruences simultaneously.
\begin{align*}
x&\equiv a ~({\rm mod}~m)\\
x&\equiv b ~({\rm mod}~n)
\end{align*}
We start with the example
\begin{align*}
x&\equiv 1 ~({\rm mod}~7)\\
x&\equiv 4 ~({\rm mod}~11)
\end{align*}
Our task is to find all integers $x$ that satisfies both equations.
\end{frame}

\begin{frame}{Solving two equations}
We start with the example
\begin{align*}
x&\equiv 1 ~({\rm mod}~7)\\
x&\equiv 4 ~({\rm mod}~11)
\end{align*}
By the first congruence, there is an integer $k$, such that
\[
1+7k=x\equiv 4~({\rm mod} 11)\qquad\Rightarrow\qquad 7k\equiv 3~({\rm mod}~11)
\]
\begin{center}
\textcolor{red}{We reduced it to a single equation!}
\end{center}
Solution (Problem 1 on WS):
\[
k = 2 + 11j,\qquad j\in\mathbb{Z},
\]
and therefore
\[
x=1+7(2+11j)=15+77j,\qquad j\in\mathbb{Z}.
\]
\end{frame}


\begin{frame}
\bl[Chinese remainder theorem]{
The pair of congruences where $gcd(m,n)=1$,\vspace{-0.3cm}
\[
x\equiv x_1\textrm{ (mod $m$)},\qquad x\equiv x_2 \textrm{ (mod $n$)}
\]\vspace{-0.7cm}\\
has a unique solution $x_0$ with $0\leq x_0<mn$. Furthermore, every mutual solution to these congruences differs from $x_0$ by a multiple of $mn$.
}
WLOG assume $m<n$. From $x\equiv x_1$ $(\textrm{mod } m)$, we have $x=x_1+km$ where $k\in \mathbb{Z}$. Plugging this into the other equation
\[
x_1+km\equiv x_2\textrm{ (mod $n$)}\qquad\to\qquad km\equiv x_2-x_1\textrm{ (mod n)}
\]
Note that adding or subtracting a multiple of $n$ from either side does not change this equation and therefore let 
\[
c=x_2-x_1\textrm{ mod }n.
\]
This yields the equation $km\equiv c\textrm{ (mod $n$)}$ which is also
\[
(k\textrm{ mod } n)\otimes m=c,\qquad\textrm{in $\mathbb{Z}_{n}$}
\]\vspace{-1cm}

[...]
\end{frame}
\begin{frame}
\bl[Chinese remainder theorem]{
The pair of congruences\vspace{-0.3cm}
\[
x\equiv x_1\textrm{ (mod $m$)},\qquad x\equiv x_2 \textrm{ (mod $n$)}
\]\vspace{-0.7cm}\\
has a unique solution $x_0$ with $0\leq x_0<mn$. Furthermore, every mutual solution to these congruences differs from $x_0$ by a multiple of $mn$.
}
[...]
\[
(k\textrm{ mod }n)\otimes m=c,\qquad\textrm{in $\mathbb{Z}_{n}$}
\]
Since $gcd(m,n)=1$, $m$ has a reciprocal $m^{-1}$ in $\mathbb{Z}_n$. Then
\[
(k\textrm{ mod }n)=c\otimes m^{-1}=:d,\qquad\textrm{in $\mathbb{Z}_{n}$}.
\]
This can be written as
\[
k=d+jn,\qquad\textrm{for some $j\in\mathbb{Z}$}.
\]
[...]
\end{frame}

\begin{frame}
\bl[Chinese remainder theorem]{
The pair of congruences\vspace{-0.3cm}
\[
x\equiv x_1\textrm{ (mod $m$)},\qquad x\equiv x_2 \textrm{ (mod $n$)}
\]\vspace{-0.7cm}\\
has a unique solution $x_0$ with $0\leq x_0<mn$. Furthermore, every mutual solution to these congruences differs from $x_0$ by a multiple of $mn$.
}
[...]
\[
k=d+jn,\qquad\textrm{for some $j\in\mathbb{Z}$}.
\]
Plugging this back into $x=x_1+km$, we get
\[
x=x_1+(d+jn)m=x_1+dm+jnm
\]
from which
\[
x_0=x_1+dm\textrm{ mod }nm
\]
with $d=((x_2-x_1)\textrm{ mod } n)\otimes m^{-1}$ where the reciprocal is taken in $\mathbb{Z}_n$.\qed
\end{frame}

\section{Factoring}

\begin{frame}{Fundamental theorem of arithmetics}
\bl[Theorem]{Let $n$ be a positive integer. Then $n$ factors into a product of primes. Furthermore, this factorization is unique up to the order of the primes}
In other words, for every positive integer $n$, there is $l\in\mathbb{N}$ and primes $p_1,\dots,p_l$, such that 
\[
n=\prod_{i=1}^lp_i
\]
and the only ambiguity lies in reidexing the $p_i$.

\center Believing this is true, do Problem 3 on the worksheet!
\end{frame}

\begin{frame}{Auxilliary lemma 1}
\bl[Lemma]{Suppose $a,b,p\in\mathbb{Z}$ and $p$ is a prime. If $p|ab$, then $p|a$ or $p|b$.}
\bl[Proof]{
Suppose for the sake of contradiction that there are integers $a,b,p$ such that $p|ab$ but $p$ divides neither $a$ nor $b$.\\
~~~~Since $p$ is a prime, its only divisors are $\pm 1,\pm p$. We also know $p\not|~a$, and thus $gcd(a,p)=1$. Similarly, $gcd(b,p)=1$. \\
~~~~By two lectures before,
\[
ax+py=1,\qquad bz+pw=1
\]
for some integers $x,y,z,w$.\\
$[...]$
}

\end{frame}

\begin{frame}{Auxiliary Lemma 1}
\bl[Lemma]{Suppose $a,b,p\in\mathbb{Z}$ and $p$ is a prime. If $p|ab$, then $p|a$ or $p|b$.}

\bl[Proof]{
$[...]$\\
~~~~By two lectures before,
\[
ax+py=1,\qquad bz+pw=1
\]
for some integers $x,y,z,w$. Multiplying these equations, we get
\[
1=(ax+py)(bz+pw)=abxz+pybz+paxw+p^2yw.
\]
Since $p|ab$, all four terms are divisible by $p$ and thus $p|1$. $\Rightarrow\Leftarrow$\qed
}
\end{frame}

\begin{frame}{Auxiliary Lemma 2}
\bl[Lemma]{Suppoese $p,q_1,\dots, q_t$ are prime numbers. If\vspace{-0.2cm}
\[
p|(q_1\dots q_t),
\]\vspace{-0.6cm}\\
then $p=q_i$ for some $1\leq i\leq t$.}

\bl[Proof]{
We prove this by induction on $t$.\\
~~~The base case $t=1$ is clear as if $p|q_1$ then since $p$ and $q_1$ are primes, $p=q_1$.\\
~~~Suppose this is true for $t=k$ and consider $p|(q_1\dots q_kq_{k+1})$. Let\vspace{-0.2cm}
\[
a=q_1\dots q_k,\qquad b=q_{k+1}
\]\vspace{-0.6cm}\\
Then $p|ab$ and by the previous lemma, either $p|a$ or $p|b$.\\
$[...]$
}
\end{frame}

\begin{frame}{Auxiliary Lemma 2}
\bl[Lemma]{Suppoese $p,q_1,\dots, q_t$ are prime numbers. If\vspace{-0.2cm}
\[
p|(q_1\dots q_t),
\]\vspace{-0.6cm}\\
then $p=q_i$ for some $1\leq i\leq t$.}
\bl[Proof]{
$[...]$\vspace{-0.3cm}
\[
a=q_1\dots q_k,\qquad b=q_{k+1}
\]\vspace{-0.6cm}\\
Then $p|ab$ and by the previous lemma, either $p|a$ or $p|b$.
\itemb
\item If $p|a=q_1\dots q_k$, then by the induction hypothesis, $p=q_i$ for some $1\leq i\leq t$.
\item If $p|b=q_{k+1}$ then $p=q_{k+1}$ as before.
\iteme
In either case, the result is proven for $t=k+1$ and therefore by induction the result holds for all $t$.\qed
}

\end{frame}

\begin{frame}{Proof of existence of factorization}
~~~~Suppose for the sake of contradiction, that not all positive integers factor into primes and let $X$ be the set of these numbers.
\itemb
\item $1=\prod_{i=1}^{0}p_i$ (empty product) and so $1\notin X$.
\item If $p$ is a prime itself, then $p\notin X$.
\iteme
~~~~By the WOP, let $x$ be the least element of $X$. We have $x\neq 1$ and that $x$ is not a prime. Therefore $x$ is composite.

~~~~This means that there is an $a\in\mathbb{Z}$ with $1<a<x$ such that $a|x$. Then there is a $b\in\mathbb{Z}$ such that $ab=x$ and clearly, $1<b<x$. 

~~~~Sice $a<x$ and $b<x$, they have a factorization, let's say given by\vspace{-0.2cm}
\[
a=p_1\dots p_s,\qquad b=q_1\dots q_t
\]\vspace{-0.6cm}\\
Then \vspace{-0.2cm}
\[
x=ab=p_1\dots p_sq_1\dots q_t
\]\vspace{-0.6cm}\\
is a factorization of $x$ contradicting $x\in X$.$\Rightarrow\Leftarrow$.
\end{frame}

\begin{frame}{Proof of Uniqueness of factorization}
Suppose for the sake of contradiction that some positive integer can be factored into primes in two distinct ways and let $Y$ be the set of these numbers.
\itemb
\item $1\notin Y$ as the empty product is the only representation.
\iteme
By the WOP, let $y$ be the least element of $Y$. Thus $y$ can be factored into primes in two distinct ways:
\[
y=p_1p_2\dots p_s,\qquad y=q_1q_2\dots q_t
\]
Note that $p_1|y=q_1q_2\dots q_t$ and therefore by Aux Lemma 2, $p_1=q_j$ for some $j$. Then
\[
\frac{y}{p_1}=p_2\dots p_j,\qquad \frac{y}{p_1}=\prod_{i=1\atop i\neq j}^tq_j.
\]
But $\frac{y}{p_1}<y$ and it has two distinct factorization which contradict $y$ being the smallest element of $Y$.$\Rightarrow\Leftarrow$.
\end{frame}

\section{Applications of the factorization theorem}

\begin{frame}{Infinitely many primes}
\bl[Theorem]{There are infinitely many prime numbers}
FTSC, assume that there are finitely many primes: 
\[
2,3,5,7,\dots,p.
\]
Let $n=(2\cdot 3\cdot 5\cdot\dots\cdot p)+1$. 
\itemb
\item Is $n$ a prime? Clearly, $n>p$ and therefore it is not, so $n$ is composite.
\item Let $q$ be any prime and note
\[
n=(2\cdot 3\cdot\dots\cdot q\cdot\dots\cdot p)+1.
\]
Then $n\mod q=1$ and $q\not|~ n$. Since the choice of the prime $q$ was arbitrary, there is no prime factorization of $n$.    $\Rightarrow\Leftarrow$ \qed
\iteme
\end{frame}

\begin{frame}{Formula for greatest common divisor}
\bl[Theorem]{Let $a$ and $b$ be positive integers with
\[
a=2^{e_2}\cdot 3^{e_3}\cdot 5^{e_2}\cdot 7^{e_7}\dots,\qquad b=2^{f_2}\cdot 3^{f_3}\cdot 5^{f_5}\cdot 7^{f_7}\cdot\dots,
\]
where $e_i$-s are naturals (possibly zero). Then 
\[
gcd(a,b)=2^{min(e_2,f_2)}\cdot 3^{min(e_3,f_3)}\cdot 5^{min(e_5,f_5)}\cdot 7^{min(e_7,f_7)}\cdot\dots
\]}
For example, if $a=24$, $b=30$,
\[
24=2^3\cdot 3^1\cdot 5^0\cdot 7^0\cdot \dots,\qquad 30=2^1\cdot 3^1\cdot 5^1\cdot 7^0\cdot\dots
\]
and
\[
gcd(24,30)=2^{1}\cdot 3^1\cdot 5^0\cdot 7^0\cdot\dots=6.
\]
\begin{center}
Do Problem 4 on the worksheet!
\end{center}
\end{frame}

\begin{frame}{Irrationality of $\sqrt{2}$}
Do Problem 6 on the worksheet first!
\bl[Proposition]{There is no rational number $x$ such that $x^2=2$.}
FTSC, suppose that there is a rational number $x=\frac{a}{b}$ where $a$ and $b$ are integers, such that $x^2=2$.

This implies $\left(\frac{a}{b}\right)^2=2$ which in turn implies $a^2=2b^2$. Consider the prime factorization of $n=a^2=2b^2$.
\itemb
\item $2$ must appear in the factorization of $n=a^2$ an even number of times.
\item $2$ must appear in the factorization of $n=2b^2$ an odd number of times.
\iteme
$\Rightarrow\Leftarrow$\qed
\end{frame}


\end{document}