\documentclass[11pt]{preprint}

\setlength{\topmargin}{0mm} \setlength{\oddsidemargin}{0mm}
\setlength{\textwidth}{160mm} \setlength{\textheight}{215mm}

\usepackage{amssymb,amsmath,amscd,amsthm}
\usepackage{graphics}
\usepackage{tikz}

\def\enumb{\begin{enumerate}}
\def\enume{\end{enumerate}}
\def\itemb{\begin{itemize}}
\def\iteme{\end{itemize}}
\def\integers{\mathbb{Z}}

\def\multiset#1#2{\ensuremath{\left(\kern-.3em\left(\genfrac{}{}{0pt}{}{#1}{#2}\right)\kern-.3em\right)}}



\newtheorem{proposition}{Proposition}
\newtheorem{theorem}{Theorem}

\title{Discrete Mathematics, 2016 Fall - Worksheet 18}
\author{Instructor: Zsolt Pajor-Gyulai, CIMS}
\date{November 30, 2016}



\begin{document}

\maketitle

In all of the above problems explain your answer in full English sentences.

\enumb
\item Solve the single congruence
\[
7k\equiv 3~({\rm mod}~11)
\]

Note that $7^{-1}=8$ in $\mathbb{Z}_11$. Therefore $k_0=7^{-1}\otimes 3=8\otimes 3=2$ and the general solution of the congruence is given by
\[
k=2+11j,\qquad j\in\mathbb{Z}
\]

\item Solve the following system of equation
\[
x\equiv 4~({\rm mod}~5),\qquad x\equiv 7~({\rm mod}~11)
\]

By the first congruence, we get $x=4+5k$ for some integer $k$. Plugging this back into the second one we get
\[
4+5k\equiv 7~(11)\qquad\to\qquad 5k\equiv 3~(11)
\]
Since $5^{-1}=9$, we have that $k_0=9\otimes 3 =5$ and therefore
\[
k=5+11j\qquad j\in\mathbb{Z},
\]
which implies
\[
x=4+5(5+11j)=29+55j\qquad j\in\mathbb{Z}
\]
\item Factor the following positive integers into primes.
\enumb
\item $25=5\cdot 5$
\item $4200=2\cdot 2100=2\cdot 3\cdot 7\cdot 100=2^3\cdot 3\cdot 5^2\cdot 7$
\item $10^{10}=2^{10}\cdot 5^{10}$
\item $19=19$
\item $1=1$
\enume
\item Let $a$ and $b$ be positive integers. Prove that $a$ and $b$ are relatively prime if and only if there is no prime $p$ such that $p|a$ and $p|b$.

\begin{proof}
 The only if part is easy, we prove the contrapositive, i.e. that the existence of a prime $p$ with $p|a$ and $p|b$ implies that $a$ and $b$ are not relatively prime. But this is straightforward because then $p$ is a common divisor of $a$ and $b$ and therefore $gcd(a,b)\geq p$.
 
 To prove the if part assume that there are no primes dividing both $a$ and $b$. If the prime factorizations are
\[
a =\dots p^{\alpha}\dots,\qquad b=\dots p^{\beta}\dots.
\]
then clearly either $\alpha$ or $\beta$ are zero and hence $\min(\alpha,\beta)=0$ for all primes $p$. By the formula for the gcd in terms of prime factorizations, this clearly implies $gcd(a,b)=1$.
\end{proof}

\item Let $a$ and $b$ be positive integers. Prove that $2^a$ and $2^b-1$ are relatively prime by considering their prime factorizations.

\begin{proof}
Note that the prime factorization of $2^a$ consists only of $2$-s. However $2^b-1$ is an odd number and therefore there are no $2$-s in their prime factorization. 
\end{proof}

\item Prove that if $a,p\in\mathbb{Z}$ with $p$ prime and $p|a^2$, then $p|a$.

\begin{proof}
Note that $p|a^2$ is $p|a\cdot a$ and by the auxiliary lemma in class, we get $p|a$.
\end{proof}

\enume
\end{document}