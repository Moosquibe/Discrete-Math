\documentclass{beamer}

\mode<presentation>
{
  \usetheme{Frankfurt}
  \usecolortheme{orchid}
  \setbeamercovered{invisible}
  \setbeamertemplate{footline}[frame number]
}

\usepackage[english]{babel}
\usepackage[latin1]{inputenc}
\usepackage{times}
\usepackage[T1]{fontenc}
\usepackage{tikz}
\usepackage{array}
\usepackage{cancel}


\usetikzlibrary{shapes,backgrounds}

\def\multiset#1#2{\ensuremath{\left(\kern-.3em\left(\genfrac{}{}{0pt}{}{#1}{#2}\right)\kern-.3em\right)}}

\def\blue{\color{blue}~}
\def\black{\color{black}~}
\def\bl[#1]#2{\begin{block}{#1}#2\end{block}}
\def\integers{\mathbb{Z}}
\def\enumb{\begin{enumerate}}
\def\enume{\end{enumerate}}
\def\itemb{\begin{itemize}}
\def\iteme{\end{itemize}}


\usepackage{remreset}
\makeatletter
\@removefromreset{subsection}{section}
\makeatother
\setcounter{subsection}{1}

\title{Discrete Mathematics, Section 001, Fall 2016}
\subtitle{Lecture 12: Linear recurrence relations}
\date{October 24, 2016}

\author[Zsolt]{Zsolt Pajor-Gyulai \\ \texttt{zsolt@cims.nyu.edu}}

\pgfdeclareimage[height=1cm]{NYUlogo}{NYUlogo.jpg}

\institute[NYU] 
{
\normalsize Courant Institute of Mathematical Sciences
}
\titlegraphic{\pgfuseimage{NYUlogo}}

\begin{document}

\begin{frame}
  \titlepage
\end{frame}

\AtBeginSection[]
{
\begin{frame}
\frametitle{Outline}
\tableofcontents[currentsection]
\end{frame}}


\section{Recurence relations}

\begin{frame}{How do we come up with complicated formulas?}
\bl[Proposition]{
\[
0^2+1^2+\dots+n^2=\frac{(2n+1)(n+1)(n)}{6}
\]
}
or
\bl[Proposition]{The $n$th Fibonacci number is
\[
F_n=\frac{\left(\frac{1+\sqrt{5}}{2}\right)^{n+1}-\left(\frac{1-\sqrt{5}}{2}\right)^{n+1}}{\sqrt{5}}
\]
}
\end{frame}

\begin{frame}{Recurrence relations}
\bl[Definition]{A recurrence relation is a formula that specifies how each term of a sequence is produced from earlier terms.
}
For example 
\[
a_n=3a_{n-1}+4a_{n-2},\qquad a_0=3,\quad a_1=2.
\]
Then
\begin{align*}
a_2&=3a_1+4a_0=3\times 2+4\times 3=18,\\
a_3&=3a_2+4a_1=3\times 18+4\times 2=62,\\
a_4&=3a_3+4a_2=3\times 62+4\times 18=258.
\end{align*}
We are going to learn how to derive
\[
a_n=4^n+2(-1)^n
\]
\end{frame}

\begin{frame}
The order of a recurrence relation is the number of previous elements we need to compute the current one.
\begin{align*}
\textrm{First order:}&\qquad a_n=3a_{n-1}+5\\
\textrm{Second order:}&\qquad a_n=3a_{n-1}+2a_{n-2}+6\\
\vdots&\qquad\vdots\\
\textrm{$k$th order:}& \qquad a_n=2a_{n-1}+4a_{n-2}-8a_{n-3}+\dots+2a_{n-k}+9 
\end{align*}
\itemb
\item For a $k$th order recurrence relation, we need to know the first $k$ elements.
\item When a recurrence relation has the above structure, it is linear, otherwise it is called non-linear.
\iteme
\center Warm up by doing Problems 1-2 on the worksheet.
\end{frame}

\section{First order recurrence relations}
\begin{frame}{Simple cases}
\itemb
\item Simplest example:
\[
a_n=a_{n-1},\qquad\to\qquad a_n=a_0.
\]
\item Slightly more difficult:
\[
a_n=2a_{n-1},\qquad a_0=5.
\]
Then the sequence is $5,10,20,40,80,\dots$ and so 
\[
a_n=5\cdot 2^n.
\]

More generally if $a_n=sa_{n-1}$,
\[
a_n=a_0\cdot s^n.
\]
\iteme
\end{frame}

\begin{frame}
Consider
\[
a_{n}=2a_{n-1}+3
\]\vspace{-0.7cm}

and note\vspace{-0.3cm}
\begin{align*}
a_1&=2a_0+3\\
a_2&=2a_1+3=2^2a_0+2\cdot 3+3\\
a_3&=2a_2+3=2^3a_0+2^2\cdot 3+2\cdot 3+3\\
a_4&=2a_3+3=2^4a_0+(2^3+2^2+2+1)\cdot 3\\
\vdots&\qquad\vdots\\
a_n&=2a_{n-1}+3=2^na_0+(2^{n-1}+2^{n-2}+\dots+2+1)\cdot 3
\end{align*}
Using $1+2+\dots+2^{n-1}=2^n-1$, we can conjecture\vspace{-0.1cm}
\bl[]{\vspace{-0.1cm}
\[
a_n=2^n(a_0+3)-3.
\]}\vspace{-0.3cm}

\center\color{red}This still needs to be proved (e.g. by induction)!\color{black}
\end{frame}


\begin{frame}{In general}
Consider
\[
a_{n}=s\cdot a_{n-1}+t
\]\vspace{-0.7cm}

and note\vspace{-0.3cm}
\begin{align*}
a_1&=sa_0+t\\
a_2&=sa_1+t=s^2a_0+s\cdot t+t\\
a_3&=sa_2+t=s^3a_0+s^2\cdot t+s\cdot t+t\\
\vdots&\qquad\vdots\\
a_n&=sa_{n-1}+t=s^na_0+(s^{n-1}+s^{n-2}+\dots+s+1)\cdot t
\end{align*}
Recall
\[
1+2+\dots+s^{n-1}=\left\{\begin{array}{cc}
\frac{s^n-1}{s-1},&s\neq 1,\\
n,&s=1.
\end{array}\right.
\]
\end{frame}

\begin{frame}{In general}
Consider\vspace{-0.3cm}
\[
a_{n}=s\cdot a_{n-1}+t
\]\vspace{-0.6cm}

and note
\[
a_n=sa_{n-1}+t=s^na_0+(s^{n-1}+s^{n-2}+\dots+s+1)\cdot t
\]
Recall
\[
1+2+\dots+s^{n-1}=\left\{\begin{array}{cc}
\frac{s^n-1}{s-1},&s\neq 1,\\
n,&s=1.
\end{array}\right.
\]\vspace{-0.3cm}
\bl[Proposition]{
When $s\neq 1$,\vspace{-0.1cm}
\[
a_n=s^n\left(a_0+\frac{t}{s-1}\right)-\frac{t}{s-1}
\]
and $a_n=a_0+nt$ when $s=1$.}\vspace{-0.3cm}
\end{frame}

\begin{frame}{In general}
\bl[Proposition]{
When $s\neq 1$,\vspace{-0.1cm}
\[
a_n=s^n\left(a_0+\frac{t}{s-1}\right)-\frac{t}{s-1}
\]
and $a_n=a_0+nt$ when $s=1$.}\vspace{-0.3cm}
\begin{proof}Straightforward by induction, left to the reader.\end{proof}
\end{frame}

\begin{frame}
A more convenient form to remember when $s\neq 1$:
\bl[Proposition]{
When $s\neq 1$, there are constants $c_1,c_2$ such that\vspace{-0.1cm}
\[
a_n=c_1s^n+c_2.
\]}
For example, let us solve $a_n=5a_{n-1}+3$ where $a_0=1$.
\itemb
\item By the proposition, $a_n=c_15^n+c_2$.
\item Note that
\begin{align*}
a_0&=1=c_1+c_2\\
a_1&=8=5c_1+c_2.
\end{align*}
\item Solving these, we find $c_1=\frac{7}{4}$ and $c_2=-\frac{3}{4}$ and thus
\bl[]{
\[
a_n=\frac{7}{4}\cdot 5^n-\frac{3}{4}
\]}
\iteme
Practice this by doing Problem 3 on the WS.
\end{frame}

\section{Second order recurrence relations}

\begin{frame}{An example}
Consider
\[
a_n=5a_{n-1}-6a_{n-2}
\]\vspace{-0.7cm}

\itemb
\item Based on the previous case, our first guess is that the solution should be some kind of power. 
\item Should it be $5^n$ or $6^n$?
\item Let's stay safe and try $r^n$ with $r$ left to be figured out later:
\[
r^n=5r^{n-1}-6r^{n-2}.
\]
\item Dividing both sides by $r^{n-2}$ gives
\[
r^2=5r-6.
\]
\item This has roots $r=2,3$, and we get two solutions:
\iteme
\bl[]{
\[
a_n=2^n,\qquad a_n=3^n
\]
}
\end{frame}

\begin{frame}{In general}
\bl[Proposition]{If $r$ is a root of the equation $x^2=s_1x+s_2$, then $a_n=r^n$ solves
\[
a_n=s_1a_{n-1}+s_2a_{n-2}
\]
}
\begin{proof}
We directly verify
\[
s_1r^{n-1}+s_2r^{n-2}=r^{n-2}(s_1r+s_2)=r^{n-2}r^2=r^n
\]
which proves the claim.
\end{proof}
\center But note that $r^0=1$, so this only gives solutions with $a_0=1$!

\end{frame}

\begin{frame}{Ways to build other solutions}
\bl[Proposition]{If $a^{(1)}_n$ and $a^{(2)}_n$ are both solutions of the recurrence relation\vspace{-0.3cm}
\[
a_n=s_1a_{n-1}+s_2a_{n-2}
\]\vspace{-0.8cm}\\
then for any real numbers $c_1, c_2$, the linear combination\vspace{-0.2cm}
\[
c_1a^{(1)}_n+c_2a^{(2)}_n
\]\vspace{-0.8cm}\\
is also a solution.}
\begin{proof}
We directly verify
\begin{align*}
&s_1(c_1a^{(1)}_{n-1}+c_2a^{(2)}_{n-1})+s_2(c_1a^{(1)}_{n-2}+c_2a^{(2)}_{n-2})=\\
&=c_1(s_1a^{(1)}_{n-1}+s_2a^{(1)}_{n-2})+c_2(s_1a^{(2)}_{n-1}+s_2a^{(2)}_{n-2})=c_1a^{(1)}_{n}+c_2a^{(2)}_n
\end{align*}
which proves the Proposition.
\end{proof}
\end{frame}

\begin{frame}{When $r_1\neq r_2$}
When the equation $x^2=s_1x+s_2$ has two distinct roots, then $r_1^n$ and $r_2^n$ are both solutions and thus
\[
a_n=c_1r_1^n+c_2r_2^n
\]
is also a solution for any numbers $c_1$ and $c_2$. This we can adjust to any initial conditions:\vspace{-0.2cm}
\begin{align*}
a_0=c_1+c_2, \qquad a_1=c_1r_1+c_2r_2.
\end{align*}
We can always solve these equations to be
\[
c_1=\frac{a_1-a_0r_2}{r_1-r_2},\qquad c_2=\frac{-a_1+a_0r_1}{r_1-r_2}
\]\vspace{-0.4cm}\\
\bl[Theorem]{If the equation $x^2=s_1x+s_2$ has two distinct roots $r_1\neq r_2$, then all the solutions of the recurrence relation $a_n=s_1a_{n-1}+s_2a_{n-2}$ is of the form
\[
c_1r_1^n+c_2r_2^n.
\]
}
\end{frame}

\begin{frame}{Example}
Consider
\[
a_n=5a_{n-1}-6a_{n-2}
\]
We have found that $r=2,3$ and therefore
\[
a_n=c_12^n+c_23^n.
\]
Let's assume $a_0=-1$ and $a_1=1$, which means
\[
-1=c_1+c_2,\qquad 1=2c_1+3c_2.
\]
This has solution
\[
c_1=-4,\qquad c_2=3
\]
and therefore
\[
a_n=-4\cdot 2^n+3\cdot 3^n.
\]
\center Do this yourself on Problem 4(a).
\end{frame}

\begin{frame}{Complex roots}
Consider
\[
a_n=2a_{n-1}-2a_{n-2},\qquad a_0=1,a_1=3.
\]
The associated quadratic equation is $x^2=2x-2$ or
\[
0=x^2-2x+2
\]
The quadratic formula yields that this has two complex roots $1\pm i$ and
\[
a_n=c_1(1+i)^n+c_2(1-i)^n.
\]
The equations to solve are
\[
1=c_1+c_2,\qquad 3=c_1(1+i)+c_2(1-i).
\]
which has solutions $c_1=\frac{1}{2}-i$ and $c_2=\frac{1}{2}+i$. Therefore
\bl[]{
\[
a_n=\left(\frac{1}{2}-i\right)(1+i)^n+\left(\frac{1}{2}+i\right)(1-i)^n.
\]}
\end{frame}

\begin{frame}{Repeated roots}
What happens when $r_1=r_2$, e.g.
\[
a_n=4a_{n-1}-4a_{n-2},\qquad a_0=1,a_1=3.
\]
The equation $x^2=4x-4$ can be written
\[
0=x^2-4x+4=(x-2)(x-2)
\]
 has only one repeated root $r=2$, which only gives
\[
a_n=c2^n.
\]
But this is not enough to handle the initial condition:
\[
1=c,\qquad 3=c2
\]
which cannot be satisfied by a single $c$.
\end{frame}



\begin{frame}{Repeated roots}
\[
a_n=4a_{n-1}-4a_{n-2},\qquad a_0=1,a_1=3.
\]
The first five values are $1,3,8,20,48$. Let's see what $c$ would need to be used in each step if $a_n=c 2^n$.
\begin{align*}
n=0&\qquad 1=c&\to&\qquad c=1\\
n=1&\qquad 3=c2&\to& \qquad c=\frac{3}{2}\\
n=2&\qquad 8=c2^2&\to&\qquad c=2\\
n=3&\qquad 20=c2^3&\to&\qquad c=\frac{5}{2}\\
n=4&\qquad 48=c2^4&\to&\qquad c=3.
\end{align*}
We can read of the pattern $c(n)=(1+\frac{n}{2})$ and so
\[
a_n=\left(1+\frac{n}{2}\right) 2^n=2^n+\frac{1}{2}n2^n
\]
\end{frame}

\begin{frame}{Repeated roots}
\bl[Theorem]{Let the quadratic equation $x^2-s_1x-s_2=0$ have exactly one solution $r\neq 0$. Then every solution of the recurrence relation\vspace{-0.2cm}
\[
a_n=s_1a_{n-1}+s_2a_{n-2}
\]\vspace{-0.6cm}\\
is of the form\vspace{-0.2cm}
\[
c_1r^n+c_2nr^n.
\]}
\bl[Proof]{
Since the quadratic equation has only one root, factorization gives\vspace{-0.2cm}
\[
x^2-s_1x-s_2=(x-r)^2,
\]\vspace{-0.4cm}\\
and therefore $s_1=2r$ and $s_2=-r^2$ and the recurrence relation becomes\vspace{-0.2cm}
\[
a_n=2ra_{n-1}-r^2a_{n-2}
\]\vspace{-0.8cm}\\
~[...]}


\end{frame}

\begin{frame}
\bl[Theorem]{Then every solution of the recurrence relation\vspace{-0.2cm}
\[
a_n=s_1a_{n-1}+s_2a_{n-2}
\]\vspace{-0.6cm}\\
is of the form\vspace{-0.2cm}
\[
c_1r^n+c_2nr^n.
\]}
\bl[Proof]{
[...]
\[
a_n=2ra_{n-1}-r^2a_{n-2}
\]
Then the given form satisfies the recurrence relation.
\begin{align*}
&2r(c_1r^{n-1}+c_2(n-1)r^{n-1})-r^2(c_1r^{n-2}+c_2(n-2)r^{n-2})=\\
&=(2c_1r^n-c_1r^n)+(2c_2(n-1)r^n-c_2(n-2)r^n)=\\
&=c_1r^n+c_2nr^n.
\end{align*}
[...]}


\end{frame}

\begin{frame}
\bl[Theorem]{Then every solution of the recurrence relation\vspace{-0.2cm}
\[
a_n=s_1a_{n-1}+s_2a_{n-2}
\]\vspace{-0.6cm}\\
is of the form\vspace{-0.2cm}
\[
c_1r^n+c_2nr^n.
\]}
\bl[Proof]{
[...]
To show that every solution can be written in this form, note that matching the initial conditions gives
\[
a_0=c_1,\qquad a_1=r(c_1+c_2).
\]
Since by assumption $r\neq 0$, \vspace{-0.3cm}
\[
c_1=a_0\qquad c_2=\frac{a_0r-a_1}{r}.
\]\vspace{-0.6cm}\\
which means that for every initial condition the solution can be given in the desired form and the theorem is proved.\qed}

\end{frame}

\begin{frame}{Example}
Consider
\[
a_{n}=8a_{n-1}-16a_{n-2}, a_0=2,a_1=5
\]
The corresponding quadratic equation is
\[
0=x^2-8x+16=(x-4)^2
\]
and therefore by the Theorem,
\[
a_n=c_14^n+c_2n4^n.
\]
To match the initial conditions, write
\[
2=a_0=c_1,\qquad 5=a_1=4(c_1+c_2),
\]
which implies $c_1=2$ and $c_2=-\frac{3}{4}$ and therefore the solution is
\bl[]{
\[
a_n=2\cdot 4^n-\frac{3}{4}n4^n.
\]}
\center Do Problem 4 (b) on the worksheet!
\end{frame}

\begin{frame}{Non-homogeneous equation}
Consider 
\[
a_n=5a_{n-1}-6a_{n-2}+2, a_0=1, a_1=2.
\]
\itemb
\item We already know how to solve the homogeneous equation
\[
a_n^h=5a_{n-1}^h-6a_{n-2}^h
\]
to get
\[
a_n^h=c_12^n+c_23^n
\]
\item We also know that the sum of solutions is a solution.
\item Therefore we can try to find the inhomogeneous solution as
\[
a_n=c_12^n+c_23^n+a_n^i
\]
where $a_n^i$ is any solution of the inhomogeneous equation.
\item This is already adjustable to the inital conditions by finding the right $c_1$ and $c_2$.
\iteme
\end{frame}

\begin{frame}{Non-homogeneous equation}
Consider \vspace{-0.3cm}
\[
a_n=5a_{n-1}-6a_{n-2}+2, a_0=1, a_1=2.
\]\vspace{-0.5cm}
\itemb
\item Try to find the inhomogeneous solution as
\[
a_n=c_12^n+c_23^n+a_n^i
\]
\item This is already adjustable to the inital conditions by finding the right $c_1$ and $c_2$.
\item Easiest thing to try is $a_n^{i}=c_3$. Then
\[
c_3=5c_3-6c_3+2
\]
which works if $c_3=1$. Therefore 
\[
a_n=c_12^n+c_23^n+1,
\]
and after matching the initial conditions gives $c_1=-1, c_2=1$ and thus
\[
a_n=-2^n+3^n+1.
\]
\iteme
\end{frame}

\begin{frame}
\center{Do Problem 5 on the worksheet!}
\end{frame}
\end{document}