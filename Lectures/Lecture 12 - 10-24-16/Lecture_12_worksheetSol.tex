\documentclass[11pt]{preprint}

\setlength{\topmargin}{0mm} \setlength{\oddsidemargin}{0mm}
\setlength{\textwidth}{160mm} \setlength{\textheight}{215mm}

\usepackage{amssymb,amsmath,amscd,amsthm}
\usepackage{graphics}
\usepackage{tikz}

\def\enumb{\begin{enumerate}}
\def\enume{\end{enumerate}}
\def\itemb{\begin{itemize}}
\def\iteme{\end{itemize}}
\def\integers{\mathbb{Z}}

\def\multiset#1#2{\ensuremath{\left(\kern-.3em\left(\genfrac{}{}{0pt}{}{#1}{#2}\right)\kern-.3em\right)}}



\newtheorem{proposition}{Proposition}
\newtheorem{theorem}{Theorem}
\newtheorem*{solution}{Solution}

\title{Discrete Mathematics, 2016 Fall - Worksheet 12}
\author{Instructor: Zsolt Pajor-Gyulai, CIMS}
\date{October 24, 2016}



\begin{document}

\maketitle

In all of the above problems explain your answer in full English sentences.
\enumb

\item Calculate the first six terms of the sequence (that is, $a_0$ through $a_5$) 

\[
a_n=2a_{n-1}+2,\qquad a_0=1
\]

\begin{solution}
Simple computaton reveals $a_1= 4$, $a_2=10$, $a_3=22$, $a_4=46$, $a_5=94$.
\end{solution}

\item Let $e_0=1$, $e_1=4$ and, for $n>1$, let $e_n=4(e_{n-1}-e_{n-2})$. What are the first five terms of the sequence $e_0,e_1,e_2,\dots$? Prove $e_n=(n+1)2^n$ for any natural $n$.

\begin{solution}
$e_0=1$, $e_1=4$, $e_2=12$, $e_3=32 $, $e_4=80$. We show the formula by strong induction. It clearly holds for $n=0,1$. Assume that it holds for any $n<k$ with $k\geq 2$. Then we have, in particular, that
\[
e_{k-1}=k2^{k-1},\qquad e_{k-2}=(k-1)2^{k-2}.
\]
Then by the recursion,
\[
e_k = 4(e_{k-1}-e_{k-2})=4(k2^{k-1}-(k-1)2^{k-2})=k2^{k+1}-(k-1)2^k=2^k(2k-k+1)=(k+1)2^k
\]
\end{solution}

%\item Let $F_n$ denote the $n$th Fibonacci number. Prove:
%\[
%F_n=\frac{\left(\frac{1+\sqrt{5}}{2}\right)^{n+1}-\left(\frac{1-\sqrt{5}}{2}\right)^{n+1}}{\sqrt{5}}
%\]

\item Solve the following first order recurrence relations.
\enumb
\item $a_n=\frac{2}{3}a_{n-1}, a_0=4$.
\begin{solution}
We know from class that the solution has the form $a_n=c_1\left(\frac{2}{3}\right)^n+c_2$. Matching with the initial condition we get 
\[
4= a_0 =c_1+c_2=4,\qquad \frac{8}{3}=a_1=\frac{2}{3}c_1+c_2.
\]
This can be solved to get $c_1=4$, $c_2=0$ and the solution to the recurrence relation is $a_n=4\left(\frac{2}{3}\right)^n$. 
\end{solution}

\item $a_n=2a_{n-1}+2,a_0=2$.

\begin{solution}
Again, by what we learned in class, we look for the solution in the form $c_12^n+c_2$. Matching with the initial condition gives
\[
2=a_0=c_1+c_2,\qquad 6=a_1=2c_1+c_2
\]
which can be solved to get $c_1=4$ and $c_2=-2$ and thus the solution is
\[
a_n=4\cdot 2^n-2
\]
\end{solution}


%\item $a_n=a_{n-1}+7, a_0=-3$.
%\item $a_n=4-2a_{n-1}, a_0=0$.
\enume

\item Solve the following second order recurrence relations.
\enumb
\item $a_n=3a_{n-1}+4a_{n-2}, a_0=3, a_1=2$.
\begin{solution}
The characteristic equation is $r^2=3r+4$ which has roots $r_1=4$ and $r_2=-1$, therefore we can look for the solutions in the form $a_n=c_14^n+c_2(-1)^n$. Matching the initial condition gives
\[
3=a_0=c_2+c_2,\qquad 2=a_1=c_14-c_2
\]
which can be solved to be $c_1=5$ and $c_2=-2$, and the solution therefore is
\[
a_n=5\cdot 4^n-2(-1)^n
\]
\end{solution}
\item $a_n=-6a_{n-1}-9a_{n-2}, a_0=3, a_1=6$.
\begin{solution}
The characteristic equation is $r^2=-6r-9$ which has a double root $r=-3$, therefore we can look for the solutions in the form $a_n=c_1(-3)^n+c_2n(-3)^n$. Matching with inital conditions gives
\[
3=a_0=c_1,\qquad 6=a_1=c_1(-3)+c_2(-3)
\]
which has solution $c_1=3$ and $c_2=-5$ and therefore the solution is
\[
a_n=(3-5n)\cdot(-3)^n.
\]
\end{solution}
\enume
\item What can go wrong with the technique we used to solve the non-homogeneous equation on the slides? Try to solve
\itemb
\item $a_n=4a_{n-1}+5a_{n-2}+4, a_0=2, a_1=3$.

\begin{solution}
The characteristic equation of the homogeneous solution is $r^2=4r+5$ which has roots $r=5,-1$ and thus
\[
a^h_n=c_15^n+c_2(-1)^n.
\]
Now we can try a constant for the inhomogeneous solution:
\[
c_3=4c_3+5c_3+4
\]
which gives us $c_3=-1/2$ and we have 
\[
a_n=c_15^n+c_2(-1)^n-\frac{1}{2}
\]
Matching with the initial condition gives
\[
2=a_0=c_1+c_2-\frac{1}{2},\qquad 3=a_1=5c_1-c_2-1/2
\]
which has solution $c_1=1$ and $c_2=3/2$ and finally the solution is
\[
a_n=5^n+\frac{3}{2}(-1)^n-\frac{1}{2}.
\]
In this case everything was alright.
\end{solution}
\item $a_n=3a_{n-1}-2a_{n-2}+5, a_0=a_1=3$.

\begin{solution}
The characteristic equation of the homogeneous part is $r^2-3r+2=0$ with roots $r=1,2$, which means
\[
a_n^h=c_12^n+c_2.
\]
If we try to find a constant solution to the inhomogeneous equation we get
\[
c_3=3c_2-2c_3+5\qquad\rightarrow 0=5
\]
which is clearly impossible. No constant solves the inhomogeneous equation. Instead, we will try the next simplest $a_n=c_3n$. Then plugging this back into the recurrence relation, we get
\[
c_3n=3c_3(n-1)-2c_3(n-2)+5\qquad\rightarrow c_3=-5
\]
and we can write the solution of the inhomogeneous equation in the form
\[
a_n=c_12^n+c_2-5n
\]
Matching with the initial condition is left to the reader.
\end{solution}

\item $a_n=2a_{n-1}-a_{n-2}+2,a_0=4,a_1=2$.

This is very similar to the previous example, except that now even $c_3 n$ will give no solution to the in homogeneous equation. Try $c_3n^2$.
\iteme
\enume
\end{document}