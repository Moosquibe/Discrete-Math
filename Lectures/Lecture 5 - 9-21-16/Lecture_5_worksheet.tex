\documentclass[11pt]{preprint}

\setlength{\topmargin}{0mm} \setlength{\oddsidemargin}{0mm}
\setlength{\textwidth}{160mm} \setlength{\textheight}{215mm}

\usepackage{amssymb,amsmath,amscd,amsthm}

\def\enumb{\begin{enumerate}}
\def\enume{\end{enumerate}}
\def\itemb{\begin{itemize}}
\def\iteme{\end{itemize}}
\def\integers{\mathbb{Z}}



\newtheorem{proposition}{Proposition}
\newtheorem{theorem}{Theorem}

\title{Discrete Mathematics, 2016 Spring - Worksheet 5}
\author{Instructor: Zsolt Pajor-Gyulai, CIMS}
\date{September 21, 2016}



\begin{document}

\maketitle

In all of the above problems explain your answer in full English sentences.

\enumb
\item Rewrite the following sentences using the quantifier notation.
\enumb
\item Every integer is a prime.
\item There is an integer whose square is $2$.
\item All integers are divisible by $5$.
\item Some integer is divisible by $7$.
\item For every integer $x$, there is an integer $y$ such that $xy=1$.
\item There are an integer $x$ and an integer $y$ such that $x/y=10$.
\item There is an integer that, when multiplied by any integer, always gives the result $0$.
\item No matter what integer you choose, there is always another integer that is larger.
\enume

\item Write the negation of each of the sentences in the previous problem, first with quantifiers and then in plain English. In the first way, move the $\neg$ symbol as far to the right as possible.

\item Label each of the following sentences about integers as either true or false. (No need to prove them)
\enumb
\item $\forall x,\forall y, xy=0$
\item $\forall x, \exists y, xy=0$
\item $\exists x,\forall y, xy=0$
\item $\exists x,\exists y, xy=0$.
\enume



\item Let $A=\{1,2,3,4,5\}$ and let $B=\{4,5,6,7\}$. Compute
\enumb
\item $A\cup B$
\item $A\cap B$
\item $A-B$
\item $B-A$
\item $A\Delta B$
\item $A\times B$
\item $B\times A$
\enume

\item Prove the following theorem and illustrate it with a Venn-diagram (you can look at p57 for what this means).
%\begin{enumerate}
%\item
%\begin{theorem}(Distributive properties)
%Let $A$, $B$, and $C$ denote sets. Then we have
%\[
%A\cup(B\cap C)=(A\cup B)\cap (A\cup C),\qquad A\cap(B\cup C)=(A\cap B)\cup (A\cap C).
%\]
%\end{theorem}
(First DeMorgan's Law)
\begin{theorem}
Let $A$, $B$, and $C$ sets. Then
\[
A-(B\cup C)=(A-B)\cap(A-C)
\]
\end{theorem}


%\item Prove or disprove the following statements.
%\enumb
%\item $A\cup B=A\cap B$ if and only if $A=B$.
%\item $A-\emptyset=A$ and $\emptyset-A=\emptyset$.
%\item $A\Delta A=\emptyset$ and $A\Delta \emptyset= A$.
%\item $A\subseteq B$ if and only if $A-B=\emptyset$.
%\item $A-(B-C)=(A-B)-C$.
%\item $(A\cup B)-C=(A-C)\cap(B-C)$.
%\item If $B=A\cup C$ then $A=B-C$.
%\item $|A-B|=|A|-|B|$.
%\enume



\item Let $A$ and $B$ be sets with $|A|=10$ and $|B|=7$.
\enumb
\item Calculate $|A\cap B|+|A\cup B|$.
\item Find an upper bound $y$ and a lower bound $x$ for $|A\cup B|$, that are sharp. That is 
\[
x\leq|A\cup B|\leq y.
\]
To show that your answer is sharp, find sets such that $|A\cup B|=x$ and $|A\cup B|=y$ exactly.
\enume
\item Prove the following proposition:
\begin{proposition}
Let $n$ be an integer. Then
\[
2^0+2^1+...+2^{n-1}=2^n-1
\]
\end{proposition}
Hints: 
\itemb 
\item How many subsets does the set $N=\{1,2,\dots, n\}$ have?
\item How many subsets are there whose largest element is $j$? (Write this out for $j=1,2,3,4$ to see the pattern.)
\item How do the two answers to the previous questions relate?
\iteme 
%If you get stuck, you can look into the proof of Proposition 13.1 on p66.

\enume 
\end{document}