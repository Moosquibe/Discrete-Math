\documentclass{beamer}

\mode<presentation>
{
  \usetheme{Frankfurt}
  \usecolortheme{orchid}
  \setbeamercovered{invisible}
  \setbeamertemplate{footline}[frame number]
}

\usepackage[english]{babel}
\usepackage[latin1]{inputenc}
\usepackage{times}
\usepackage[T1]{fontenc}
\usepackage{tikz}
\usepackage{array}
\usepackage{cancel}


\usetikzlibrary{shapes,backgrounds}

\def\multiset#1#2{\ensuremath{\left(\kern-.3em\left(\genfrac{}{}{0pt}{}{#1}{#2}\right)\kern-.3em\right)}}

\def\blue{\color{blue}~}
\def\black{\color{black}~}
\def\bl[#1]#2{\begin{block}{#1}#2\end{block}}
\def\integers{\mathbb{Z}}
\def\enumb{\begin{enumerate}}
\def\enume{\end{enumerate}}
\def\itemb{\begin{itemize}}
\def\iteme{\end{itemize}}
\def\div{~\textrm{div}~}
\def\mod{~\textrm{mod}~}


\usepackage{remreset}
\makeatletter
\@removefromreset{subsection}{section}
\makeatother
\setcounter{subsection}{1}

\title{Discrete Mathematics, Section 001, Fall 2016}
\subtitle{Lecture 19: Modular arithmetic and congruences}
\date{November 28, 2016}

\author[Zsolt]{Zsolt Pajor-Gyulai \\ \texttt{zsolt@cims.nyu.edu}}

\pgfdeclareimage[height=1cm]{NYUlogo}{NYUlogo.jpg}

\institute[NYU] 
{
\normalsize Courant Institute of Mathematical Sciences
}
\titlegraphic{\pgfuseimage{NYUlogo}}

\begin{document}

\begin{frame}
  \titlepage
\end{frame}

\AtBeginSection[]
{
\begin{frame}
\frametitle{Outline}
\tableofcontents[currentsection]
\end{frame}}

\section{Modular arithmetic}

\begin{frame}{New context for basic operations}
\itemb
\item Arithmetic is the study of basic operations: $+,-,\cdot,/$.
\item The usual context for studying these are number systems like $\mathbb{Z},\mathbb{Q},\mathbb{R}$.
\iteme
For example, look at divison $/$:
\itemb
\item In the context of rational numbers $\mathbb{Q}$, $x/y$ makes sense whenever $y\neq 0$.
\item In the context of integers, $\mathbb{Z}$, $x/y$ makes sense if and only if $y|x$.
\iteme
\begin{center}
Division takes on slightly different meaning depending on the context!
\end{center}

\textbf{In this lecture:} We introduce a new context for these operations, by performing arithmetic in the set
\[
\mathbb{Z}_n:=\{0,1,2,\dots,n-1\}
\]
\end{frame}

\begin{frame}{Modular addition and multiplication}
\bl[Definition]{Let $n$ be a positive integer and $a,b\in\mathbb{Z}_n$  We define
\begin{align*}
a\oplus b&:=(a+b)\textrm{ mod }n\\
a\otimes b&:= (ab)\textrm{ mod }n
\end{align*}
$\oplus$ is called \textbf{addition modulo $n$}, while $\otimes$ is called \textbf{multiplication modulo $n$}.}


For example in $\mathbb{Z}_{10}$,
\[
\begin{array}{cc}
5\oplus 5=0&9\oplus 8=7\\
5\otimes 5=5&9\otimes 8=2
\end{array}
\]

\center Do Problem 1-2.
\end{frame}

\begin{frame}
\bl[Closure property]{Let $a,b\in\mathbb{Z}_{n}$. Then $a\oplus b\in Z_n$ and $a\otimes b\in \mathbb{Z}_n$.}
\begin{proof}
Straightforward by the definition of $\textrm{mod}$.
\end{proof}
What about other properties? (Homework)
\bl[Proposition]{Let $n$ be an integer with $n\geq 2$.
\enumb
\item For all $a,b\in\mathbb{Z}_n$, $a\oplus b=b\oplus a$ and $a\otimes b=b\otimes a$
\item For $a,b,c\in\mathbb{Z}_n$, $a\oplus (b\oplus c)=(a\oplus b)\oplus c$
\item For $a\in\mathbb{Z}_n$, $a\oplus 0=a$, $a\otimes 1=a$ and $a\otimes 0=0$.
\item For $a,b,c\in\mathbb{Z}_n$, $a\otimes (b\oplus c)= (a\otimes b)\oplus (a\otimes c)$
\enume}
In other words $\oplus$ and $\otimes$ are commutative, associative operations with identities $0$ and $1$ respectively. The last bulletpoint is called the distributive property. 
\end{frame}

\begin{frame}{Modular substraction}
Ordinary substraction in terms of addition:
\bl[Definition]{Let $a,b\in\mathbb{Z}$. We define $a-b$ to be the solution of the equation $a=b+x$.}
Then we would prove
\enumb
\item The equation $a=b+x$ has a solution.
\item The equation $a=b+x$ has only one solution.
\enume

\center \color{red}We want to use the same approach defining modular substraction.\color{red}
\end{frame}

\begin{frame}{Modular substraction}
\bl[Proposition]{Let $n$ be a positive integer, and let $a,b\in\mathbb{Z}_n$. Then there is one and only one $x\in\mathbb{Z}_n$ such that $a=b\oplus x$.}


Let $x=(a-b)\textrm{ mod }n$.
\itemb
\item By the definition of $\textrm{mod}$, $x\in\mathbb{Z}_n$.
\item Note that $x=(a-b)+kn$ for some $k\in\mathbb{Z}$. We calculate
\begin{align*}
b\oplus x&=(b+x)\textrm{ mod } n=\\
&=[b+(a-b+kn)]\textrm{ mod }n=(a+kn)\textrm{ mod }n=a
\end{align*}
\iteme
So the existence part is proved.

[...]
\end{frame}

\begin{frame}{Modular substraction}
\bl[Proposition]{Let $n$ be a positive integer, and let $a,b\in\mathbb{Z}_n$. Then there is one and only one $x\in\mathbb{Z}_n$ such that $a=b\oplus x$.}
[...]

FTSC, suppose there were two $x,y\in\mathbb{Z}_n$ with $x\neq y$ for which\vspace{-0.2cm}
\[
a=b\oplus x,\qquad a=b\oplus y.
\]\vspace{-0.6cm}

This means that there are $k,j\in\mathbb{Z}$, such that\vspace{-0.2cm}
\[
a=b\oplus x = (b+x)\textrm{ mod } n=b+x+kn
\]\vspace{-0.6cm}
\[
a=b\oplus y = (b+y)\textrm{ mod } n=b+y+jn
\]\vspace{-0.6cm}

Combining these\vspace{-0.2cm}
\[
b+x+kn=b+y+jn
\]
[...]
\end{frame}

\begin{frame}
\bl[Proposition]{Let $n$ be a positive integer, and let $a,b\in\mathbb{Z}_n$. Then there is one and only one $x\in\mathbb{Z}_n$ such that $a=b\oplus x$.}
[...]
Combining these\vspace{-0.2cm}
\[
b+x+kn=b+y+jn
\]
which in turn implies
\[
x=y+(k-j)n\qquad\Rightarrow\qquad  x\equiv y \textrm{ (mod $n$)}.
\]
But since $0\leq x,y<n$, this implies $x=y$.$\Rightarrow\Leftarrow$\qed

\bl[Definition]{Let $n$ be a positive integer and let $a,b\in\mathbb{Z}_n$. Then $a\ominus b$ is the unique $x\in\mathbb{Z}_n$ such that $a=b\oplus x$.}
Note that the above proof also shows $a\ominus b=(a-b)\textrm{ mod }n$.
\end{frame}

\begin{frame}
\center Do Problem 3 on the worsheet. This shows that $\ominus$ is not commutative.
\end{frame}

\begin{frame}{Modular division}
This operation is very different from its usual counterpart:
\itemb
\item In ordinary division the only division not allowed is by zero.
\item In the context of $\mathbb{Z}_{10}$,
\[
5\otimes 2=5\otimes 4=0\qquad\textrm{but}\qquad 2\neq 4
\]
and so we can't just $\oslash$ by 5.
\iteme
Given $a,b\in\mathbb{Z}_{10}$ with $b\neq 0$, must there be a solution to $a=b\otimes x$?
\itemb
\item Let $a=6$, $b=2$. Then $x=3$ and $x=8$ are both solutions.
\item Let $a=7$, $b=2$. There are no solutions (check all options).
\item Let $a=7$, $b=3$. There is only one solution $x=9$.
\iteme
\center\color{red} We need to do something more elaborate!\color{black}
\end{frame}

\begin{frame}{Modular division}
In the context of $\mathbb{Q}$, we define divison by multiplication by the reciprocal:
\[
\frac{a}{b}=a\cdot b^{-1}
\]
Then we cannot divide by $0$ as it doesn't have a reciprocal.

\bl[Modular reciprocal]{Let $n$ be a positive integer and let $a\in\mathbb{Z}_n$. A \textbf{reciprocal} of $a$ is an element $b\in\mathbb{Z}_n$ such that $a\otimes b=1$. An element of $\mathbb{Z}_n$ that has a reciprocal is called \textbf{invertible}.}

Natural questions:
\itemb
\item What elements have reciprocals?
\item Can an element have more than one reciprocals?
\iteme

\end{frame}

\begin{frame}{Modular division}
\begin{columns}
\column{0.65\textwidth}
\begin{tabular}{|c|cccccccccc|}
\hline
$\otimes$&0&1&2&3&4&5&6&7&8&9\\
\hline
0&0&0&0&0&0&0&0&0&0&0\\
1&0&1&2&3&4&5&6&7&8&9\\
2&0&2&4&6&8&0&2&4&6&8\\
3&0&3&6&9&2&5&8&1&4&7\\
4&0&4&8&2&6&0&4&8&2&6\\
5&0&5&0&5&0&5&0&5&0&5\\
6&0&6&2&8&4&0&6&2&8&4\\
7&0&7&4&1&8&5&2&9&6&3\\
8&0&8&6&4&2&0&8&6&4&2\\
9&0&9&8&7&6&5&4&3&2&1\\
\hline
\end{tabular}
\vspace{0.3cm}

\color{red}Now we make use of these observations!\color{black}
\column{0.35\textwidth}
$0$ has no reciprocal.\vspace{0.3cm}

$2,4,6,8$ have no reciprocals either.\vspace{0.3cm}

$1,3,7,9$ are invertible.\vspace{0.3cm}

The invertible ones have ony one reciprocal.\vspace{0.3cm}

The invertible ones are precisely the ones relatively primes to $10$.\vspace{0.3cm}

$3^{-1}=7$ and $7^{-1}=3$, 

$1^{-1}=9$ and $9^{-1}=1$.
\end{columns}

\end{frame}

\begin{frame}{Modular division}
\bl[Proposition]{Let $n$ be a positive integer and $a\in\mathbb{Z}_n$. If $a$ has a reciprocal in $\mathbb{Z}_n$, then it has only one reciprocal.}

\begin{proof}
FTSC, suppose $a$ had two reciprocals, $b,c\in\mathbb{Z}_n$ with $b\neq c$. Then
\[
b\otimes (a\otimes c)=b\otimes 1=b,
\]
\[
(b\otimes a)\otimes c=1\otimes c=c.
\]
By the associative property, these two are equal and so $b=c$. $\Rightarrow\Leftarrow$.
\end{proof}
Therefore it make sense to talk about \textit{the} reciprocal of an element $a$.
\end{frame}


\begin{frame}{Modular division}
\bl[Propostition]{Let $n$ be a positive integer and $a\in\mathbb{Z}_n$. Suppose $a$ is invertible. $b=a^{-1}$, then $b$ is invertible and $a=b^{-1}$. In other words $(a^{-1})^{-1}=a$.}

\begin{proof}
Since $a^{-1}$ is the reciprocal of $a$,
\[
a^{-1}\otimes a=1.
\]
But this also shows that the reciprocal of $a^{-1}$ is $a$.
\end{proof}
\end{frame}

\begin{frame}{Modular division}
\bl[Modular division]{Let $n$ be a positive integer and let $b$ be an invertible element of $\mathbb{Z}_n$. Let $a\in\mathbb{Z}_n$ be arbitrary. Then $a\oslash b$ is defined to be $a\otimes b^{-1}$.}

For example, in the context of $\mathbb{Z}_{10}$, 
\[
7^{-1}=3\qquad\Rightarrow\qquad 2\oslash 7=2\otimes 3=6
\]
Practice some by doing Problem 4 on the Worksheet.\vspace{0.3cm}

We still have not answered:
\itemb
\item In $\mathbb{Z}_n$, which elements are invertible?
\item In $\mathbb{Z}_n$, given that $a$ is invertible, how do we calculate $a^{-1}$?
\iteme
We could just write out the multiplication table, like we did for $\mathbb{Z}_{10}$. However, good luck with $\mathbb{Z}_{1000}$.
\end{frame}

\begin{frame}{Invertible elements of $\mathbb{Z}_n$}
\bl[Theorem]{Let $n$ be a positive integer and let $a\in\mathbb{Z}_n$. Then $a$ is invertible if and only if $a$ and $n$ are relative primes.}
Recall:
\bl[]{$gcd(a,n)=1$ if and only if there are $b,k\in\mathbb{Z}_n$ such that $ab+kn=1$.}
($\Rightarrow$): Suppose $a$ is invertible. This means there is a $b\in\mathbb{Z}_n$ such that 
\[
1=a\otimes b = (ab)\textrm{ mod } n.
\]
This means that there is $k\in\mathbb{Z}$ such that
\[
ab+kn=1
\]
and therefore $a$ and $n$ are relative primes.

[...]
\end{frame}

\begin{frame}{Invertible elements of $\mathbb{Z}_n$}
\bl[Theorem]{Let $n$ be a positive integer and let $a\in\mathbb{Z}_n$. Then $a$ is invertible if and only if $a$ and $n$ are relative primes.}
[...]

($\Leftarrow$) Suppose $a$ and $n$ are relative primes. Then there are integers $x,y$ such that $ax+ny=1$. Let 
\[
b=x\textrm{ mod } n\qquad\Rightarrow\qquad b=x+kn
\]
for some $k\in\mathbb{Z}$. Therefore
\[
1=ax+ny=a(b-kn)+ny=ab+(y-ka)n,
\]
and $a\otimes b=(ab)\textrm{ mod }n=1$ and $b=a^{-1}$.\qed.
\end{frame}

\begin{frame}{Invertible elements of $\mathbb{Z}_n$}
This also helps us find $a^{-1}$ for invertible $a$-s in $\mathbb{Z}_n$.
\enumb
\item Use Euclid's algorithm to find $x,y$ such that
\[
ax+ny=1.
\]
\item Then $a^{-1}= x\textrm{ mod }n$.
\enume
For example, in $\mathbb{Z}_{431}$, let's find $29^{-1}$.

\center This is problem $5$ on the worksheet.

\end{frame}

\begin{frame}{Solving equations in $\mathbb{Z}_n$}
\itemb
\item Consider $2\otimes x=4$ in $\mathbb{Z}_{11}$. Then clearly $gcd(2,11)=1$ and $2^{-1}=6$ and so
\[
x=(2^{-1}\otimes 2)\otimes x= 2^{-1}\otimes (2\otimes x)=2^{-1}\otimes 4=6\otimes 4=2
\]
\item Consider, however, $2\otimes x=4$ in $\mathbb{Z}_{10}$. Now $2$ doesn't have a reciprocal and the only thing we can do at this point is guess. Checking all the values, we see that $x=2$ and $x=7$ are the two solutions.
\iteme

\begin{center}
Do Problem 6 on the Worksheet.
\end{center}
\end{frame}

\section{Congruences}

\begin{frame}{An application: Solving congruences}
For example, try to find all integers $x$ such that
\[
3x\equiv 4\textrm{ (mod $11$)}.
\]
Note that if $x_0$ is a solution, then
\[
3(x_0+k\cdot 11)=3x_0+33k\equiv 3x_0\equiv 4\textrm{ (mod $11$)}
\]
for any $k\in\mathbb{Z}$ and therefore $x+k\cdot 11$ is also a solution.\vspace{0.5cm}


\begin{centering}
\color{red} This means that it is enough to find the solutions in $\mathbb{Z}_{11}$, then all other solutions can be obtained by adding (or substracting) some multiple of $11$.\color{black}
\end{centering}
\end{frame}

\begin{frame}{An application: Solving congruences}
We need to find $x\in\mathbb{Z}_{11}$ for which $3x\equiv 4$ (mod 11). Note
\[
3x\equiv 4\textrm{ mod $11$}\qquad\Leftrightarrow\qquad (3x)\textrm{ mod }11=4\qquad\Leftrightarrow\qquad 3\otimes x=4.
\]

But we know how to solve this as $3^{-1}=4$ in $\mathbb{Z}_{11}$ and therefore
\[
x=(3^{-1}\otimes 3)\otimes x=3^{-1}\otimes(3\otimes x)=3^{-1}\otimes 4=4\otimes 4=5
\]
Therefore the solutions to the original congruence is
\[
\{5+k\cdot 11: k\in\mathbb{Z}\}.
\]
\end{frame}

\begin{frame}{An application: Solving congruences}
\bl[Proposition]{Let $a,b,n\in\mathbb{Z}$ with $n>0$. Suppose $a$ and $n$ are relatively prime and consider the equation
\[
ax\equiv b\textrm{ (mod $n$)}.
\]
The set of solutions to this equation is
\[
\{x_0+kn:k\in\mathbb{Z}\}
\]
where $x_0=a_0^{-1}\otimes b_0$ with $a_0=a\textrm{ mod }n$ and $b_0=b\textrm{ mod }n$ and $\otimes$ is meant in the context of $\mathbb{Z}_n$.}

Practice this by Problem 7 on the worksheet.
\end{frame}

\begin{frame}{For quiz}
\itemb
\item Understand the operations $\oplus,\ominus,\otimes$ within the context of $\mathbb{Z}_n$.
\item Understand the notion of the reciprocal within the context of $\mathbb{Z}_n$ and how to compute this.
\item Understand how to use the reciprocals to solve equations in $\mathbb{Z}_n$.
\iteme
\end{frame}
\end{document}