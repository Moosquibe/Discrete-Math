\documentclass[11pt]{preprint}

\setlength{\topmargin}{0mm} \setlength{\oddsidemargin}{0mm}
\setlength{\textwidth}{160mm} \setlength{\textheight}{215mm}

\usepackage{amssymb,amsmath,amscd,amsthm}
\usepackage{graphics}
\usepackage{tikz}

\def\enumb{\begin{enumerate}}
\def\enume{\end{enumerate}}
\def\itemb{\begin{itemize}}
\def\iteme{\end{itemize}}
\def\integers{\mathbb{Z}}
\def\mod{{\rm ~mod~}}
\def\div{{\rm div}}


\def\multiset#1#2{\ensuremath{\left(\kern-.3em\left(\genfrac{}{}{0pt}{}{#1}{#2}\right)\kern-.3em\right)}}



\newtheorem{proposition}{Proposition}
\newtheorem{theorem}{Theorem}

\title{Discrete Mathematics, 2016 Fall - Worksheet 19}
\author{Instructor: Zsolt Pajor-Gyulai, CIMS}
\date{November 28, 2016}



\begin{document}

\maketitle

In all of the above problems explain your answer in full English sentences.

\enumb
\item In the context of $\mathbb{Z}_{10}$, calculate
\enumb
\item $3\oplus 3 = (3+3)\mod 10 = 6$

\item $7\otimes 3= (7\cdot 3)\mod 10 = 21\mod 10= 1$
\enume
\item In the context of $\mathbb{Z}_ {12}$, calculate
\enumb
\item $9\oplus 8= 17\mod 12 = 5$
\item $11\otimes 5 = 55 \mod 12 = 7$
\enume
\item In the context of $\mathbb{Z}_9$, calculate
\enumb
\item $5\ominus 8=(-3)\mod 9 = 6$
\item $8\ominus 5=3\mod 9 = 3$
\enume

\item In the context of $\mathbb{Z}_{10}$, calculate
\enumb
\item $8\oslash 7 = 8\otimes 7^{-1}=8\otimes 3 = 4$
\item $5\oslash 9 = 5\otimes 9^{-1}=5\otimes 1 = 5$
\enume

\item In $\mathbb{Z}_{431}$, find $29^{-1}$.
Use the Euclidean algorithm to find the solution of the Diophantine equation
\[
29x+431y=1
\]
\begin{align*}
431& =14 \cdot 29+25\\
29& = 1\cdot 25+ 4\\
25& = 6\cdot 4 + 1
\end{align*}
Now working backwards,
\[
1=25-6\cdot 4=25-6(29-25)=7\cdot 25-6\cdot 29 =7\cdot(431-14\cdot 29)-6\cdot 29=7\cdot 431-104\cdot 29
\]
and therefore $x=-104$ and $y=7$ is a solution and therefore $29^{-1}=(-104)\mod 431 = 327$.
\item Solve 
\enumb
\item $4\otimes (x\ominus 8)=9$ in $\mathbb{Z}_{11}$.
Multiplying by $4^{-1}=3$ from the left we get
\[
x\ominus 8 = 3\otimes 9 =27\mod 11 = 5
\]
$\oplus 8$ both sides gives
\[
x = 5\oplus 8 = 13 \mod 11 = 2.
\]
\item $2\otimes x=3$ in $\mathbb{Z}_{10}$.
$2$ does not have a reciprocal in $\mathbb{Z}_10$ because $gcd(2,10)=2\neq 1$. Trying all the numbers $0,1,2,3,4,5,6,7,8,9$, we can see that none of them gives a solution and therefore the equation has no solutions.
\enume

\item Find all solutions of
\enumb
\item $3x\equiv 17$ (mod $20$)

We first solve $3\otimes x_0 = 17$ in $\mathbb{Z}_{20}$. Since $3^{-1}=7$, we have 
\[
x_0=3^{-1}\otimes 17 =7\otimes 17=119\mod 20=19.
\]
Therefore we can get all the solutions of the congruence in as
\[
x=19+20k,\qquad k\in\mathbb{Z}.
\]
\item $2x\equiv 12$ (mod $15$)
Since $2^{-1}=8$ in $\mathbb{Z}_{15}$, we have $x_0=8\cdot 12\mod 15=6$ and the set of solutions of the congruence is given by
\[
x=6+15k,\qquad k\in\mathbb{Z}.
\]

\enume
\enume
\end{document}