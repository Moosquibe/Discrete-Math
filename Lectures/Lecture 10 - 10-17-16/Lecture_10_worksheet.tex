\documentclass[11pt]{preprint}

\setlength{\topmargin}{0mm} \setlength{\oddsidemargin}{0mm}
\setlength{\textwidth}{160mm} \setlength{\textheight}{215mm}

\usepackage{amssymb,amsmath,amscd,amsthm}

\def\enumb{\begin{enumerate}}
\def\enume{\end{enumerate}}
\def\itemb{\begin{itemize}}
\def\iteme{\end{itemize}}
\def\integers{\mathbb{Z}}

\def\multiset#1#2{\ensuremath{\left(\kern-.3em\left(\genfrac{}{}{0pt}{}{#1}{#2}\right)\kern-.3em\right)}}



\newtheorem{proposition}{Proposition}
\newtheorem{theorem}{Theorem}

\title{Discrete Mathematics, 2016 Fall - Worksheet 10}
\date{October 17, 2016}
\author{Instructor: Zsolt Pajor-Gyulai, CIMS}



\begin{document}

\maketitle

In all of the above problems explain your answer in full English sentences.

\enumb
\item Please state the contrapositive of each of the following statements:
\enumb
\item If $x$ is odd, then $x^2$ is odd.
\item If $x$ is non-zero, then $x^2$ is positive.
\enume

\item  Prove by the contrapositive method that if $a$ does not divide $b$, then the equation $ax^2+bx+b-a=0$ has no positive integer solution for $x$.

\item For each of the following statements, write the first sentences of a proof by contradiction (do not attempt to complete the proofs). Please use the phrase ``for the sake of contradiction''.
\enumb
 \item If $A\subseteq B$ and $B\subseteq C$, then $A\subseteq C$.
 \item The sum of two negative integers is a negative integer.
 \item If the square of a rational number is an integer, then the rational number must also be an integer.
\enume

\item Prove the following statements by contradiction.
\enumb
\item Consecutive integers cannot be both even.
\item Consecutive integers cannot be both odd.
\item If the sum of two primes is prime, then one of the primes must be $2$ (you may assume that every integer is either even or odd, but never both.)
\item Suppose $n$ is an integer that is divisible by $4$. Then $n+2$ is not divisible by $4$.
\item Let $A$ and $B$ be sets. Then $(A-B)\cap (B-A)=\emptyset$.
\enume

\item Prove by the method of smallest counterexample that $1+2+3+\dots+n=n(n+1)/2$ for all positive integer $n$.
\item Prove by the method of smallest counterexmaple that $n<2^n$ for all $n\in \mathbb{N}$.
\item Prove by the method of smallest counterexmaple that when $a\neq 0,1$, then
\[
a^0+a^1+a^2+\dots+a^n=\frac{a^{n+1}-1}{a-1},\qquad\forall n\in\mathbb{N}.
\]
 \item For all integers $n\geq 5$, we have $2^n>n^2$.
\enume

\end{document}