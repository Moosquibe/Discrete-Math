\documentclass{beamer}

\mode<presentation>
{
  \usetheme{Frankfurt}
  \usecolortheme{orchid}
  \setbeamercovered{invisible}
  \setbeamertemplate{footline}[frame number]
}

\usepackage[english]{babel}
\usepackage[latin1]{inputenc}
\usepackage{times}
\usepackage[T1]{fontenc}
\usepackage{tikz}
\usepackage{array}
\usepackage{cancel}


\usetikzlibrary{shapes,backgrounds}

\def\blue{\color{blue}~}
\def\black{\color{black}~}
\def\bl[#1]#2{\begin{block}{#1}#2\end{block}}
\def\integers{\mathbb{Z}}
\def\enumb{\begin{enumerate}}
\def\enume{\end{enumerate}}
\def\itemb{\begin{itemize}}
\def\iteme{\end{itemize}}


\usepackage{remreset}
\makeatletter
\@removefromreset{subsection}{section}
\makeatother
\setcounter{subsection}{1}

\title{Discrete Mathematics, Section 002, Fall 2016}
\subtitle{Lecture 7: Equivalence classes, Partitions}
\date{September 28, 2016}

\author[Zsolt]{Zsolt Pajor-Gyulai \\ \texttt{zsolt@cims.nyu.edu}}

\pgfdeclareimage[height=1cm]{NYUlogo}{NYUlogo.jpg}

\institute[NYU] 
{
\normalsize Courant Institute of Mathematical Sciences
}
\titlegraphic{\pgfuseimage{NYUlogo}}

\begin{document}

\begin{frame}
  \titlepage
\end{frame}

\AtBeginSection[]
{
\begin{frame}
\frametitle{Outline}
\tableofcontents[currentsection]
\end{frame}}

\section{Equivalence classes}
\begin{frame}{Equivalence Classes}
\only<1>{\center Do Problem 1 on your worksheet!}
\only<2>{
We have seen on Problem 1 on the worksheet that two numbers are congruent mod $2$ if and only if they are either both odd or both even. 
\itemb
\item Any two odd numbers are congruent mod $2$.
\item Any two even numbers are congruent mod $2$.
\[
even+odd\to \textrm{all } \integers
\]
\iteme

\begin{Definition}
Let $R$ be an equivalence relation on a set $A$ and let $a\in A$. The \textbf{equivalence class} of $a$, denoted $[a]$, is the set of all elements of $A$ related to $a$, that is
\[
[a]=\{x\in A: x R a\}.
\]
\end{Definition}
Example: Do Problem 2 on Worksheet!}
\end{frame}

\begin{frame}
\begin{Definition}
Let $R$ be an equivalence relation on a set $A$ and let $a\in A$. The \textbf{equivalence class} of $a$, denoted $[a]$, is the set of all elements of $A$ related to $a$, that is
\[
[a]=\{x\in A: x R a\}.
\]
\end{Definition}
For example, let $\equiv$ (mod $2$). Then
\[
[1]=\{x\in\integers: x\equiv 1 \textrm{ (mod $2$)}\}
\]
This is the set of all integers $x$ such that 
\[
2|(x-1),\qquad\textrm{ i.e.}\qquad x-1=2k
\]
for some $k\in\integers$. Therefore $x=2k+1$ and thus $x$ is odd.
\end{frame}

\begin{frame}
\begin{Definition}
Let $R$ be an equivalence relation on a set $A$ and let $a\in A$. The \textbf{equivalence class} of $a$, denoted $[a]$, is the set of all elements of $A$ related to $a$, that is
\[
[a]=\{x\in A: x R a\}.
\]
\end{Definition}
For example, let $\equiv$ (mod $2$). Then
\[
[1]=\{x\in\integers: x\equiv 1 \textrm{ (mod $2$)}\}=\textrm{odd numbers}
\]
\[
[0]=\{x\in\integers: x\equiv 0 \textrm{ (mod $2$)}\}=\uncover<2->{\textrm{even numbers}}
\]
\uncover<3->{What about $[3]$? $\to$ Problem 3 Worksheet!}
\end{frame}

\begin{frame}{Equivalence class fun facts.}
1. Every element is the member of its own equivalence class.
\bl[Proposition]{
Let $R$ be an equivalence relation on a set $A$ and let $a\in A$. Then 
\[a\in[a].
\]
}
\begin{proof}
Note that $[a]=\{x\in A: x R a\}$. To show that $a\in [a]$, we just need to show that $a R a$, and that is true by definition since $R$ is reflexive.
\end{proof}

2. The union of all equivalence classes is $A$.
\bl[Proposition]{
\[
\cup_{a\in A}[a]=A,\qquad \textrm{(Problem 4 on Worksheet!)}
\]
}
\end{frame}

\begin{frame}
3. Equivalent elements have identical equivalence classes.
\bl[Proposition]{
Let $R$ be an equivalence relation on a set $A$ and let $a,b\in A$. Then $a R b$ if and only if $[a]=[b]$.
}
\begin{proof}
\itemb
\item[($\Rightarrow$)] Suppose $a R b$, we will show that $[a]$ and $[b]$ are the same. Suppose $x\in[a]$. This means that $x R a$. Since $a R b$, we have (by transitivity) $x R b$. Therefore $x\in[b]$.

~~~~On the other hand suppose $y\in [b]$, i.e. $y R b$. We are given $a R b$, and thus $b R a$ by symmetry. Transitivity implies $y R a$, i.e. $y\in [a]$. Hence $[a]=[b]$.

\item[($\Leftarrow$)] Suppose $[a]=[b]$. We have seen that $a\in [a]$. But $[a]=[b]$, so $a\in [b]$. Therefore $a R b$.\qedhere
\iteme
\end{proof}
\end{frame}

\begin{frame}
4. Two elements from the same equivalence class are equivalent.
\bl[Proposition]{Let $R$ be an equivalence relation on $A$ and $a,x,y\in A$. If $x,y\in[a]$, then $x R y$.}

\begin{proof}
Homework. 
\end{proof}
\end{frame}

\begin{frame}
5. Equivalence classes are either disjoint or coincide.
\bl[Proposition]{Let $R$ be an equivalence relation on $A$ and suppose $[a]\cap[b]\neq\emptyset$. Then $[a]=[b]$.}

\begin{proof}
Let $R$ be an equivalence relation on $A$ and suppose $[a]$ and $[b]$ are equivalence classes with $[a]\cap[b]\neq \emptyset$. Hence $\exists x\in[a]\cap[b]$. So $x R a$ and $x R b$. By symmetry, we have $a R x$ and therefore by transitivity $a R b$. We have seen that this implies $[a]=[b]$.
\end{proof}


\bl[Corollary]{The equivalence classes of an equiv. rel $R$ are nonempty, pairwise disjoint subsets of $A$ whose union is $A$.}
\end{frame}

\section{Partitions}


\begin{frame}{Definition of a partition}
\bl[Theorem]{
Let $R$ be an equivalence relation on a set $A$. The equivalence classes of $R$ are nonempty, pairwise disjoint subsets of $A$ whose union is $A$.}
In other words we say that the equivalence classes of $R$ from a partition of $A$.

\bl[Partition]{
Let $A$ be a set. A \textbf{partition} of $A$ is a set of nonempty, pairwise disjoint sets whose union is $A$.
}
\itemb
\item A partition is a subset of $2^A$. Its members are called \textbf{parts}.
\item The parts of the partition are non-empty. 
\item The parts are pairwise disjoint.
\item The union of all the parts is the original set.
\iteme
\end{frame}

\begin{frame}{Example}
Let $A=\{1,2,3,4,5,6\}$ and let
\[
\mathcal{P}=\{\{1,2\},\{3\},\{4,5,6\}\}
\]
This is a partition of $A$ into three parts.

\vspace{0.5cm}
Two trivial partitions:
\[
\{\{1,2,3,4,5,6\}\},\qquad \{\{1\},\{2\},\{3\},\{4\},\{5\},\{6\}\}.
\]
Practice: Do Problem 5 on the worksheet.
\end{frame}

\begin{frame}{Equivalence relations and partitions}
\bl[Theorem]{
Let $R$ be an equivalence relation on a set $A$. The equivalence classes of $R$ form a partition of the set $A$.}

We can also go the other way.

\bl[Definition]{
Let $\mathcal{P}$ be a partition of a set $A$. We define an equivalence relation $\stackrel{\mathcal{P}}{\equiv}$ as follows. For $a,b\in A$,
\[
a\stackrel{\mathcal{P}}{\equiv} b,\quad\Leftrightarrow \exists P\in\mathcal{P}: a,b\in P
\]
}
In other words, $a$ and $b$ are equivalent under the partition $\mathcal{P}$ provided they belong to the same part $P\in\mathcal{P}$.
\end{frame}

\begin{frame}
\bl[Proposition]{
The relation $\stackrel{\mathcal{P}}{\equiv}$ is an equivalence relation on $A$.
}
\begin{proof}
We will show that $\stackrel{\mathcal{P}}{\equiv}$ is reflexive, symmetric and transitive.
\itemb
\item We show that $\stackrel{\mathcal{P}}{\equiv}$ is reflexive. Let $a$ be an arbitrary element of $A$. Since $\mathcal{P}$ is a partition, there must be a part $P\in\mathcal{P}$ that contains $a$ since the union of all parts is the entire set. Since $a,a\in P\in\mathcal{P}$, we have $a\stackrel{\mathcal{P}}{\equiv}a$.
\item We show that $\stackrel{\mathcal{P}}{\equiv}$ is symmetric. Suppose $a\stackrel{\mathcal{P}}{\equiv}b$ for some $a,b\in A$. Then there is a $P\in\mathcal{P}$ such that $a,b\in P$. This also implies $b\stackrel{\mathcal{P}}{\equiv}a$.
\iteme
\end{proof}
\end{frame}

\begin{frame}
\bl[Proposition]{
The relation $\stackrel{\mathcal{P}}{\equiv}$ is an equivalence relation on $A$.
}
\begin{proof}
We will show that $\stackrel{\mathcal{P}}{\equiv}$ is reflexive, symmetric and transitive.

[\dots]
\itemb
\item We show that $\stackrel{\mathcal{P}}{\equiv}$ is transitive. Let $a,b,c\in A$ and suppose $a\stackrel{\mathcal{P}}{\equiv}b$, and $b\stackrel{\mathcal{P}}{\equiv}c$. Since $a\stackrel{\mathcal{P}}{\equiv}b$, there is a part $P\in\mathcal{P}$ containing both $a$ and $b$. Since $b\stackrel{\mathcal{P}}{\equiv}c$, there is a part $Q\in\mathcal{P}$ with $b,c\in Q$. Notice that $b$ is in both $P$ and $Q$. Since the parts are pairwise disjoint, this is only possible if $P=Q$. Therefore $a,c\in P$, which implies $a\stackrel{\mathcal{P}}{\equiv}c$.
\iteme
\end{proof}
\end{frame}

\begin{frame}
\bl[Proposition]{
The relation $\stackrel{\mathcal{P}}{\equiv}$ is an equivalence relation on $A$.
}
What are the equivalence classes?
\bl[Proposition]{
The equivalence classes of $\stackrel{\mathcal{P}}{\equiv}$ are exactly the parts of $\mathcal{P}$.
}

\end{frame}

\begin{frame}{Counting parts}
\bl[Question]{
How many ways can the letters in the word WORD be rearranged?
}

\bl[Answer]{
4 letters to first place, 3 choices for second, $\dots \to\quad4!=24$.}
What happens if a letter occurs more than once?

\bl[Question]{
How many different ways can the letters in the word HELLO be rearranged?
}

\end{frame}

\begin{frame}{Counting parts}
\bl[Question]{
How many different ways can the letters in the word HELLO be rearranged?
}
\itemb
\item If there were no repeated letters $\quad\to\quad 5!=120$.
\item But this counts
\[
HEL_1L_2O,\qquad HEL_2L_1O
\]
as different.
\item Guess?
\iteme

\end{frame}

\begin{frame}{Counting parts}
\bl[Question]{
How many different ways can the letters in the word HELLO be rearranged?
}
\itemb
\item Let
\[
A=\{\textrm{All rearrangements of }  H,E,L_1,L_2,O\},\qquad |A|=120.
\]
\item Next define a relation $R$ such that for $a,b\in A$,
\[
aRb\quad\Leftrightarrow\quad \textrm{$a$ and $b$ differ only by $L_1\leftrightarrow L_2$}
\]
\item Check that this is an equivalence relation.
\[
[HL_1EOL_2]=\{HL_1EOL_2,HL_2EOL_1\}
\]
\item We need to count the number of equivalence classes!
\iteme
\end{frame}

\begin{frame}{Counting parts}
\bl[Question]{
How many different ways can the letters in the word HELLO be rearranged?
}
\itemb
\item $A=\{\textrm{All rearrangements of }  H,E,L_1,L_2,O\},\qquad |A|=120.$
\item Define \textbf{an equivalence relation} $R$ such that for $a,b\in A$,
\[
aRb\quad\Leftrightarrow\quad \textrm{$a$ and $b$ differ only by $L_1\leftrightarrow L_2$}
\]
\item Every class has to elements: $|[a]|=2$ for all $a\in A$.
\item Therefore there are 
\[
|A|/|[a]|=120/2=60
\]
equivalence classes which are the possible different rearrangements when the two L's are not distinguished.
\iteme
\end{frame}

\begin{frame}{Counting parts}
\bl[Question]{
How many different ways can the letters in the word AARDVARK be rearranged?
}
Again, the problem boils down to figuring out how big the equivalence classes
\[
[R_1A_1DA_2KR_2A_3V].
\]
\vspace{-0.4cm}
\itemb
\only<1-2>{\item 2 choices for first $R$.}
\only<3>{\item 3 choices for the first $A$.}
\only<4>{\item $D$ is fixed.}
\only<5>{\item 2 choices for second $A$.}
\only<6>{\item $K$ is fixed.}
\only<7>{\item 1 choice for second $R$.}
\only<8>{\item 1 choice for last $A$.}
\only<9->{\item $V$ is fixed.}
\iteme
\vspace{0.2cm}
\[
\uncover<2->{2\cdot}\uncover<3->{3\cdot}\uncover<4->{1\cdot}\uncover<5->{2\cdot}\uncover<6->{1\cdot}\uncover<7->{1\cdot}\uncover<8->{1\cdot}\uncover<9->{1=12}
\]
\uncover<10>{And therefore the number of rearrangements is
\[
\frac{8!}{3!2!}=\frac{40320}{12}=3360.
\]
}
\end{frame}




\end{document}