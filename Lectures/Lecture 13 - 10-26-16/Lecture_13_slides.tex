\documentclass{beamer}

\mode<presentation>
{
  \usetheme{Frankfurt}
  \usecolortheme{orchid}
  \setbeamercovered{invisible}
  \setbeamertemplate{footline}[frame number]
}

\usepackage[english]{babel}
\usepackage[latin1]{inputenc}
\usepackage{times}
\usepackage[T1]{fontenc}
\usepackage{tikz}
\usepackage{array}
\usepackage{cancel}


\usetikzlibrary{shapes,backgrounds}

\def\multiset#1#2{\ensuremath{\left(\kern-.3em\left(\genfrac{}{}{0pt}{}{#1}{#2}\right)\kern-.3em\right)}}

\def\blue{\color{blue}~}
\def\black{\color{black}~}
\def\bl[#1]#2{\begin{block}{#1}#2\end{block}}
\def\integers{\mathbb{Z}}
\def\enumb{\begin{enumerate}}
\def\enume{\end{enumerate}}
\def\itemb{\begin{itemize}}
\def\iteme{\end{itemize}}


\usepackage{remreset}
\makeatletter
\@removefromreset{subsection}{section}
\makeatother
\setcounter{subsection}{1}

\title{Discrete Mathematics, Section 001, Fall 2016}
\subtitle{Lecture 13: Sequences generated by polynomials}

\author[Zsolt]{Zsolt Pajor-Gyulai \\ \texttt{zsolt@cims.nyu.edu}}
\date{October 26, 2016}

\pgfdeclareimage[height=1cm]{NYUlogo}{NYUlogo.jpg}

\institute[NYU] 
{
\normalsize Courant Institute of Mathematical Sciences
}
\titlegraphic{\pgfuseimage{NYUlogo}}

\begin{document}

\begin{frame}
  \titlepage
\end{frame}

\AtBeginSection[]
{
\begin{frame}
\frametitle{Outline}
\tableofcontents[currentsection]
\end{frame}}

\section{Sequences generated by polynomials}

\begin{frame}
Consider the following identities:
\bl[]{
\[
0^2+1^2+\dots+n^2=\frac{(2n+1)(n+1)n}{6}
\]}
\bl[]{
\[
0^3+1^3+\dots+n^3=\frac{n^2(n+1)^2}{4}
\]}
\itemb
\item These are all polynomial expressions.
\item Proving them is simple by induction.
\item How do we come up with them in the first place?
\iteme
We are going to learn:
\itemb
\item How to decide whether a sequence of numbers is generated by a polynomial expression.
\item How to determine the polynomial in question.
\iteme
\end{frame}

\begin{frame}{The difference operator}
Let $a_0,a_1,a_2,\dots$ be a sequence of numbers. We can form the new sequence
\[
a_1-a_0,\qquad a_2-a_1,\qquad a_3-a_2,\qquad\dots
\]
\bl[Definition]{If $a$ is a sequence, then $\Delta a$ is a sequence defined by 
\[
\Delta a_n=a_{n+1}-a_n.
\]
$\Delta$ is called the \textbf{difference operator}.}
For example
\[
\begin{array}{rcccccccccccccc}
a:&&0&&2&&7&&15&&26&&40&&57\\
\Delta a:&&&2&&5&&8&&11&&14&&17
\end{array}
\]
\end{frame}

\begin{frame}{The difference operator}
What does $\Delta$ do to sequences given by polynomials. For example if $a_n=n^3-5n+1$,
\begin{align*}
\Delta a_n&=a_{n+1}-a_n=\\
&=[(n+1)^3-5(n+1)+1]-[n^3-5n+1]=\\
&=n^3+3n^2+3n+1-5n-5+1-n^3+5n-1=\\
&=3n^2+3n-4
\end{align*}
\center\color{red} $\Delta$ took a degree-3 polynomial and turned it into a \\degree-2 polynomial.
\end{frame}

\begin{frame}{The difference operator}
\bl[Proposition]{Let $a$ be a sequence of numbers in which $a_n$ is given by a degree-d polynomial in $n$ where $d\geq 1$. Then $\Delta a$ is a sequence given by a polynomial of degree $d-1$.}

\bl[Proof]{
Suppose\vspace{-0.3cm}
\[
a_n=c_dn^d+c_{d-1}n^{d-1}+\dots+c_1 n+c_0,\qquad c_d\neq 0.
\]\vspace{-0.8cm}\\
Then\vspace{-0.3cm}
\begin{align*}
\Delta a_n&=a_{n+1}-a_n=\\
&=\left[c_d(n+1)^d+c_{d-1}(n+1)^{d-1}+\dots c_1(n+1)+c_0\right]-\\
&-\left[c_dn^d+c_{d-1}n^{d-1}+\dots+c_1n+c_0\right].
\end{align*}\\ \vspace{-0.2cm}
[...]
}
\end{frame}

\begin{frame}
\bl[Proposition]{Let $a$ be a sequence of numbers in which $a_n$ is given by a degree-d polynomial in $n$ where $d\geq 1$. Then $\Delta a$ is a sequence given by a polynomial of degree $d-1$.}

\bl[Proof]{
[...]\vspace{-0.2cm}
\begin{align*}
\Delta &a_n=\left[c_d(n+1)^d+c_{d-1}(n+1)^{d-1}+\dots c_1(n+1)+c_0\right]-\\
&-\left[c_dn^d+c_{d-1}n^{d-1}+\dots+c_1n+c_0\right]=\\
&=c_d\left[(n+1)^d-n^d\right]+c_{d-1}\left[(n+1)^{d-1}-n^{d-1}\right]+\dots\\
&+c_1[(n+1)-n]+c_0[1-1]
\end{align*}\\ \vspace{-0.2cm}
[...]
}
\end{frame}

\begin{frame}
\bl[Proposition]{Let $a$ be a sequence of numbers in which $a_n$ is given by a degree-d polynomial in $n$ where $d\geq 1$. Then $\Delta a$ is a sequence given by a polynomial of degree $d-1$.}

\bl[Proof]{
[...]\vspace{-0.3cm}
\begin{align*}
\Delta &a_n=c_d\left[(n+1)^d-n^d\right]+c_{d-1}\left[(n+1)^{d-1}-n^{d-1}\right]+\dots\\
&+c_1[(n+1)-n]+c_0[1-1]
\end{align*}\\ \vspace{-0.2cm}
Note that by the Binomial theorem,\vspace{-0.3cm}
\[
(n+1)^j=n^j+\sum_{k=0}^{j-1}\binom{j}{k}n^k\qquad j=0,\dots d
\]
and therefore $(n+1)^j-n^j$ is a polynomial of degree $j-1$. This proves the claim.\qed
}
\end{frame}

\begin{frame}{Multiple applications of $\Delta$}
\[
\begin{array}{rcccccccccccccc}
a:&&0&&2&&7&&15&&26&&40&&57\\
\Delta a:&&&2&&5&&8&&11&&14&&17&\\
\Delta^2 a:&&&&3&&3&&3&&3&&3&&\\
\Delta^3 a:&&&&&0&&0&&0&&0&&&
\end{array}
\]
\bl[Corollary]{If a sequence $a$ is generated by a polynomial of degree $d$, then $\Delta^{d+1}a$ is the all-zeros sequence.}
\center\color{red}Now we seek to prove the converse of this!
\end{frame}

\begin{frame}{Properties of $\Delta$}
\bl[Proposition]{Let $a$, $b$, be sequences of numbes and let $s$ be a number.
\enumb
\item If, for all $n$, then $\Delta (a_n+b_n)=\Delta a_n+\Delta b_n$.
\item If, for all $n$, then $\Delta(s a_n)=s\Delta a_n$.
\enume}
\begin{proof}
(1),(2) are simple (Worksheet Problem 1).
\end{proof}

\end{frame}

\begin{frame}{Binomial coefficients as polynomials}
Let 
\[
a_n=\binom{n}{3}=\frac{n!}{(n-3)!3!}=\frac{n(n-1)(n-2)}{3\cdot 2\cdot 1}=\frac{1}{6}n(n-1)(n-2)
\]

Strictly speaking we only proved this for $n\geq 3$. Note, however that it applies perfectly to $n=0,1,2$ as well.

\begin{center}{\textcolor{red}{This is a polynomial!}}
\end{center}
\[
\begin{array}{rcccccccccccccccc}
a:&&0&&0&&0&&1&&4&&10&&20&&35\\
\Delta a:&&&0&&0&&1&&3&&6&&10&&15\\
\Delta^2 a:&&&&0&&1&&2&&3&&4&&5&&6\\
\Delta^3 a:&&&&&1&&1&&1&&1&&1&&1\\
\Delta^4 a:&&&&&&0&&0&&0&&0&&0&&0
\end{array}
\]
\begin{center}\textcolor{blue}{You can find Pascal's triangle in this in the $\nearrow$ direction}\end{center}
\end{frame}

\begin{frame}{Binomial coefficients as polynomials}
\[
\begin{array}{rcccccccccccccccc}
a:&&0&&0&&0&&1&&4&&10&&20&&35\\
\Delta a:&&&0&&0&&1&&3&&6&&10&&15\\
\Delta^2 a:&&&&0&&1&&2&&3&&4&&5&&6\\
\Delta^3 a:&&&&&1&&1&&1&&1&&1&&1\\
\Delta^4 a:&&&&&&0&&0&&0&&0&&0&&0
\end{array}
\]

\begin{align*}
\Delta\binom{n}{3}&=\binom{n+1}{3}-\binom{n}{3}\\
&=\frac{1}{6}(n+1)n(n-1)-\frac{1}{6}n(n-1)(n-2)\\
&=\frac{(n^3-n)-(n^3-3n^2+2n)}{6}=\frac{3n^2-3n}{6}\\
&=\frac{1}{2}n(n-1)=\binom{n}{2}
\end{align*}
\end{frame}

\begin{frame}{Binomial coefficients as polynomials}


\bl[Proposition]{If $k>0$ and $a_n=\binom{n}{k}$, then
\begin{enumerate}
\item[1)] $\Delta a_n=\binom{n}{k-1}$.
\item[2)] $a_0=\Delta a_0=\Delta^2 a_0=\dots=\Delta^{k-1}a_0=0$
but $\Delta^k a_0=1$.
\end{enumerate}
}
\begin{proof}
To see $1)$, note that.
\[
\Delta\binom{n}{k}=\binom{n+1}{k}-\binom{n}{k}=\binom {n}{k-1} \qquad k\geq n. 
\]
by Pascal's identity. For $k>n$, all three binomial coefficients are zero and therefore the identity holds again.

~~~~$2)$ is homework.
\end{proof}

\end{frame}

\begin{frame}{Characterizing properties of polynomial sequences}
For the sequence $a_n=\binom{n}{k}$, we know
\begin{itemize}
\item $\Delta^{k+1}a_0=0$ for all $n$.
\item The value of $a_0$.
\item The value of $\Delta^{j}a_0$ for all $1\leq j<k$.
\end{itemize}

This is enough to determine a polynomial sequence $a_n$!

\bl[Proposition]{
Let $a$ and $b$ be sequences of numbers and let $k$ be a positive integer. Suppose that
\begin{itemize}
\item $\Delta^ka_n$ and $\Delta^kb_n$ are zero for all $n$,
\item $a_0=b_0$,
\item $\Delta^ja_0=\Delta^jb_0$ for all $1\leq j<k$.
\end{itemize}
Then $a_n=b_n$.
}
\end{frame}

\begin{frame}{Characterizing properties of polynomial sequences}
\bl[Proposition]{
Let $a$ and $b$ be sequences of numbers and let $k$ be a positive integer. Suppose that
\begin{itemize}
\item $\Delta^ka_n$ and $\Delta^kb_n$ are zero for all $n$,
\item $a_0=b_0$,
\item $\Delta^ja_0=\Delta^jb_0$ for all $1\leq j<k$.
\end{itemize}
Then $a_n=b_n$.
}

\bl[Proof]{
We prove this by induction on $k$. The basis case is $k=1$, in which case we already have\vspace{-0.3cm}
\[
\Delta a_n=\Delta b_n=0
\]\\\vspace{-0.3cm}
which means that both sequences are constants. Since we have $a_0=b_0$, this means that the sequences are identical.[...]
}

\end{frame}

\begin{frame}{Characterizing properties of polynomial sequences}
Hypotheses:
\begin{itemize}
\item $\Delta^ka_n$ and $\Delta^kb_n$ are zero for all $n$,
\item $a_0=b_0$,
\item $\Delta^ja_0=\Delta^jb_0$ for all $1\leq j<k$.
\end{itemize}
\begin{proof}

[...]Suppose now that the result is true for $k=l$ and let $a$ and $b$ be sequences satisfying the hypotheses of the theorem with $k=l+1$. Form the new sequences\vspace{-0.2cm}
\[
a_n'=\Delta a_n,\qquad b_n'=\Delta b_n.
\]\\\vspace{-0.2cm}
Clearly, $a_n'$ and $b_n'$ satisfies the hypotheses with $k=l$ and therefore $a'=b'$.

[...]

\end{proof}
\end{frame}

\begin{frame}{Characterizing properties of polynomial sequences}
\begin{proof}

[...]\vspace{-0.2cm}
\[
a_n'=\Delta a_n,\qquad b_n'=\Delta b_n.
\]\\\vspace{-0.2cm}
Then $a_n'=b_n'$ for every $n$.\vspace{0.2cm}

Now we show that $a_n=b_n$ for every $n$. Suppose FTSC that $a$ and $b$ were different. Then there is a smallest $m$ such that\vspace{-0.2cm}
\[
a_m\neq b_m
\]\\\vspace{-0.2cm}
Note that $m\neq 0$ because $a_0=b_0$ and that by the minimality of $m$, we have $a_{m-1}=b_{m-1}$. Then\vspace{-0.2cm}
\[
a_m-a_{m-1}=a_{m-1}'=b_{m-1}'=b_m-b_{m-1}
\]\\\vspace{-0.3cm}
which implies\vspace{-0.2cm}
\[
a_m-b_m=a_{m-1}-b_{m-1}=0.
\]\\\vspace{-0.2cm}
This means $a_m=b_m$ $\Rightarrow\Leftarrow$.
\end{proof}
\end{frame}


\begin{frame}
\bl[Theorem]{Let $a$ be a sequence of numbers. The terms $a_n$ can be expressed as a polynomial expression in $n$ if and only if there is a non-negative integer $k$ such that for all $n\geq 0$, we have $\Delta^{k+1}a_n=0$. In this case,
\[
a_n=a_0\binom{n}{0}+(\Delta a_0)\binom{n}{1}+(\Delta^2a_0)\binom{n}{2}+\dots+(\Delta^ka_0)\binom{n}{k}.
\]}
\begin{proof}
We have already proved that $a_n$ is a polynomial of degree $d$, then $\Delta^{d+1}a_n=0$, we only have to prove the other direction and the formula. This is going to be a reading exercise on your homework.
\end{proof}
\end{frame}

\begin{frame}{An example}
\[
\begin{array}{rcccccccccccccc}
a:&&0&&2&&7&&15&&26&&40&&57\\
\Delta a:&&&2&&5&&8&&11&&14&&17&\\
\Delta^2 a:&&&&3&&3&&3&&3&&3&&\\
\Delta^3 a:&&&&&0&&0&&0&&0&&&
\end{array}
\]
Then by the theorem with $k=3$,
\[
a_n=0\binom{n}{0}+2\binom{n}{1}+3\binom{n}{2}=0+2n+3\frac{n(n-1)}{2}=\frac{n(3n+1)}{2}.
\]

\center Do problems 2-3 from the worksheet!
\end{frame}



\end{document}