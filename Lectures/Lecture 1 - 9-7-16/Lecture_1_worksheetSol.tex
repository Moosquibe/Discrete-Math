\documentclass[11pt]{preprint}

\setlength{\topmargin}{0mm} \setlength{\oddsidemargin}{0mm}
\setlength{\textwidth}{160mm} \setlength{\textheight}{215mm}

\usepackage{amssymb,amsmath,amscd,amsthm}

\title{Discrete Mathematics, 2016 Fall - Worksheet 1}
\author{Instructor: Zsolt Pajor-Gyulai, CIMS}
\date{September 7, 2016}

\newtheorem{Solution}{Solution}

\begin{document}

\maketitle

In all of the above problems explain your answer in full English sentences.

\begin{enumerate}
\item Please determine which of the following are true and which are false using the definition of divisibility. 
\begin{enumerate}
\item $3|99$ YES, because $33$ is an integer such that $99=3\cdot 33$.
\item $-3|3$ YES, \dots.
\item $-5|-5$ YES, \dots.
\item $-2|-7$ NO, the only number k such that $-7=-2*k$ is $3.5$ which is not an integer.
\item $0|4$ NO, because for every integer $k$, we have $0\cdot k = 0 \neq 4$.
\item $4|0$ YES, because for any integer $k$, we have $0 = 4\cdot k$.
\item $0|0$ YES, because for any integer $k$, we have $0 = 0\cdot k$.
\end{enumerate}
\item Alternative definition to divisibility: We say that an integer $a$ is divisible by an integer $b$ provided $\frac{a}{b}$ is an integer. Explain why this alternative definition is different than the original one. Here different means different concepts. To answer this question you should find integers $a$ and $b$ such that $a$ is divisible by $b$  according to one definition but $a$ is not divisible by $b$ according to the other definition.

\begin{Solution}
According to the definition, $0$ divides $0$ (see Problem 1 (g)). However, $\frac{0}{0}$ is not defined and therefore it cannot be an integer.
\end{Solution}

\item Show using the definition of divisibility (the original one) that if $a,b,c$ are integers and $c|a$ and $c|b$ then $c|a+b$.

\begin{Solution}
Since $c|a$ and $c|b$, there are integers $k_1$ and $k_2$ such that $a=k_1c$ and $b=k_2c$. Then
\[
a+b=k_1c+k_2c=(k_1+k_2)c.
\]
As $k_1+k_2$ is an integer, this means that $c|(a+b)$ and the claim is proved.
\end{Solution}
\newpage
\item None of the following numbers is prime. Explain why they fail to satisfy the definition of a prime number. Which one of these is a composite number?
\begin{enumerate}
\item 21. Not a prime because for example $7$ is a positive divisor that is not $1$ and not $21$.
\item 0. Not a prime because $0\not> 1$.
\item $\pi$. Not a prime because it is not an integer.
\item $-2$. Not a prime because $-2\not> 1$.
\item $1/2$. Not a prime because it is not an integer.
\end{enumerate}


\item Formulate a definition of the following concepts:
\begin{enumerate}
\item Define what it means for a number to be the \textbf{square root} of another number.

\vspace{0.4cm}
The number $a$ is the square root of a number $b$ provided $a\cdot a = b$.
\vspace{0.4cm}

\item Midpoint of a line segment. (Assume the notion of distance and line segment is given.)

\vspace{0.4cm}
The midpoint of a line segment is the point on the line segment which is at equal distance from the two endpoints.
\vspace{0.4cm}
\item Teenager. (Assume the notions of age and human are given.)

\vspace{0.4cm}
A teenager is a human whose age is between $10$ and $19$.
\vspace{0.4cm}

\item Grandmother. (Assume the notion of child and woman are given.)

\vspace{0.4cm}
A grandmother is a woman who has a child and whose child has a child.
\end{enumerate}


\end{enumerate}


\end{document}