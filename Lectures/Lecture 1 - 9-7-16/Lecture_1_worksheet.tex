\documentclass[11pt]{preprint}

\setlength{\topmargin}{0mm} \setlength{\oddsidemargin}{0mm}
\setlength{\textwidth}{160mm} \setlength{\textheight}{215mm}

\usepackage{amssymb,amsmath,amscd,amsthm}

\title{Discrete Mathematics, 2016 Fall - Worksheet 1}
\author{Instructor: Zsolt Pajor-Gyulai, CIMS}
\date{September 7, 2016}



\begin{document}

\maketitle

In all of the above problems explain your answer in full English sentences.


\begin{enumerate}
\item Please determine which of the following are true and which are false using the definition of divisibility. 
\begin{enumerate}
\item $3|99$,
\item $-3|3$,
\item $-5|-5$,
\item $-2|-7$,
\item $0|4$,
\item $4|0$,
\item $0|0$.
\end{enumerate}
\item Alternative definition to divisibility: We say that an integer $a$ is divisible by an integer $b$ provided $\frac{a}{b}$ is an integer. Explain why this alternative definition is different than the original one. Here different means different concepts. To answer this question you should find integers $a$ and $b$ such that $a$ is divisible by $b$  according to one definition but $a$ is not divisible by $b$ according to the other definition.

\item Show using the definition of divisibility (the original one) that if $a,b,c$ are integers and $c|a$ and $c|b$ then $c|a+b$.

\item None of the following numbers is prime. Explain why they fail to satisfy the definition of a prime number. Which one of these is a composite number?
\begin{enumerate}
\item 21,
\item 0,
\item $\pi$,
\item $-2$
\item $1/2$.
\end{enumerate}
\newpage

\item Formulate a definition of the following concepts:
\begin{enumerate}
\item Define what it means for a number to be the \textbf{square root} of another number.
\item Midpoint of a line segment. (Assume the notion of distance and line segment is given)
\item Teenager. (Assume the notions of age and human are given)
\item Grandmother. (Assume the notion of child and woman are given)
\end{enumerate}

\end{enumerate}

\textbf{Optional computer exercises - no credit}

\begin{enumerate}
\item [PE 1)] If we list all the natural numbers below 10 that are multiples of 3 or 5, we get 3, 5, 6 and 9. The sum of these multiples is 23. Find the sum of all the multiples of 3 or 5 below 1000.
\item [PE 7)] By listing the first six prime numbers: 2, 3, 5, 7, 11, and 13, we can see that the 6th prime is 13. What is the 10 001st prime number?
\end{enumerate}


\end{document}