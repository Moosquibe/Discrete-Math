\documentclass[11pt]{preprint}

\setlength{\topmargin}{0mm} \setlength{\oddsidemargin}{0mm}
\setlength{\textwidth}{160mm} \setlength{\textheight}{215mm}

\usepackage{amssymb,amsmath,amscd,amsthm}

\title{Discrete Mathematics, 2016 Fall - HW 1}
\author{Instructor: Zsolt Pajor-Gyulai}
\institute{Courant Institute of Mathematical Sciences, NYU}
\date{Due date: September 14, 2016}


\begin{document}

\maketitle

To get full credit  in all of the problems, use rigorous justification and unless otherwise indicated, make sure that your solution reads as a perfect English sentence. You should only assume integers, operations and order relations as given. If you use a statement or a definition from the textbook, make sure to indicate it.
\vspace{0.2cm}

\textbf{Section 3}
\begin{enumerate}
\item[5)] A rational number is a number formed by dividing two integers $a/b$ where $b\neq 0$. The set of all rational numbers is denoted by $\mathbb{Q}$. Explain why every integer is a rational number, but not all rational numbers are integers.

\item[6)] Define what it means for an integer to be a perfect square. For example, the integers $0, 1,4,9$ and $16$ are perfect squares. Your definition should begin as: "An integer $x$ is called a perfect square provided".

\item[12)] How many positive divisors does each of the following numbers have?
\begin{enumerate}
\item $8$,
\item $32$,
\item (Optional, hand in only to check yourself) $2^n$.
\end{enumerate}
In the first two, you can just list the divisors as justification. For the third one however you need to justify your answer rigorously.


\item[13)] An integer $n$ is called perfect provided it equals the sum of all of its divisors that are both positive and less than $n$. For example, $28$ is perfect because the positive divisors of $28$ are $1,2,4,7,14$ and $28$. Note that $1+2+4+7+14=28$.
\begin{enumerate}
\item There is a perfectsmaller than $28$. Find it.
\item (Optional) Write computer code to find the next perfect number.
\end{enumerate}
\end{enumerate}


\end{document}