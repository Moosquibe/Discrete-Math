\documentclass[11pt]{preprint}

\setlength{\topmargin}{0mm} \setlength{\oddsidemargin}{0mm}
\setlength{\textwidth}{160mm} \setlength{\textheight}{215mm}

\usepackage{amssymb,amsmath,amscd,amsthm}

\newtheorem{proposition}{Proposition}

\def\enumb{\begin{enumerate}}
\def\enume{\end{enumerate}}
\def\integers{\mathbb{Z}}

\title{Discrete Mathematics, 2016 Spring - HW 3}
\author{Instructor: Zsolt Pajor-Gyulai}
\institute{Courant Institute of Mathematical Sciences, NYU}
\date{Due: September 28, 2016}



\begin{document}

\maketitle

To get full credit  in all of the problems, use rigorous justification and unless otherwise indicated, make sure that your solution reads as a perfect English sentence. You should only assume integers, operations and order relations as given. If you use a statement or a definition from the textbook, make sure to indicate it.
\vspace{0.2cm}

\textbf{Section 9}

\begin{enumerate}
\item[7)] Computing $n!$ for large values of $n$ can computationally costly. Stirling's formula gives the following approximation:
\[
n!\approx \sqrt{2\pi n} \left(\frac{n}{e}\right)^n
\]
Compute this with $n=10$. Compute the actual value and compare the two.

\item[8)] Calculate the following products:
\begin{enumerate}
\item $\prod_{k=-3}^1k$.
\item $\prod_{k=1}^n\frac{k+1}{k}$.
\end{enumerate}

\item[15)] The double factorial is defined for odd integers. It is the product of all the odd numbers from $1$ to $n$ inclusive. E.g.
\[
7!!=1\times 3\times 5\times 7=105
\]
\begin{enumerate}
\item Evaluate $9!!$.
\item Write an expression for $n!!$ using the product notation.
\item For an odd number, are $n!!$ and $(n!)!$ the same?
\end{enumerate}

\end{enumerate}

\textbf{Section 10}

\enumb


\item[12-13)]
\enumb
\item Let $C=\{x\in\integers: x|12\}$ and let $D=\{x\in\integers: x|36\}$. Prove that $C\subseteq D$.
\item Generalize the previous problem. Let $c,d\in\integers$ and let
\[
C=\{x\in\integers: x|c\},\qquad D=\{x\in\integers: x|d\}
\]
Find and prove a necessary and sufficient condition for $C\subseteq D$.
\enume
%\item[15)] With $T$ and $P$ as in Proposition 10.5, show that $T\neq P$.
\enume 

\textbf{Section 11}
\enumb

\item[7)] The notation $\exists !$ is sometimes used to indicate that there is exactly one object that satisfies the condition. The notation can be pronounced "there is a unique". Which of the following statements are true?
\enumb
\item $\exists ! x\in\mathbb{N}, x^2=4$
\item $\exists ! x\in\integers, x^2=4$
\item $\exists ! x\in\integers,\forall y\in\integers, xy=x$
\enume
\item[8)] A subset of the plane is a \textbf{convex region} provided that, given any two points of the region, every point on the line segment connecting the two is also in that region.
\enumb
\item Rewrite the definition of convex region using quantifiers. Use $R$ to stand for the region and $L(a,b)$ to stand for the line segment with endpoints $a$ and $b$.
\item Using quantifiers, write what it means for a region not to be convex.
\item Write your answer to the previous part in English without quantifiers.
\enume
\enume

\textbf{Section 12}
\enumb
\item[19)] Prove the following proposition.

\begin{proposition}
Let $A$ and $B$ be finite sets. Then $|A\times B|=|B\times A|=|A|\cdot|B|$.
\end{proposition}

\item[22)] Let $X$ be a set and let $A\subseteq X$ be a subset. The \textbf{complement} of $A$ relative to $X$ is the set $X-A$. When it is well understood what the big set ("Universe") $X$ is, e.g. when you are writing hundreds of pages about a theory where $X$ is always the same thing (like $X=\integers$ in number theory or the probability space in  probability theory), then the shorthand $\bar{A}$ is used.  Prove the following about set complements (with $A,B\subseteq X$).
\enumb
\item $A=B$ if and only if $\bar{A}=\bar{B}$.
\item $\bar{\bar{A}}=A$.
\item $\overline{A\cup B\cup C}=\bar{A}\cap\bar{B}\cap\bar{C}$.
\enume

\textbf{Section 13}
\item[3)]Expanding the parentheses gives
\[
(x-1)(1+x+x^2+\dots x^{n-1}) = x^n-1
\]
for any $x$. On Problem 7 of Worksheet 5, we gave a combinatorial proof of the case $x=2$. Substituting $x=3$, we get
\[
2\cdot 3^0+ 2\cdot 3^1+\dots + 2\cdot 3^{n-1}=3^n-1
\]
Give a combinatorial proof for this formula. For a hint on what the right question to ask is, you may consult the back of the textbook.
\enume


\end{document}