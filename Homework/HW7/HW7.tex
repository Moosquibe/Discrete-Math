\documentclass[11pt]{preprint}

\setlength{\topmargin}{0mm} \setlength{\oddsidemargin}{0mm}
\setlength{\textwidth}{160mm} \setlength{\textheight}{215mm}

\usepackage{amssymb,amsmath,amscd,amsthm}
\usepackage{tikz}

\newtheorem{proposition}{Proposition}

\def\enumb{\begin{enumerate}}
\def\enume{\end{enumerate}}
\def\integers{\mathbb{Z}}
\def\multiset#1#2{\ensuremath{\left(\kern-.3em\left(\genfrac{}{}{0pt}{}{#1}{#2}\right)\kern-.3em\right)}}

\title{Discrete Mathematics, 2016 Spring - HW 7}
\author{Instructor: Zsolt Pajor-Gyulai}
\institute{Courant Institute of Mathematical Sciences, NYU}



\begin{document}

\maketitle

To get full credit  in all of the problems, use rigorous justification and unless otherwise indicated, make sure that your solution reads as a perfect English sentence. You should only assume integers, operations and order relations as given. If you use a statement or a definition from the textbook, make sure to indicate it.
\vspace{0.2cm}

%\textbf{Section 22}
%\enumb
%\item[24)] Prove, using strong induction, that every natural number can be expressed as the sum of distinct powers of $2$. For example $21=2^4+2^2+2^0$ (Use the hint in the back of your textbook).
%\enume

\textbf{Section 23}
\enumb

%\item[N/A)] Prove the following proposition by induction.
%
%\begin{proposition}
%Let $a_n$ be the solution of the recurrence relation
%\[
%a_n=sa_{n-1}+t.
%\]
%Show that when $s\neq 1$,\vspace{-0.1cm}
%\[
%a_n=s^n\left(a_0+\frac{t}{s-1}\right)-\frac{t}{s-1}
%\]
%and $a_n=a_0+nt$ when $s=1$.
%\end{proposition}


\item[2)] Solve each of the following recurrence relations by giving an explicit formula for $a_n$.
\enumb
\item $a_n=10a_{n-1}, a_0=3$.
\item $a_n=a_{n-1}+3, a_0=2$.
\item $a_n=7a_{n-1}-2, a_0=-1$.
\item $F_n=F_{n-1}+F_{n-2}, F_0=1, F_1=1$. (Fibonacci sequence)
\item $a_n=2a_{n-1}-a_{n-2}, a_0=2, a_1=-1$.
\enume
%\item[11)] For a natural number $n$, the $n$-cube is a figure created by the following recipe. The $0$-cube is simply a point. For $n>0$, we construct an $n$-cube by taking two disjoint copies of an $(n-1)$-cube and then joining corresponding points in the two cubes by line segments. Thus, a $1$-cube is simply a line segment and a $2$-cube is a quadrilateral. The figure on p164 shows the construction of a $4$-cube from two copies of a $3$-cube. Note that an $n$-cube has twice as many points as an $(n-1)$-cube; Therefore, an $n$ cube has $2^n$ points. The question is, how many line segments does an $n$-cube have? Let $a_n$ denote this number, then we have $a_0=0$, $a_1=1$, $a_2=4$, $a_3=12$, $a_4=32$.
%\enumb 
%\item Find $a_5$.
%\item Find a formula for $a_n$ in terms of $a_{n-1}$.
%\item Use part (b) to find a formula for $a_n$ just in terms of $n$.
%\enume

\item[15)] Extrapolate the discussion during lecture and solve the following third-order recurrence relation.
\[
a_n=4a_{n-1}-a_{n-2}-6a_{n-3},a_0=8,a_1=3, a_2=27.
\]

\item[17)] There are many types of non-linear recurrence relations that are of different forms from those presented in the lecture. Most of them can be really hard to solve, however sometimes a little guesswork reveals the solution. Try your hand at conjecturing a solution to the following ones and then prove them by induction.
\enumb
\item $a_n=na_{n-1},a_0=1$.
\item $a_n=a_{n-1}^2,a_0=2$.
\item $a_n=a_0+a_1+a_2+\dots+a_{n-1},a_0=1$.
\enume

\item[3)] Each of the following sequences is generated by a polynomial expression. For each, find the polynomial expression that gives $a_n$.
\enumb
\item $4,4,10,28,64,124,214,340,508,724$
\item $5,16,41,116,301,680,1361,2476,4181,6656$
\enume


\item[5)] 
 Let $k$ be a positive integer and let $a_n=\binom{n}{k}$. Prove that $a_0=\Delta a_0=\Delta^2a_0=\dots=\Delta^{k-1}a_0=0$ and that $\Delta^ka_0=1$.
 
\item[7)]
Find a polynomial formula for $1^4+2^4+3^4+\dots+n^4$.


\item[N/A)] Read and understand the proof of Theorem 23.17 on page 161. You do not need to hand in anything for this problem, the point of this problem is to practice reading proofs even if otherwise you don't look at the textbook.
\enume


\end{document}