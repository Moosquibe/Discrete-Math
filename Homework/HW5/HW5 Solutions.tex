\documentclass[11pt]{article}
\usepackage{latexsym,amsmath,amssymb,epsfig,amsthm}
\usepackage{multirow}
%\newcommand\bbR{\mathbb R}
%\def\xb{{\bf x}}%
%\def\yb{{\bf y}}%
%\def\zb{{\bf z}}%
%\def\vb{{\bf v}}%
%\def\wb{{\bf w}}%
%\def\ib{{\bf i}}%
%\def\jb{{\bf j}}%
%\def\kb{{\bf k}}%
%\def\gb{{\bf g}}%
%\def\fb{{\bf f}}%
%\def\cb{{\bf c}}%
%\def\nb{{\bf n}}%
%
\begin{document}

\vspace{-1in}

\begin{center}
{\bf Discrete Mathematics, 2016 Spring - HW 5 Solutions} 
\end{center}

\begin{center}
Grader XZ\\
March 1, 2016\\
\end{center}
\vspace{0.1in}
\begin{center}  {\bf General comments:\\}
\end{center}  
{\bf
$\bullet$ As some of you might have noticed, somtimes the electronic submission does not work the way you expected. Especially if you are submitting Word documents, once after you upload it to NYU Classes your math formulas might not stay the way you format it, and sometimes they might even disappear, which would cost you many points on your homework. So make sure you have everything perfectly formatted before your submission and check your document on NYU Classes once after you submit it. \\
\\
$\bullet$ One quick trick to "freeze" the formatting is to convert your Word document into a PDF file. More generally, scientists and mathematicians commonly use a software called LaTex to ensure nice and clear formatting for the purpose of communication and publication of scientific documents. The quizzes, exams, and weekly assignments that you have seen are usually generated by LaTex. And so is the hw solutions. So I would highly suggest you learn it now and get used to it if you are thinking about going to graduate school.\\
\\
$\bullet$ If you want to skip the tedious job of formmating your solutions, just go old-fashioned with pen and paper! You are not required to do electronic submission for this course and it is usually a safer option. \\
}


\vspace{0.3in}
\\
{\bf 
\begin{Large}
Section 18
\end{Large}}
\\
\\
{\bf (8) Express $\[
 \bigg (\[
 \bigg (\begin{tabular}{c}
  n\\
  k  
  \end{tabular}
\bigg) \bigg )$ using factorial notation.\\
N/A) 8 identical prizes are given out to chosen students in a class of size 32.\\
(a) How many ways can this be done if one student can only get one prize?\\
(b) How many ways can this be done if any student can get any number of prizes?}\\
\\
{\bf Solution:} \\
By definition, $\[
 \bigg (\[
 \bigg (\begin{tabular}{c}
  n\\
  k  
  \end{tabular}
\bigg) \bigg )$ = $\[
 \bigg (\begin{tabular}{c}
  n+k-1\\
  k  
  \end{tabular}
\bigg )$ = $\frac{(n+k-1)!}{k!\cdot (n+k-1-k)!}$ = $\frac{(n+k-1)!}{k!\cdot (n-1)!}$. \\
\\
(a) Part(a) is a normal set counting problem: \\
$\[
 \bigg (\begin{tabular}{c}
32\\
  8  
  \end{tabular}
\bigg )$ = $\frac{32!}{(32-8)! \cdot 8!}$ = $\frac{32!}{24! \cdot 8!}$\\
\\
(b) Part(b) is a multiset set counting problem: \\
$\[
 \bigg (\[
 \bigg (\begin{tabular}{c}
  32\\
  8  
  \end{tabular}
\bigg) \bigg )$ = $\[
 \bigg (\begin{tabular}{c}
  32+8-1\\
  8  
  \end{tabular}
\bigg )$ =  $\[
 \bigg (\begin{tabular}{c}
39\\
  8  
  \end{tabular}
\bigg )$ = $\frac{39!}{(39-8)! \cdot 8!}$ = $\frac{39!}{31! \cdot 8!}$\\
\\
\\
\\
{\bf (7,11) (a) Calculate
$\[
 \bigg (\[
 \bigg (\begin{tabular}{c}
  8\\
  4  
  \end{tabular}
\bigg) \bigg )$ 
 and
$\[
 \bigg (\[
 \bigg (\begin{tabular}{c}
  4\\
  8  
  \end{tabular}
\bigg) \bigg )$ . Notice anything interesting?\\
(b) Show that for any positive integer a,
$\[
 \bigg (\[
 \bigg (\begin{tabular}{c}
  2a\\
  a  
  \end{tabular}
\bigg) \bigg )$ = 2 $\[
 \bigg (\[
 \bigg (\begin{tabular}{c}
  a\\
  2a  
  \end{tabular}
\bigg) \bigg )$.}\\
\\
{\bf Solution:} \\
(a) \\
$\[
 \bigg (\[
 \bigg (\begin{tabular}{c}
  8\\
  4  
  \end{tabular}
\bigg) \bigg )$  = $\[
 \bigg (\begin{tabular}{c}
  8+4-1\\
  4  
  \end{tabular}
\bigg )$ = $\frac{11!}{4! \cdot 7!}$ = 330. \\
$\[
 \bigg (\[
 \bigg (\begin{tabular}{c}
  4\\
  8  
  \end{tabular}
\bigg) \bigg )$  = $\[
 \bigg (\begin{tabular}{c}
  8+4-1\\
  8  
  \end{tabular}
\bigg )$ = $\frac{11!}{8! \cdot 3!}$ = 165.\\
By observation, we can conclude that $\[
 \bigg (\[
 \bigg (\begin{tabular}{c}
  8\\
  4  
  \end{tabular}
\bigg) \bigg )$ = 2 $\[
 \bigg (\[
 \bigg (\begin{tabular}{c}
  4\\
  8  
  \end{tabular}
\bigg) \bigg )$.\\
\\
(b) \\
$\[
 \bigg (\[
 \bigg (\begin{tabular}{c}
  2a\\
  a  
  \end{tabular}
\bigg) \bigg )$ = $\[
 \bigg (\begin{tabular}{c}
  2a+a-1\\
  a  
  \end{tabular}
\bigg )$ = $\[
 \bigg (\begin{tabular}{c}
  3a-1\\
  a  
  \end{tabular}
\bigg )$ = \\
$\frac{(3a-1)!}{(3a-1-a)! \cdot a!}$ \\
= $\frac{(3a-1)!}{(2a-1)! \cdot a!}$ \\
= $\frac{(3a-1)!}{(2a-1)! \cdot a \cdot (a-1)!}$ \\
= $\frac{2 \cdot (3a-1)!}{[(2a-1)! \cdot (2a)]\cdot (a-1)!}$ \\
= 2 $\cdot$ $\frac{(3a-1)!}{(2a)! \cdot (a-1)!}$ \\
= 2 $\cdot \frac{(3a-1)!}{(a-1)! \cdot (2a)!}$\\
= 2 $\cdot \frac{(3a-1)!}{(a-1)! \cdot (2a)!}$\\
= 2 $\cdot \frac{(3a-1)!}{(3a-1-2a)! \cdot (2a)!}$\\
= 2 $\[
 \bigg (\begin{tabular}{c}
  3a-1\\
  2a  
  \end{tabular}
\bigg )$
= 2 $\[
 \bigg (\begin{tabular}{c}
  a+2a-1\\
  2a  
  \end{tabular}
\bigg )$ = 2 $\[
 \bigg (\[
 \bigg (\begin{tabular}{c}
  a\\
  2a  
  \end{tabular}
\bigg) \bigg )$.
\\
\\
\\
{\bf 
\begin{Large}
Section 19
\end{Large}}
\\
\\
{\bf (3) How many integers between 1 and 1, 000, 000 (inclusive) are not divisible by 2, 3, or 5?}\\
\\
{\bf Solution:} \\
Let all integers between 1 and 1, 000, 000 (inclusive) be a set called $\Omega$. Let the sbubset of $\Omgea$ composed of those divisible by 2 to be set A,  the sbubset of $\Omgea$ composed of those divisible by 3 to be set B, and the sbubset of $\Omgea$ composed of those divisible by 5 to be set C. Then we know that the number of integers in $\Omega$ that are not divisible by 2, 3, or 5 is \\
$|\Omega|$ - $|A \cup B \cup C|$ \\
= $|\Omega|$ - $[ (|A| + |B| +|C|)$ - $|A \cap B|$ - $|B \cap C|$ - $|A \cap C|$ + $|A \cap B \cap C| ]$\\
= 1000000 - [(1000000/2 + $\lfloor$1000000/3$\rfloor$ + 1000000/5) - $\lfloor$1000000/6$\rfloor$ - $\lfloor$1000000/10$\rfloor$ - $\lfloor$1000000/15$\rfloor$ + $\lfloor$1000000/30$\rfloor$]\\
 = 1000000 - [(500000 + $\lfloor$333333.33$\rfloor$ + 200000) - $\lfloor$166666.66$\rfloor$ - 100000 - $\lfloor$66666.66$\rfloor$ + $\lfloor$33333.33$\rfloor$]\\
  = 1000000 - [(1033333 - 166666 - 100000 - 66666 + 33333]\\
= 266666.\\
\\
\\
\\
{\bf (8) How many lattice paths through the grid in the figure avoids both locations A and B?}\\
\\
{\bf Solution:} \\
Let $\Omega$ be the set of all possible paths, A be the set of paths passing through A, B be the set of paths passing through B, C be the set of paths passing through both A and B. Then, $|\Omega|$ = $\[
 \bigg (\begin{tabular}{c}
  18\\
  9  
  \end{tabular}
\bigg )$, 
$|A|$ = $\[
 \bigg (\begin{tabular}{c}
  6\\
  4  
  \end{tabular}
\bigg )$$\[
 \bigg (\begin{tabular}{c}
  12\\
  5  
  \end{tabular}
\bigg )$, 
$|B|$ = $\[
 \bigg (\begin{tabular}{c}
  12\\
  6  
  \end{tabular}
\bigg )$$\[
 \bigg (\begin{tabular}{c}
  6\\
  3  
  \end{tabular}
\bigg )$,
$|C|$ = $\[
 \bigg (\begin{tabular}{c}
  6\\
  4  
  \end{tabular}
\bigg )$$\[
 \bigg (\begin{tabular}{c}
  6\\
  2  
  \end{tabular}
\bigg )$$\[
 \bigg (\begin{tabular}{c}
  6\\
  3  
  \end{tabular}
\bigg )$. Hence,$|$Possible paths avoiding both A and B$|$ = $|\Omega|$ - ($|A|$ + $|B|$ - $|A \cap B|$) = 22760.



\end{document}

