\documentclass[11pt]{preprint}

\setlength{\topmargin}{0mm} \setlength{\oddsidemargin}{0mm}
\setlength{\textwidth}{160mm} \setlength{\textheight}{215mm}

\usepackage{amssymb,amsmath,amscd,amsthm}

\newtheorem{proposition}{Proposition}

\def\enumb{\begin{enumerate}}
\def\enume{\end{enumerate}}
\def\integers{\mathbb{Z}}

\title{Discrete Mathematics, 2016 Spring - HW 4}
\author{Instructor: Zsolt Pajor-Gyulai}
\institute{Courant Institute of Mathematical Sciences, NYU}



\begin{document}

\maketitle

To get full credit  in all of the problems, use rigorous justification and unless otherwise indicated, make sure that your solution reads as a perfect English sentence. You should only assume integers, operations and order relations as given. If you use a statement or a definition from the textbook, make sure to indicate it.
\vspace{0.2cm}

\textbf{Section 14}
\enumb
\item[11)] Consider the relation $\subseteq$ on $2^{\mathbb{Z}}$. Which of the properties does $\subseteq$ have? Prove your answers.

\item[13)] The property \textbf{irreflexive} is not the same as being not reflexive. To illustrate this, please do the following:
\enumb
\item Give an example of a relation on a set that is neither reflexive nor irreflexive.
\item Give an example of a relation on a set that is both reflexive and irreflexive.
\enume
Part (a) is not too hard, but for (b), you need to create a rather strange example.


%\item[15)] Prove that a relation $R$ on a set $A$ is antisymmetric if and only if
%\[
%R\cap R^{-1}\subseteq\{(a,a):a\in A\}
%\]
\enume

\textbf{Section 15}
\enumb
\item[6)] Show that $\equiv$ is a transitive relation.

\item[7)] For each equivalence relation below, find the requested equivalence class.
\enumb
\item $R$ is has-the-same-parents-as on the set of human beings. Find [you].
\item $R$ is has-the same-size-as on $2^{|\{1,2,3,4,5\}}$. Find $[\{1,3\}]$.
\enume
\item[8)] For each of the following equivalence relations, determine the number of equivalence classes that relation has.
\enumb
\item Congruence modulo $10$ (for integers)
\item Has-the-same-birthday-as (for human beings). Here same birthday means same month/day, not necessarily same year.
\enume
\item[12)] Prove the following proposition
\begin{proposition}
Ler $R$ be an equivalence relation on a set $A$ and let $a,x,y\in A$. If $x,y\in[a]$, then $x R y$.
\end{proposition}
\enume

\textbf{Section 16}
\enumb

\item[2)] How many different anagrams (including nonsensical words) can be made from each of the following
\enumb
\item MATHEMATICS
\item MISSISSIPPI
\enume
\item[16)] How many partitions of the set $\{1,2,3,\dots,100\}$ are there that satisfy the following two properties.
\enumb
\item There are exactly three parts.
\item Elements $1,2, 3$ are in different parts.
\enume
%\item[17)] Let $A$ be a $100$-elemet set. Which one is greater: the number of partitions of $A$ into $20$ parts of size $5$ or the number of partitions of $A$ into $5$ parts of size $20$?
\item[18)] Two different coins are placed on squares of a standard $8\times 8$ chess board; they may both be placed on the same square. Let two arrangements of these coins be equivalent if we can move the coins diagonally to get from one arrangement to another. How many different (inequivalent) ways can the coins be placed on the board?
\enume


\end{document}