\documentclass[11pt]{preprint}

\setlength{\topmargin}{0mm} \setlength{\oddsidemargin}{0mm}
\setlength{\textwidth}{160mm} \setlength{\textheight}{215mm}

\usepackage{amssymb,amsmath,amscd,amsthm}
\usepackage{tikz}

\newtheorem{proposition}{Proposition}

\def\enumb{\begin{enumerate}}
\def\enume{\end{enumerate}}
\def\integers{\mathbb{Z}}
\def\multiset#1#2{\ensuremath{\left(\kern-.3em\left(\genfrac{}{}{0pt}{}{#1}{#2}\right)\kern-.3em\right)}}

\title{Discrete Mathematics, 2016 Spring - HW 10}
\author{Instructor: Zsolt Pajor-Gyulai}
\institute{Courant Institute of Mathematical Sciences, NYU}



\begin{document}

\maketitle

To get full credit  in all of the problems, use rigorous justification and unless otherwise indicated, make sure that your solution reads as a perfect English sentence. You should only assume integers, operations and order relations as given. If you use a statement or a definition from the textbook, make sure to indicate it.
\vspace{0.2cm}

\textbf{Section 35}
\enumb
\item[8)] Prove Proposition 35.8 in the textbook.
\item[9)] Prove that the sum of any three consequtive integers is divisible by $3$.
%\item[12)] \textit{Polynomial division} The degree of a polynomial is the exponent of the highest power of $x$. For example, $x^{10}-5x^2+6$ has degree $10$, and the degree of $3x-\frac{1}{2}$ is $1$. When the polynomial is a constant (no $x$-terms), we say the degree is $0$, except when it is identically zero, in which case we say that its degree is $-\infty$. If $p$ is a polynomial, we write $\textrm{deg}(p)$ to stand for its degree.
%\enumb
%\item Suppose $p$ and $q$ are polynomials. Make up a careful definition of what it means for $p$ to divide $q$ (i.e. p|q). Verify that
%\[
%(2x-6)|(x^3-3x^2+3x-9)
%\]
%is true using your definition.
%\item Give an example of two polynomials $p$ and $q$ with $p\neq q$ but $p|q$ and $q|p$. What is the relationship between polynomials that divide each other? 
%\item Prove the following analogue of the division with remainder theorem:
%\begin{proposition}
%Let $a$ and $b$ be polynomials, with $b$ nonzero. Then there exist polynomials $q$ and $r$ so that $a=qb+r$ with $deg(r)<deg(b)$. For example if $a=x^5-3x^2+2x+1$ and $b=x^2+1$, then we can take $q=x^3-x-3$ and $r=3x+4$.
%\end{proposition}
%\enume
\enume

\textbf{Section 36}

\enumb

\item[1-2)] Please calculate:
\enumb
\item $gcd(54321,50)$.
\item $gcd(1739,-29341)$.
\enume
In each case, find the integers $x$ and $y$ such that $ax+by=gcd(a,b)$.
\item[11)] Prove that consecutive integers must be relatively prime.
\item[15)] Suppose that $a$ and $b$ are relatively prime integers and that $a|c$ and $b|c$. Prove that $(ab)|c$.
\item[21)] You have two measuring cups. One holds $8$ ounces and the other holds $13$ ounces. These cups have no marks to show individual ounces. All you can measure is either a full $13$ or a full $8$ ounces. If you want to measure, say, 5 ounces, you can fill the 13-ounce measuring cup, use it to fill the 8-ounce cup, and you will have $5$ ounces left in the larger cup.
\enumb
\item Show how to measure exactly $1$ ounce. You may assume you have a large bowl for holding liquid, but this large bowl has no marks for measuring. At the end, the bowl should contain exactly one ounce.
\item Generalize this problem. Suppose you have $a$ and $b$ ounces where $a$ and $b$ are positive integers. Give a necessary and sufficient conditions on $a$ and $b$ such that it is possible to measure out exactly $1$ ounce using these cups. (Hint: Think how this problem relates to the lecture in the first place.)
\item Watch \verb https://www.youtube.com/watch?v=BVtQNK_ZUJg ~and be proud that you're better than John McClane.
\enume
\enume


\end{document}