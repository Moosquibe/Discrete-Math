\documentclass[11pt]{preprint}

\setlength{\topmargin}{0mm} \setlength{\oddsidemargin}{0mm}
\setlength{\textwidth}{160mm} \setlength{\textheight}{215mm}

\usepackage{amssymb,amsmath,amscd,amsthm}
\usepackage{tikz}

\newtheorem{proposition}{Proposition}

\def\enumb{\begin{enumerate}}
\def\enume{\end{enumerate}}
\def\integers{\mathbb{Z}}
\def\multiset#1#2{\ensuremath{\left(\kern-.3em\left(\genfrac{}{}{0pt}{}{#1}{#2}\right)\kern-.3em\right)}}

\title{Discrete Mathematics, 2016 Spring - HW 8}
\author{Instructor: Zsolt Pajor-Gyulai}
\institute{Courant Institute of Mathematical Sciences, NYU}



\begin{document}

\maketitle

To get full credit  in all of the problems, use rigorous justification and unless otherwise indicated, make sure that your solution reads as a perfect English sentence. You should only assume integers, operations and order relations as given. If you use a statement or a definition from the textbook, make sure to indicate it.
\vspace{0.2cm}

\textbf{Section 24}
\enumb
%\item[Prop 24.15)] Prove the following proposition.
%\begin{proposition}
%Let $f$ be a function and suppose $f^{-1}$ is also a function. Then $\textrm{dom} f=\textrm{im} f^{-1}$ and $\textrm{im} f=\textrm{dom} f^{-1}$.
%\end{proposition}\vspace{-0.2cm}
\item [10-11)] Let $a,b,c$ real numbers.
\enumb
\item Consider the function $f:\mathbb{R}\to\mathbb{R}$ defined by $f(x)=ax+b$. For which values of $a$ and $b$ is $f$ one-to-one? ... onto $\mathbb{R}$?
\item Consider the function $f:\mathbb{R}\to\mathbb{R}$ defined by $f(x)=ax^2+bx+c$. For which values of $a$,$b$,$c$ is $f$ one-to-one?...onto $\mathbb{R}$?
\enume
%\item[14)] For each of the following, determine whether the function is one-to-one, onto, or both. Prove your assertions.
%\enumb
%\item $f:\mathbb{Z}\to\mathbb{Z}$ defined by $f(x)=10+x$.
%\item $f:\mathbb{Z}\to\mathbb{Z}$ defined by
%\[
%f(x)=\left\{\begin{array}{cc}
%\frac{x}{2}&\textrm{if $x$ is even}\\
%\frac{x-1}{2}&\textrm{if $x$ is odd}
%\end{array}\right.
%\]
%\item $f:\mathbb{R}-\mathbb{Q}\to\mathbb{R}-\mathbb{Q}$ defined by $f(x)=x^2$.
%\enume
%\item [Thm 24.21)] Read the proof of Theorem 24.21 on p173 from the textbook.
%\item[18)] Suppose $f:A\to B$ is a bijection. Prove that $f^{-1}:B\to A$ is a bijection as well.
\item[23)] Let $f:A\to B$ be a function. This induces a function (denoted again by $f$ with a big abuse of notation) $f:2^A\to 2^B$ the following way. For $x\subseteq A$, define
\[
f(X)=\{f(x):x\in X\},
\]
the set of all values $f$ takes when applied to elements of $X$. For example if $f:\mathbb{Z}\to\mathbb{Z}$ is defined by $f(x)=x^2$ and $X=\{1,3,5\}$ then $f(X)=\{1^2,3^2,5^2\}=\{1,9,25\}$. This is called the \textbf{image of $X$ under $f$}. In all the following examples, find $f(X)$.
\enumb
\item $f:\mathbb{Z}\to\mathbb{Z}$ by $f(x)=|x|$, and $X=\{-1,0,1,2\}$.
\item $f:\mathbb{R}\to\mathbb{R}$ by $f(x)=2^x$, and $X=[-1,1]$.
\item $f:\mathbb{R}\to\mathbb{R}$ by $f(x)=\sin x$, and $X=[0,\pi]$.
\enume
\item[24)] In the same spirit as the previous problem we can define the \textbf{inverse image of a set $Y\subseteq X$ under $f$}. Formally, if $f:A\to B$ is a function and $Y\subseteq B$, then define
\[
f^{-1}(Y)=\{x\in A:f(x)\in Y\},
\]
or in other words the set of all elements of $A$ that are mapped into a value in $Y$. For example suppose $f:\mathbb{Z}\to\mathbb{Z}$, $f(x)=x^2$ and $Y=\{4,9\}$, then $f^{-1}(Y)=\{-3,-2,2,3\}$. In the following examples, find $f^{-1}(Y)$.
\enumb
\item $f:\mathbb{Z}\to\mathbb{Z}$ by $f(x)=|x|$, and $Y=\{1,2,3\}$.
\item $f:\mathbb{R}\to\mathbb{R}$ by $f(x)=x^2$, and $Y=\{-2,3,4\}$.
\item $f:\mathbb{R}\to\mathbb{R}$ by $f(x)=1/(1+x^2)$, and $Y=\{-1/2\}$.
\enume
\enume

\enumb
%\item[N/A)] Show that the number of bijections $f:A\to B$ is $n!$ if $|A|=|B|=n$ and $0$ if $|A|\neq |B|$.
\item[19)]
Let $n$ be a positive integer. Let $A_n$ be the set of positive divisors of $n$ that are less than $\sqrt{n}$ and let $B_n$ be the set of positive divisors of $n$ that are greater than $\sqrt{n}$. That is:
\[
A_n=\{d\in\mathbb{N}:d|n,d<\sqrt{n}\},\qquad B_n=\{d\in\mathbb{N}:d|n,d>\sqrt{n}\}.
\]
For example when $n=24$, then $\sqrt{24}\approx 4.899$ and so $A_{24}=\{1,2,3,4\}$ and $B_{24}=\{6,8,12,24\}$.
\enumb
\item Find a bijection $f:A_n\to B_n$. This implies that $|A_n|=|B_n|$.
\item Prove that a positive integer has an odd number of positive divisors if and only if $n$ is a perfect square. (Hint: Perfect squares have a divisor that is not in $A_n\cup B_n$).
\enume

\item[21)] Let $f$ be a function. We say that $f$ is two-to-one provided for each $b\in\textrm{im}f$ there are exactly two elements $a_1,a_2\in\textrm{dom}f$ such that $f(a_1)=f(a_2)=b$. For a positive integer $n$, let $A$ be a $2n$ element set and $B$ be an $n$-element set. How many functions $f:A\to B$ are two-to-one?

%\item[22)] Let $A$ be an $n$-element set and let $i,j,k\in\mathbb{N}$ with $i+j+k=n$. How many functions $f:A\to\{0,1,2\}$ are there for which there are exactly $i$ elements $a\in A$ with $f(a)=0$, exactly $j$ elements $a\in A$ with $f(a)=1$, and exactly $k$ elements $a\in A$ with $f(a)=2$?
\enume

\textbf{Section 25}
\enumb
\item[3)] How large a group of people do we need to consider to be certain that three members of the group have the same initials (first, middle, last)?
\item[12)]
\enumb
\item Read the discussion before and proof of Theorem 25.3 on p179-180.
\item Find a sequence of nine distinct integers that does not contain a monotone subsequence of length four. Generalize your construction by showing how to construct (for every positive integer $n$) a sequence of $n^2$ distinct integers that does not contain a monotone subsequence of length $n+1$. (Use the hint at the back of the textbook)
\enume
\enume




\end{document}