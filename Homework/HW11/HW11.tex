\documentclass[11pt]{preprint}

\setlength{\topmargin}{0mm} \setlength{\oddsidemargin}{0mm}
\setlength{\textwidth}{160mm} \setlength{\textheight}{215mm}

\usepackage{amssymb,amsmath,amscd,amsthm}
\usepackage{tikz}

\newtheorem{proposition}{Proposition}

\def\enumb{\begin{enumerate}}
\def\enume{\end{enumerate}}
\def\integers{\mathbb{Z}}
\def\multiset#1#2{\ensuremath{\left(\kern-.3em\left(\genfrac{}{}{0pt}{}{#1}{#2}\right)\kern-.3em\right)}}

\title{Discrete Mathematics, 2016 Spring - HW 11}
\author{Instructor: Zsolt Pajor-Gyulai}
\institute{Courant Institute of Mathematical Sciences, NYU}



\begin{document}

\maketitle

To get full credit  in all of the problems, use rigorous justification and unless otherwise indicated, make sure that your solution reads as a perfect English sentence. You should only assume integers, operations and order relations as given. If you use a statement or a definition from the textbook, make sure to indicate it.
\vspace{0.2cm}


\textbf{Section 37}
\enumb
\item[8)] Prove Proposition $37.4$ in the textbook. Why is this proposition restricted to $n\geq 2$?
\item[3)] Solve the following equations for $x$ in the $\mathbb{Z}_n$ specified. Make sure to find all solutions.
\enumb
\item $342\otimes x=73$ in $\mathbb{Z}_{1003}$.
\item $9\otimes x=4$ in $\mathbb{Z}_{12}$.
\enume
\item[6)] Prove that for all $a,b\in\mathbb{Z}_n$, $(a\ominus b)\oplus (b\ominus a)=0$.
\item[10)] For ordinary integers, the following is true. If $ab=0$, then $a=0$ or $b=0$. Note, however that in $\mathbb{Z}_n$,
\[
2\otimes 5=0,\qquad \textrm{but}\qquad 2\neq 0, 5\neq 0.
\]
For which values of $n\geq 2$ does the implication
\[
a\otimes b=0\Leftrightarrow a=0\textrm{ or }b=0
\]
hold in $\mathbb{Z}_n$?
\enume

\textbf{Section 38}

\enumb
\item[1)] Solve $100x\equiv 74$ (mod 127). Make sure to find all solutions.
\item[3)] Solve the following system of congruences
\[
3x\equiv 8~(10)\qquad 2x+4\equiv 9~(11)
\]

\enume
\newpage
\textbf{Section 39}

\enumb
\item[10)] Let $a$ and $b$ be integers. A \textit{common multiple} of $a$ and $b$ is an integer $n$ for which $a|n$ and $b|n$. We call an integer $m$ the \textit{least common multiple} of $n$ provided $(1)$ $m$ is positive, $(2)$ $m$ is a common multiple of $a$ and $b$, and $(3)$ if $n$ is any other positive common multiple of $a$ and $b$, then $n\geq m$. For example, $lcm(24,30)=120$.
\enumb
\item Develop a formula for the least common multiple of two positive integers in terms of their prime factorization.
\item Use your formula to show that if $a$ and $b$ are positive integers, then
\[
ab=gcd(a,b)lcm(a,b)
\]
\enume
\item[14)] Let $n$ be a positive integer and suppose we factor $n$ into primes as follows:
\[
n=p_1^{e_1}p_2^{e_2}\dots p_t^{e_t},
\]
where the $p_j$-s are distinct primes and the $e_j$-s are natural numbers. 
\enumb
\item Find a formula for the number of positive divisors of $n$.
\item Recall that an integer $n$ is called \textit{perfect} if it equals the sum of all its divisors $d$ with $1\leq d<n$. For example $28$ is perfect as $28=1+2+4+7+14$. Let $a$ be a positive integer and prove that if $2^a-1$ is prime, then $n=2^{a-1}(2^a-1)$ is perfect.
\enume
\item[22)] Prove that $\log_2 3$ is irrational.
\enume


\end{document}